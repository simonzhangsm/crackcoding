\subsubsection{内容简介}

本书包含了LeetCode Online Judge(http://leetcode.com/onlinejudge) 
所有题目及答案,部分包含Careerup(http://www.careerup.com)面试题目,所有
代码经过精心编写,使用C++ + STL风格,编码规范良好,适合读者反复揣摩模仿,甚至在纸上默写。

\begindot
\item 经常使用全局变量。比如用几个全局变量,定义某个递归函数需要的数据,减少递归函数的参数
个数,就减少了递归时栈内存的消耗,可以说这几个全局变量是这个递归函数的环境。

\item Shorter is better。能递归则一定不用栈;能用STL 则一定不自己实现。

\item 不提倡防御式编程。如不需要检查malloc/new返回的指针是否为NULL;不检查内部函数入口
参数的有效性;基于C++11对象编程时,调用对象的成员方法,不需要检查对象自身是否为NULL;不使用Try/Catch/Throw异常处理机制。
\myenddot

\subsubsection{Github地址}
开源项目地址:\myurl{https://github.com/simonzhangsm/crackcoding}

\subsubsection{网络资源}
\begindot
\item Leetcode:\myurl{https://www.leetcode.com}
\item LintCode:\myurl{http://lintcode.com/zh-cn/daily}
\item CodeEval:\myurl{https://www.codeeval.com}
\item TopCoder:\myurl{https://www.topcoder.com}
\item HackerRank:\myurl{https://www.hackerrank.com}
\item ACM Home:\myurl{http://www.acmerblog.com}
\item 结构之法 算法之道博客:\myurl{http://blog.csdn.net/v_JULY_v}
\item Princeton Algorithm:\myurl{http://algs4.cs.princeton.edu/home}
\myenddot

