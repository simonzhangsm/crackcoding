\chapter{ACM基础知识}
\section{方程组}

$AX=B, A=\begin{bmatrix}
	a_{1,1} & a_{1,2} & \cdots & a_{1,n} \\
	a_{2,1} & a_{2,2} & \cdots & a_{2,n} \\
	\vdots  & \vdots  & \ddots & \vdots  \\
	a_{m,1} & a_{m,2} & \cdots & a_{m,n}
\end{bmatrix}, 
X=\begin{bmatrix}
x_{1}  \\
x_{2} \\
\vdots   \\
x_{m}
\end{bmatrix}, 
B=\begin{bmatrix}
b_{1}  \\
b_{2} \\
\vdots   \\
b_{m}
\end{bmatrix}$

高斯消元法:

克莱姆法则:$AX=B$,$x_i=\frac{|A_i|}{|A|}$

\section{多项式}
$f(x)=\sum_{i=0}^n{a_ix^i}$

【性质】
1. $(1+x)^n=\sum_{i=0}^n{C_n^ix^i}$

2. n个点可以唯一确定一个n次多项式;

3. 一个n次多项式有n个复数根

【算法】

用牛顿迭代法求解方程的根:

选择一个靠近多项式$f(x)$ 零点的点$x_0$作为迭代初值,计算$f(x_0)$和$f^{'}(x_0)$,解方程:

$$x\cdot f^{'}(x_0)+f(x_0)-x_0\cdot f^{'}(x_0)=0$$

解得$x_1$,如此反复迭代,可求出多项式的一个根。

\section{快速傅里叶变换FFT}
快速傅里叶变换主要用于求两个多项式的乘积,即给定两个阶小于n的多项式$f_{A}(x)=\sum_{i=0}^{n-1}{a_ix^i},f_{B}(x)=\sum_{i=0}^{n-1}{b_ix^i}$,求解$f_{c}(x)=\sum_{i=0}^{2n-2}{c_ix^i}=f_{a}(x)f_{b}(x)$.
 FFT求解可以$O(n\log(n))$.

\begin{Code}
	VOID WINAPI FFT(complex<double> * TD, complex<double> * FD, int r)  {  
		LONG count;  // 付立叶变换点数
		int i,j,k;  // 循环变量
		int bfsize,p;  // 中间变量  
		double angle;// 角度  
		complex<double> *W,*X1,*X2,*X;  
		count = 1 << r;  // 计算付立叶变换点数
		// 分配运算所需存储器  
		W = new complex<double>[count / 2];  
		X1 = new complex<double>[count];  
		X2 = new complex<double>[count];  
		// 计算加权系数  
		for(i = 0; i < count / 2; i++)  {  
			angle = -i * PI * 2 / count;  
			W[i] = complex<double> (cos(angle), sin(angle));  
		}  
		// 将时域点写入X1  
		memcpy(X1, TD, sizeof(complex<double>) * count);  
		for(k = 0; k < r; k++) {  // 采用蝶形算法进行快速付立叶变换 
			for(j = 0; j < 1 << k; j++)  {  
				bfsize = 1 << (r-k);  
				for(i = 0; i < bfsize / 2; i++)  {  
					p = j * bfsize;  
					X2[i + p] = X1[i + p] + X1[i + p + bfsize / 2];  
					X2[i + p + bfsize / 2] = (X1[i + p] - X1[i + p + bfsize / 2]) * W[i * (1<<k)];  
				}  
			}  
			X = X1;  
			X1 = X2;  
			X2 = X;  
		}  
		
		// 重新排序  
		for(j = 0; j < count; j++)  {  
			p = 0;  
			for(i = 0; i < r; i++)  {  
				if (j&(1<<i))  
					p+=1<<(r-i-1);  
			}  
			FD[j]=X1[p];  
		}  	
		// 释放内存  
		delete W;  
		delete X1;  
		delete X2;  
	}  
\end{Code}

\section{随机化}
随机化是一类不确定但能以极大概率正确的算法。如拉斯维加斯算法和蒙特卡洛算法。

蒙特卡洛算法:

蒙特卡罗法(Monte Carlo method)是以概率和统计的理论、方法为基础的一种计算方法,将所求解的问题同一定的概率模型相联系,用电子计算机实现统计模拟或抽样,以获得问题的近似解,故又称统计模拟法或统计试验法。

蒙特卡罗算法在一般情况下可以保证对问题的所有实例都以高概率给出正确解,但是通常无法判定一个具体解是否正确。

设p是一个实数,且1/2 <p <1。如果一个蒙特卡罗算法对于问题的任一实例得到正确解的概率不小于p,则称该蒙特卡罗算法是p正确的,且称p – 
1/2是该算法的优势。如果对于同一实例,蒙特卡罗算法不会给出2个不同的正确解答,则称该蒙特卡罗算法是一致的。

拉斯维加斯算法:

拉斯维加斯算法的一个显著特征是它所作的随机性决策有可能导致算法找不到所需的解。因此通常用一个bool型函数表示拉斯维加斯算法。

【验证矩阵乘法的正确性】
【验证两个函数是否等价】