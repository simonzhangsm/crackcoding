\chapter{Appendex}

\section{註記} %%%%%%%%%%%%%%%%%%%%%%%%%%%%%%

\subsection{dungeon-game}
動態規劃,對於每個格子記錄從它出發順利到達終點需要的最小血量。從(m-1,n-1)遞推至(0,0)。
注意對於每個格子其實有進和出兩個狀態:進入該格子時(未考慮當前格子影響);和離開該格子後(已考慮當前格子影響)。這兩種方法都行,考慮到程序的結構可能有三部分:初始化邊界條件、狀態轉移、根據狀態表示答案,需要仔細斟酌以追求代碼簡短。這裏兩種方式並無明顯代碼長度差異。
\subsection{majority-element}
Boyer-Moore majority vote algorithm http://www.cs.utexas.edu/~moore/best-ideas/mjrty/ 
這個可以拓展爲找出所有出現次數大於N/K的元素。方法是每次找出K個互不相同的元素丟掉,最後剩下的元素是候選解。時間複雜度O(N*K)。
\subsection{fraction-to-recurring-decimal}

注意INT_MIN/(-1)會溢出。
maximum-gap
平均数原理,求出极差d后,根据桶大小ceil(d/(n-1))分成若干个桶,答案必为不同桶的两个元素之差。
find-peak-element
二分查找,把區間縮小爲仍包含候選值的子區間。
\begin{Code}
	int l = 0, h = a.size();
	while (l < h-1) {
		int m = l+h >> 1;
		if (a[m-1] > a[m]) h = m;
		else if (m+1 == h || a[m] > a[m+1]) l = h = m;
		else l = m+1;
	}
	return l;
\end{Code}
\subsection{min-stack}
使用兩個棧,原棧S存放各個元素。每當新元素小於等於當前最小值時,就把新元素複製一份壓入另一個棧S'。彈出時,若元素在S'中出現,則也從S'中彈出。
另一種方法是記錄新元素與當前最小值的差值,每個元素需要多記錄1個bit。可惜C++會Memory Limit Exceeded,感覺不合理。
\subsection{find-minimum-in-rotated-sorted-array}
方法是二分,如果要寫短可以採用下面的思路: 子數組中如果有相鄰兩個元素a[i]>a[i+1],則a[i+1]是整個數組的最小值。 若子數組a[i..j]滿足a[i]>a[j],則存在i<=k<j使得a[k]>a[k+1]。 
對於子數組a[l..h]判斷a[l]>a[m]或a[m]>a[h]是否有一個成立,成立則可以去除一半候選元素,不然a[h]>a[l]爲一個逆序對。
\subsection{find-minimum-in-rotated-sorted-array-ii}
對於子數組a[l..h]判斷a[l]>a[m]或a[m]>a[h]是否有一個成立,成立則可以去除一半候選元素。 兩個條件均不滿足則a[l]<=a[m]<=a[h],若a[h]>a[l]則找到逆序對,否則a[l]=a[m]=a[h],可以把範圍縮小爲a[l+1..h-1]
\subsection{first-missing-positive}
空間複雜度O(1)。
\subsection{jump-game}
空間複雜度O(1)。
\subsection{jump-game-ii}
使用output-restricted queue優化的動態規劃,注意到動態規劃值的單增性可以用類似BFS的方式,空間複雜度O(1)。
\subsection{linked-list-cycle-ii}
Brent's cycle detection algorithm
\subsection{longest-palindromic-substring}
Manacher's algorithm
\subsection{maximum-subarray}
Kadane's algorithm
\subsection{merge-k-sorted-lists}
Tournament sort
\subsection{minimal-window-substring}
尺取法
\subsection{n-queen}
bitmask存儲列,正反斜線控制的格子。
\subsection{palindrome-partitioning}
f[i][j] = calc(f[ii][jj] : i <= ii <= jj <= j)形式的動態規劃可以採用如下計算方式。
\begin{Code}
	for (int i = n; --i >= 0; )
	for (int j = i; j < n; j++) {
		// calc [i,j]
	}
\end{Code}

\subsection{recover-binary-search-tree}
使用Morris in-order traversal找到鄰接失序對。
如果交換的元素相鄰,則有一個鄰接失序對(如0 1 2 3 4 ->0 1 3 2 4),否則有兩個鄰接失序對(如0 1 2 3 4 5 -> 0 4 2 3 1 5)。
\subsection{remove-nth-node-from-end-of-list}
http://meta.slashdot.org/story/12/10/11/0030249/linus-torvalds-answers-your-questions
使用pointers-to-pointers很多時候能簡化實現。
\subsection{regular-expression-matching}
P爲模式串,T爲文本, f[i][j]表示P的前i個字符能否匹配T的前j個字符。 根據f[i-1][*]計算f[i][*]。 這個方法也可以看作構建了精簡表示的Thompson's automaton,時間複雜度O(|P|*|T|)。
\subsection{sort-colors}
Dutch national flag problem 如果不要求000111222,允許111000222111,那麼有交換次數更少的Bentley-McIlroy算法 http://www.iis.sinica.edu.tw/~scm/ncs/2010/10/dutch-national-flag-problem-3/
\subsection{sqrtx}
Hacker's Delight (2nd) 11.1.1 46340 = floor(sqrt(INT_MAX))
\subsection{scramble-string}
我用了$O(n^3)$的空間和$O(n^4)$的時間,應該有更好的算法。
\subsection{sort-list}
Natural merge sort
\subsection{sudoku-solver}
轉化爲exact cover problem,使用dancing links + Algorithm X求解。
\subsection{trapping-rain-water}
兩個指針夾逼,空間O(1)。
\subsection{unique-binary-search-trees-ii}
子樹可以復用。
\subsection{maximal-rectangle}
秋葉拓哉(iwi)、巖田陽一(wata)和北川宜稔(kita_masa)所著,巫澤俊(watashi)、莊俊元(navi)和李津羽(itsuhane)翻譯的《挑戰程序設計競賽》
逐行掃描棋盤,掃描第i行時,h[j]表示第j列上方連續爲1的行數。若a[i][j]==0則h[j]==0, 當a[i][j]==1時,l[j]表示min(k : h[k+1..j]>=h[j]),r[j]表示max(k : h[j..k-1]>=h[j])。
\begin{Code}
	#define ROF(i, a, b) for (int i = (b); --i >= (a); )
	#define FOR(i, a, b) for (int i = (a); i < (b); i++)
	#define REP(i, n) for (int i = 0; i < (n); i++)
	class Solution {
		public:
		int maximalRectangle(vector<vector<char> > &a) {
			if (a.empty()) return 0;
			int m = a.size(), n = a[0].size(), ans = 0;
			vector<int> h(n), l(n), r(n, n-1);
			REP(i, m) {
				int ll = -1;
				REP(j, n) {
					h[j] = a[i][j] == '1' ? h[j]+1 : 0;
					if (a[i][j] == '0') ll = j;
					l[j] = h[j] ? max(h[j] == 1 ? 0 : l[j], ll+1) : j;
				}
				int rr = n;
				ROF(j, 0, n) {
					if (a[i][j] == '0') rr = j;
					r[j] = h[j] ? min(h[j] == 1 ? n-1 : r[j], rr-1) : j;
					ans = max(ans, (r[j]-l[j]+1)*h[j]);
				}
			}
			return ans;
		}
	};
\end{Code}

\subsection{4-sum}
使用了multimap,時間複雜度$O(n^2*log(n))$,得到所有答案後不需要去重操作。
合法的四元組有三類:
a<=b<c<=d,枚舉c和d,判斷是否存在和爲target-c-d的二元組
a<=b=c<=d,枚舉b和d,判斷是否存在target-b-b-d
a=b=c<=d,枚舉a,判斷是否存在target-a-a-a
分別統計,小心實現可以保證不會產生相同的候選解,從而無需去重。
\begin{Code}
	#define ROF(i, a, b) for (int i = (b); --i >= (a); )
	#define FOR(i, a, b) for (int i = (a); i < (b); i++)
	#define REP(i, n) for (int i = 0; i < (n); i++)
	class Solution {
		public:
		int maximalRectangle(vector<vector<char> > &a) {
			if (a.empty()) return 0;
			int m = a.size(), n = a[0].size(), ans = 0;
			vector<int> h(n), l(n), r(n, n-1);
			REP(i, m) {
				REP(j, n) {
					h[j] = a[i][j] == '1' ? h[j]+1 : 0;
					l[j] = j;
					while (l[j] && h[l[j]-1] >= h[j])
					l[j] = l[l[j]-1];
				}
				ROF(j, 0, n) {
					r[j] = j;
					while (r[j]+1 < n && h[j] <= h[r[j]+1])
					r[j] = r[r[j]+1];
					ans = max(ans, (r[j]-l[j]+1)*h[j]);
				}
			}
			return ans;
		}
	};
\end{Code}

\subsection{ACRush某TopCoder SRM}
計數排序h[*],從小到大插入各個指標
大意是這一行有存在若干個區間,對於一個區間[L,R],需要保證peer[L] == R \&\& peer[R] == L 插入指標時獲取所在區間的左右端點並更新peer[*]
\begin{Code}
	#define ROF(i, a, b) for (int i = (b); --i >= (a); )
	#define FOR(i, a, b) for (int i = (a); i < (b); i++)
	#define REP(i, n) for (int i = 0; i < (n); i++)
	#define REP1(i, n) for (int i = 1; i <= (n); i++)
	class Solution {
		public:
		int maximalRectangle(vector<vector<char> > &a) {
			if (a.empty()) return 0;
			int m = a.size(), n = a[0].size(), ans = 0;
			vector<int> h(n), p(n), b(m+1), s(n);
			REP(i, m) {
				REP(j, n)
				h[j] = a[i][j] == '1' ? h[j]+1 : 0;
				fill(b.begin(), b.end(), 0);
				REP(j, n)
				b[h[j]]++;
				REP1(j, m)
				b[j] += b[j-1];
				REP(j, n)
				s[--b[h[j]]] = j;
				fill(p.begin(), p.end(), -1);
				ROF(j, 0, n) {
					int x = s[j], l = x, r = x;
					p[x] = x;
					if (x && p[x-1] != -1) {
						l = p[x-1];
						p[l] = x;
						p[x] = l;
					}
					if (x+1 < n && p[x+1] != -1) {
						l = p[x];
						r = p[x+1];
						p[l] = r;
						p[r] = l;
					}
					ans = max(ans, (r-l+1)*h[x]);
				}
			}
			return ans;
		}
	};
\end{Code}

\subsection{棧維護histogram}
常見解法。