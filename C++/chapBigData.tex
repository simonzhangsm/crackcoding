\chapter{Big Data}
\section{BigURL}
You have a file with 100 billion URLS, find first unique URL.

\section{海量日志}
首先是这一天,并且是访问百度的日志中的IP取出来,逐个写入到一个大文件中。注意到IP是32位的,最多有个$2^32$个IP。同样可以采用映射的方法,比如模1000,把整个大文件映射为1000个小文件,再找出每个小文中出现频率最大的IP(可以采用hash_map进行频率统计,然后再找出频率最大的几个)及相应的频率。然后再在这1000个最大的IP中,找出那个频率最大的IP,即为所求。

算法思想:分而治之+Hash
1.IP地址最多有$2^32$=4G种取值情况,所以不能完全加载到内存中处理; 
2.可以考虑采用“分而治之”的思想,按照IP地址的Hash(IP)\%1024值,把海量IP日志分别存储到1024个小文件中。这样,每个小文件最多包含4MB个IP地址; 
3.对于每一个小文件,可以构建一个IP为key,出现次数为value的Hash map,同时记录当前出现次数最多的那个IP地址;
4.可以得到1024个小文件中的出现次数最多的IP,再依据常规的排序算法得到总体上出现次数最多的IP;

\section{检索串}
假设目前有一千万个记录(这些查询串的重复度比较高,虽然总数是1千万,但如果除去重复后,不超过3百万个。一个查询串的重复度越高,说明查询它的用户越多,也就是越热门。),请你统计最热门的10个查询串,要求使用的内存不能超过1G。

典型的Top K算法,还是在这篇文章里头有所阐述,详情请参见:十一、从头到尾彻底解析Hash表算法。

文中,给出的最终算法是:\\
第一步、先对这批海量数据预处理,在O(N)的时间内用Hash表完成统计(之前写成了排序,特此订正。July、2011.04.27);\\
第二步、借助堆这个数据结构,找出Top K,时间复杂度为$NlogK$。\\
即,借助堆结构,我们可以在log量级的时间内查找和调整/移动。因此,维护一个K(该题目中是10)大小的小根堆,然后遍历300万的Query,分别和根元素进行对比所以,我们最终的时间复杂度是:$O(N) + 
N'*O(logK)$,(N为1000万,N’为300万)。

或者:采用trie树,关键字域存该查询串出现的次数,没有出现为0。最后用10个元素的最小推来对出现频率进行排序。

\section{Top100高频词汇}
有一个1G大小的一个文件,里面每一行是一个词,词的大小不超过16字节,内存限制大小是1M。返回频数最高的100个词。

方案:顺序读文件中,对于每个词x,取hash(x)\%5000,然后按照该值存到5000个小文件(记为x0,x1,...x4999)中。这样每个文件大概是200k左右。

如果其中的有的文件超过了1M大小,还可以按照类似的方法继续往下分,直到分解得到的小文件的大小都不超过1M。
对每个小文件,统计每个文件中出现的词以及相应的频率(可以采用trie树/hash_map等),并取出出现频率最大的100个词(可以用含100个结点的最小堆),并把100个词及相应的频率存入文件,这样又得到了5000个文件。下一步就是把这5000个文件进行归并(类似与归并排序)的过程了。

\section{query的频度排序}
有10个文件,每个文件1G,每个文件的每一行存放的都是用户的query,每个文件的query都可能重复。要求你按照query的频度排序。

还是典型的TOP K算法,解决方案如下:
方案1:
顺序读取10个文件,按照hash(query)\%10的结果将query写入到另外10个文件(记为)中。这样新生成的文件每个的大小大约也1G(假设hash函数是随机的)。

找一台内存在2G左右的机器,依次对用hash_map(query, 
query_count)来统计每个query出现的次数。利用快速/堆/归并排序按照出现次数进行排序。将排序好的query和对应的query_cout输出到文件中。这样得到了10个排好序的文件(记为)。

对这10个文件进行归并排序(内排序与外排序相结合)。

方案2:
一般query的总量是有限的,只是重复的次数比较多而已,可能对于所有的query,一次性就可以加入到内存了。这样,我们就可以采用trie树/hash_map等直接来统计每个query出现的次数,然后按出现次数做快速/堆/归并排序就可以了。

方案3:
与方案1类似,但在做完hash,分成多个文件后,可以交给多个文件来处理,采用分布式的架构来处理(比如MapReduce),最后再进行合并。

\section{共同的url}
给定a、b两个文件,各存放50亿个url,每个url各占64字节,内存限制是4G,让你找出a、b文件共同的url?

方案1:可以估计每个文件安的大小为5G×64=320G,远远大于内存限制的4G。所以不可能将其完全加载到内存中处理。考虑采取分而治之的方法。

遍历文件a,对每个url求取hash(url)\%1000,然后根据所取得的值将url分别存储到1000个小文件(记为a0,a1,...,a999)中。这样每个小文件的大约为300M。

遍历文件b,采取和a相同的方式将url分别存储到1000小文件(记为b0,b1,...,b999)。这样处理后,所有可能相同的url都在对应的小文件(a0vsb0,a1vsb1,...,a999vsb999)中,不对应的小文件不可能有相同的url。然后我们只要求出1000对小文件中相同的url即可。

求每对小文件中相同的url时,可以把其中一个小文件的url存储到hash_set中。然后遍历另一个小文件的每个url,看其是否在刚才构建的hash_set中,如果是,那么就是共同的url,存到文件里面就可以了。

方案2:如果允许有一定的错误率,可以使用Bloom filter,4G内存大概可以表示340亿bit。将其中一个文件中的url使用Bloom filter映射为这340亿bit,然后挨个读取另外一个文件的url,检查是否与Bloom 
filter,如果是,那么该url应该是共同的url(注意会有一定的错误率)。

Bloom filter日后会在本BLOG内详细阐述。

\section{大数据中的重复整数}
在2.5亿个整数中找出不重复的整数,内存不足以容纳这2.5亿个整数。

方案1:采用2-Bitmap(每个数分配2bit,00表示不存在,01表示出现一次,10表示多次,11无意义)进行,共需内存$2^32 * 2 bit=1 
GB$内存,还可以接受。然后扫描这2.5亿个整数,查看Bitmap中相对应位,如果是00变01,01变10,10保持不变。所描完事后,查看bitmap,把对应位是01的整数输出即可。

方案2:也可采用与第1题类似的方法,进行划分小文件的方法。然后在小文件中找出不重复的整数,并排序。然后再进行归并,注意去除重复的元素。

\section{点查找数据}
给40亿个不重复的unsigned int的整数,没排过序的,然后再给一个数,如何快速判断这个数是否在那40亿个数当中?

与上第6题类似,我的第一反应时快速排序+二分查找。以下是其它更好的方法:
方案1:oo,申请512M的内存,一个bit位代表一个unsigned int值。读入40亿个数,设置相应的bit位,读入要查询的数,查看相应bit位是否为1,为1表示存在,为0表示不存在。

dizengrong:
方案2:这个问题在《编程珠玑》里有很好的描述,大家可以参考下面的思路,探讨一下:
又因为$2^32$为40亿多,所以给定一个数可能在,也可能不在其中;
这里我们把40亿个数中的每一个用32位的二进制来表示
假设这40亿个数开始放在一个文件中。

然后将这40亿个数分成两类:
1.最高位为0
2.最高位为1
并将这两类分别写入到两个文件中,其中一个文件中数的个数<=20亿,而另一个>=20亿(这相当于折半了);
与要查找的数的最高位比较并接着进入相应的文件再查找

再然后把这个文件为又分成两类:
1.次最高位为0
2.次最高位为1

并将这两类分别写入到两个文件中,其中一个文件中数的个数<=10亿,而另一个>=10亿(这相当于折半了);
与要查找的数的次最高位比较并接着进入相应的文件再查找。
.......
以此类推,就可以找到了,而且时间复杂度为O(logn),方案2完。

附:这里,再简单介绍下,位图方法:
使用位图法判断整形数组是否存在重复 
判断集合中存在重复是常见编程任务之一,当集合中数据量比较大时我们通常希望少进行几次扫描,这时双重循环法就不可取了。

位图法比较适合于这种情况,它的做法是按照集合中最大元素max创建一个长度为max+1的新数组,然后再次扫描原数组,遇到几就给新数组的第几位置上1,如遇到5就给新数组的第六个元素置1,这样下次再遇到5想置位时发现新数组的第六个元素已经是1了,这说明这次的数据肯定和以前的数据存在着重复。这种给新数组初始化时置零其后置一的做法类似于位图的处理方法故称位图法。它的运算次数最坏的情况为2N。如果已知数组的最大值即能事先给新数组定长的话效率还能提高一倍。



\section{最多重复数}
怎么在海量数据中找出重复次数最多的一个?

方案1:先做hash,然后求模映射为小文件,求出每个小文件中重复次数最多的一个,并记录重复次数。然后找出上一步求出的数据中重复次数最多的一个就是所求(具体参考前面的题)。

\section{TopK高频词}
上千万或上亿数据(有重复),统计其中出现次数最多的钱N个数据。

方案1:上千万或上亿的数据,现在的机器的内存应该能存下。所以考虑采用hash_map/搜索二叉树/红黑树等来进行统计次数。然后就是取出前N个出现次数最多的数据了,可以用第2题提到的堆机制完成。

\section{Top10高频词}
一个文本文件,大约有一万行,每行一个词,要求统计出其中最频繁出现的前10个词,请给出思想,给出时间复杂度分析。

方案1:这题是考虑时间效率。用trie树统计每个词出现的次数,时间复杂度是$O(n*le)$(le表示单词的平准长度)。然后是找出出现最频繁的前10个词,可以用堆来实现,前面的题中已经讲到了,时间复杂度是$O(n*lg10)$。所以总的时间复杂度,是O(n*le)与O(n*lg10)中较大的哪一个。

\section{Top100最大数}
100w个数中找出最大的100个数。

方案1:在前面的题中,我们已经提到了,用一个含100个元素的最小堆完成。复杂度为O(100w*lg100)。

方案2:采用快速排序的思想,每次分割之后只考虑比轴大的一部分,知道比轴大的一部分在比100多的时候,采用传统排序算法排序,取前100个。复杂度为O(100w*100)。

方案3:采用局部淘汰法。选取前100个元素,并排序,记为序列L。然后一次扫描剩余的元素x,与排好序的100个元素中最小的元素比,如果比这个最小的要大,那么把这个最小的元素删除,并把x利用插入排序的思想,插入到序列L中。依次循环,知道扫描了所有的元素。复杂度为O(100w*100)。

\section{海量数据处理方法大总结}
\subsection{Bloom Filter}
适用范围:可以用来实现数据字典,进行数据的判重,或者集合求交集

基本原理及要点:
对于原理来说很简单,位数组+k个独立hash函数。将hash函数对应的值的位数组置1,查找时如果发现所有hash函数对应位都是1说明存在,很明显这个过程并不保证查找的结果是100\%正确的。同时也不支持删除一个已经插入的关键字,因为该关键字对应的位会牵动到其他的关键字。所以一个简单的改进就是
\% counting Bloom filter,用一个counter数组代替位数组,就可以支持删除了。

还有一个比较重要的问题,如何根据输入元素个数n,确定位数组m的大小及hash函数个数。当hash函数个数$k=(ln2)*(m/n)$时错误率最小。在错误率不大于E的情况下,m至少要等于$n*lg(1/E)$才能表示任意n个元素的集合。
但m还应该更大些,因为还要保证bit数组里至少一半为0,则$m>=nlg(1/E)*lge$
 大概就是$1.44*nlg(1/E)$(lg表示以2为底的对数)。

举个例子我们假设错误率为0.01,则此时m应大概是n的13倍。这样k大概是8个。

注意这里m与n的单位不同,m是bit为单位,而n则是以元素个数为单位(准确的说是不同元素的个数)。通常单个元素的长度都是有很多bit的。所以使用bloom filter内存上通常都是节省的。

扩展:
Bloom filter将集合中的元素映射到位数组中,用k(k为哈希函数个数)个映射位是否全1表示元素在不在这个集合中。Counting  filter(CBF)将位数组中的每一位扩展为一个counter,从而支持了元素的删除操作。Spectral 
Bloom Filter(SBF)将其与集合元素的出现次数关联。SBF采用counter中的最小值来近似表示元素的出现频率。

问题实例:给你A,B两个文件,各存放50亿条URL,每条URL占用64字节,内存限制是4G,让你找出A,B文件共同的URL。如果是三个乃至n个文件呢?

根据这个问题我们来计算下内存的占用,4G=$2^32$大概是40亿*8大概是340亿,n=50亿,如果按出错率0.01算需要的大概是650亿个bit。现在可用的是340亿,相差并不多,这样可能会使出错率上升些。另外如果这些urlip是一一对应的,就可以转换成ip,则大大简单了。

\subsection{Hashing}
适用范围:快速查找,删除的基本数据结构,通常需要总数据量可以放入内存

基本原理及要点:
hash函数选择,针对字符串,整数,排列,具体相应的hash方法。
碰撞处理,一种是open hashing,也称为拉链法;另一种就是closed hashing,也称开地址法,opened addressing。

扩展:
d-left hashing中的d是多个的意思,我们先简化这个问题,看一看2-left hashing。2-left 
hashing指的是将一个哈希表分成长度相等的两半,分别叫做T1和T2,给T1和T2分别配备一个哈希函数,h1和h2。在存储一个新的key时,同时用两个哈希函数进行计算,得出两个地址h1[key]和h2[key]。这时需要检查T1中的h1[key]位置和T2中的h2[key]位置,哪一个位置已经存储的(有碰撞的)key比较多,然后将新key存储在负载少的位置。如果两边一样多,比如两个位置都为空或者都存储了一个key,就把新key存储在左边的T1子表中,2-left也由此而来。在查找一个key时,必须进行两次hash,同时查找两个位置。

问题实例:
1).海量日志数据,提取出某日访问百度次数最多的那个IP。
IP的数目还是有限的,最多$2^32$个,所以可以考虑使用hash将ip直接存入内存,然后进行统计。

\subsection{bit-map}
适用范围:可进行数据的快速查找,判重,删除,一般来说数据范围是int的10倍以下

基本原理及要点:使用bit数组来表示某些元素是否存在,比如8位电话号码

扩展: filter可以看做是对bit-map的扩展

问题实例:
1)已知某个文件内包含一些电话号码,每个号码为8位数字,统计不同号码的个数。
8位最多99 999 999,大概需要99m个bit,大概10几m字节的内存即可。
2)2.5亿个整数中找出不重复的整数的个数,内存空间不足以容纳这2.5亿个整数。

将bit-map扩展一下,用2bit表示一个数即可,0表示未出现,1表示出现一次,2表示出现2次及以上。或者我们不用2bit来进行表示,我们用两个bit-map即可模拟实现这个2bit-map。

\subsection{堆}
适用范围:海量数据前n大,并且n比较小,堆可以放入内存

基本原理及要点:最大堆求前n小,最小堆求前n大。方法,比如求前n小,我们比较当前元素与最大堆里的最大元素,如果它小于最大元素,则应该替换那个最大元素。这样最后得到的n个元素就是最小的n个。适合大数据量,求前n小,n的大小比较小的情况,这样可以扫描一遍即可得到所有的前n元素,效率很高。

扩展:双堆,一个最大堆与一个最小堆结合,可以用来维护中位数。

问题实例:
1)100w个数中找最大的前100个数。
用一个100个元素大小的最小堆即可。

\subsection{双层桶划分}
适用范围:第k大,中位数,不重复或重复的数字
基本原理及要点:因为元素范围很大,不能利用直接寻址表,所以通过多次划分,逐步确定范围,然后最后在一个可以接受的范围内进行。可以通过多次缩小,双层只是一个例子。

扩展:
问题实例:
1).2.5亿个整数中找出不重复的整数的个数,内存空间不足以容纳这2.5亿个整数。
有点像鸽巢原理,整数个数为$2^32$,也就是,我们可以将这$2^32$个数,划分为$2^8$个区域(比如用单个文件代表一个区域),然后将数据分离到不同的区域,然后不同的区域在利用bitmap就可以直接解决了。也就是说只要有足够的磁盘空间,就可以很方便的解决。

2).5亿个int找它们的中位数。
这个例子比上面那个更明显。首先我们将int划分为$2^16$个区域,然后读取数据统计落到各个区域里的数的个数,之后我们根据统计结果就可以判断中位数落到那个区域,同时知道这个区域中的第几大数刚好是中位数。然后第二次扫描我们只统计落在这个区域中的那些数就可以了。

实际上,如果不是int是int64,我们可以经过3次这样的划分即可降低到可以接受的程度。即可以先将int64分成$2^24$个区域,然后确定区域的第几大数,在将该区域分成$2^20$个子区域,然后确定是子区域的第几大数,
然后子区域里的数的个数只有$2^20$,就可以直接利用direct
 addr table进行统计了。

\subsection{数据库索引}
适用范围:大数据量的增删改查

基本原理及要点:利用数据的设计实现方法,对海量数据的增删改查进行处理。

\subsection{倒排索引(Inverted index)}
适用范围:搜索引擎,关键字查询

基本原理及要点:为何叫倒排索引?一种索引方法,被用来存储在全文搜索下某个单词在一个文档或者一组文档中的存储位置的映射。

以英文为例,下面是要被索引的文本:
\begin{Code}
	T0 = "it is what it is"
	T1 = "what is it"
	T2 = "it is a banana"
\end{Code}


我们就能得到下面的反向文件索引:
\begin{Code}
	"a":      {2}
	"banana": {2}
	"is":     {0, 1, 2}
	"it":     {0, 1, 2}
	"what":   {0, 1}
\end{Code}


检索的条件"what","is"和"it"将对应集合的交集。

正向索引开发出来用来存储每个文档的单词的列表。正向索引的查询往往满足每个文档有序频繁的全文查询和每个单词在校验文档中的验证这样的查询。在正向索引中,文档占据了中心的位置,每个文档指向了一个它所包含的索引项的序列。也就是说文档指向了它包含的那些单词,而反向索引则是单词指向了包含它的文档,很容易看到这个反向的关系。

扩展:
问题实例:文档检索系统,查询那些文件包含了某单词,比如常见的学术论文的关键字搜索。

\subsection{外排序}
适用范围:大数据的排序,去重

基本原理及要点:外排序的归并方法,置换选择败者树原理,最优归并树

扩展:

问题实例:
1).有一个1G大小的一个文件,里面每一行是一个词,词的大小不超过16个字节,内存限制大小是1M。返回频数最高的100个词。

这个数据具有很明显的特点,词的大小为16个字节,但是内存只有1m做hash有些不够,所以可以用来排序。内存可以当输入缓冲区使用。

\subsection{trie树}
适用范围:数据量大,重复多,但是数据种类小可以放入内存

基本原理及要点:实现方式,节点孩子的表示方式

扩展:压缩实现。

问题实例:
1).有10个文件,每个文件1G,每个文件的每一行都存放的是用户的query,每个文件的query都可能重复。要你按照query的频度排序。\\
2).1000万字符串,其中有些是相同的(重复),需要把重复的全部去掉,保留没有重复的字符串。请问怎么设计和实现?\\
3).寻找热门查询:查询串的重复度比较高,虽然总数是1千万,但如果除去重复后,不超过3百万个,每个不超过255字节。


\subsection{分布式处理 mapreduce}
适用范围:数据量大,但是数据种类小可以放入内存

基本原理及要点:将数据交给不同的机器去处理,数据划分,结果归约。

扩展:
问题实例:
1).The canonical example application of MapReduce is a process to count the appearances of
each different word in a set of documents:
2).海量数据分布在100台电脑中,想个办法高效统计出这批数据的TOP10。
3).一共有N个机器,每个机器上有N个数。每个机器最多存$O(N)$个数并对它们操作。如何找到$N^2$个数的中数(median)?

\section{经典问题分析}
上千万or亿数据(有重复),统计其中出现次数最多的前N个数据,分两种情况:可一次读入内存,不可一次读入。

可用思路:trie树+堆,数据库索引,划分子集分别统计,hash,分布式计算,近似统计,外排序

所谓的是否能一次读入内存,实际上应该指去除重复后的数据量。如果去重后数据可以放入内存,我们可以为数据建立字典,比如通过 
map,hashmap,trie,然后直接进行统计即可。当然在更新每条数据的出现次数的时候,我们可以利用一个堆来维护出现次数最多的前N个数据,当然这样导致维护次数增加,不如完全统计后在求前N大效率高。

如果数据无法放入内存。一方面我们可以考虑上面的字典方法能否被改进以适应这种情形,可以做的改变就是将字典存放到硬盘上,而不是内存,这可以参考数据库的存储方法。

当然还有更好的方法,就是可以采用分布式计算,基本上就是map-reduce过程,首先可以根据数据值或者把数据hash(md5)后的值,将数据按照范围划分到不同的机子,最好可以让数据划分后可以一次读入内存,这样不同的机子负责处理各种的数值范围,实际上就是map。得到结果后,各个机子只需拿出各自的出现次数最多的前N个数据,然后汇总,选出所有的数据中出现次数最多的前N个数据,这实际上就是reduce过程。

实际上可能想直接将数据均分到不同的机子上进行处理,这样是无法得到正确的解的。因为一个数据可能被均分到不同的机子上,而另一个则可能完全聚集到一个机子上,同时还可能存在具有相同数目的数据。比如我们要找出现次数最多的前100个,我们将1000万的数据分布到10台机器上,找到每台出现次数最多的前
 
100个,归并之后这样不能保证找到真正的第100个,因为比如出现次数最多的第100个可能有1万个,但是它被分到了10台机子,这样在每台上只有1千个,假设这些机子排名在1000个之前的那些都是单独分布在一台机子上的,比如有1001个,这样本来具有1万个的这个就会被淘汰,即使我们让每台机子选出出现次数最多的1000个再归并,仍然会出错,因为可能存在大量个数为1001个的发生聚集。因此不能将数据随便均分到不同机子上,而是要根据hash
 后的值将它们映射到不同的机子上处理,让不同的机器处理一个数值范围。

而外排序的方法会消耗大量的IO,效率不会很高。而上面的分布式方法,也可以用于单机版本,也就是将总的数据根据值的范围,划分成多个不同的子文件,然后逐个处理。处理完毕之后再对这些单词的及其出现频率进行一个归并。实际上就可以利用一个外排序的归并过程。

另外还可以考虑近似计算,也就是我们可以通过结合自然语言属性,只将那些真正实际中出现最多的那些词作为一个字典,使得这个规模可以放入内存。

\section{海量数据处理}
所谓海量数据处理,无非就是基于海量数据上的存储、处理、操作。何谓海量,就是数据量太大,所以导致要么是无法在较短时间内迅速解决,要么是数据太大,导致无法一次性装入内存。

那解决办法呢?针对时间,我们可以采用巧妙的算法搭配合适的数据结构,如Bloom 
filter/Hash/bit-map/堆/数据库或倒排索引/trie树,针对空间,无非就一个办法:大而化小,分而治之(hash映射),你不是说规模太大嘛,那简单啊,就把规模大化为规模小的,各个击破不就完了嘛。

至于所谓的单机及集群问题,通俗点来讲,单机就是处理装载数据的机器有限(只要考虑cpu,内存,硬盘的数据交互),而集群,机器有多辆,适合分布式处理,并行计算(更多考虑节点和节点间的数据交互)。

再者,通过本blog内的有关海量数据处理的文章:Big Data Processing,我们已经大致知道,处理海量数据问题,无非就是:

\begindot
\item 分而治之/hash映射 + hash统计 + 堆/快速/归并排序;
\item 双层桶划分
\item Bloom filter/Bitmap;
\item Trie树/数据库/倒排索引;
\item 外排序;
\item 分布式处理之Hadoop/Mapreduce。
\myenddot

第一部分、从set/map谈到hashtable/hash_map/hash_set
稍后本文第二部分中将多次提到hash_map/hash_set,下面稍稍介绍下这些容器,以作为基础准备。一般来说,STL容器分两种,
序列式容器(vector/list/deque/stack/queue/heap),
关联式容器。关联式容器又分为set(集合)和map(映射表)两大类,以及这两大类的衍生体multiset(多键集合)和multimap(多键映射表),这些容器均以RB-tree完成。此外,还有第3类关联式容器,如hashtable(散列表),以及以hashtable为底层机制完成的hash_set(散列集合)/hash_map(散列映射表)/hash_multiset(散列多键集合)/hash_multimap(散列多键映射表)。也就是说,set/map/multiset/multimap都内含一个RB-tree,而hash_set/hash_map/hash_multiset/hash_multimap都内含一个hashtable。
所谓关联式容器,类似关联式数据库,每笔数据或每个元素都有一个键值(key)和一个实值(value),即所谓的Key-Value(键-值对)。当元素被插入到关联式容器中时,容器内部结构(RB-tree/hashtable)便依照其键值大小,以某种特定规则将这个元素放置于适当位置。

包括在非关联式数据库中,比如,在MongoDB内,文档(document)是最基本的数据组织形式,每个文档也是以Key-Value(键-值对)的方式组织起来。一个文档可以有多个Key-Value组合,每个Value可以是不同的类型,比如String、Integer、List等等。
 
{ "name" : "July",  
	"sex" : "male",  
	"age" : 23 }  

set/map/multiset/multimap
set,同map一样,所有元素都会根据元素的键值自动被排序,因为set/map两者的所有各种操作,都只是转而调用RB-tree的操作行为,不过,值得注意的是,两者都不允许两个元素有相同的键值。
不同的是:set的元素不像map那样可以同时拥有实值(value)和键值(key),set元素的键值就是实值,实值就是键值,而map的所有元素都是pair,同时拥有实值(value)和键值(key),pair的第一个元素被视为键值,第二个元素被视为实值。
至于multiset/multimap,他们的特性及用法和set/map完全相同,唯一的差别就在于它们允许键值重复,即所有的插入操作基于RB-tree的insert_equal()而非insert_unique()。

hash_set/hash_map/hash_multiset/hash_multimap
hash_set/hash_map,两者的一切操作都是基于hashtable之上。不同的是,hash_set同set一样,同时拥有实值和键值,且实质就是键值,键值就是实值,而hash_map同map一样,每一个元素同时拥有一个实值(value)和一个键值(key),所以其使用方式,和上面的map基本相同。但由于hash_set/hash_map都是基于hashtable之上,所以不具备自动排序功能。为什么?因为hashtable没有自动排序功能。
至于hash_multiset/hash_multimap的特性与上面的multiset/multimap完全相同,唯一的差别就是它们hash_multiset/hash_multimap的底层实现机制是hashtable(而multiset/multimap,上面说了,底层实现机制是RB-tree),所以它们的元素都不会被自动排序,不过也都允许键值重复。
所以,综上,说白了,什么样的结构决定其什么样的性质,因为set/map/multiset/multimap都是基于RB-tree之上,所以有自动排序功能,而hash_set/hash_map/hash_multiset/hash_multimap都是基于hashtable之上,所以不含有自动排序功能,至于加个前缀multi_无非就是允许键值重复而已。
此外,
关于什么hash,请看blog内此篇文章;
关于红黑树,请参看blog内系列文章,
关于hash_map的具体应用:请看这里,关于hash_set:请看此文。

\section{Hadoop面试和学习小结}
随着大数据的盛行,Hadoop也流行起来。面过一些公司,包括开发Hadoop :如Cloudera, Hortonworks, MapR, Teradata, Greenplum, Amazon EMR, 使用Hadoop的除了Google,数不胜数了.
Hadoop 2.0转型基本无可阻挡,今年下半年要正式发布了,它的出现让大家知识体系都 要更新了。Hadoop1.0搞了8年才发布,2.0不到2年就出来了。2.0的核心是YARN,它的 诞生还是有趣的故事
YARN介绍
\begindot
\item \href{http://hortonworks.com/hadoop/yarn/}{Yarn from Hortonworks} 
\item \href{http://www.ibm.com/developerworks/cn/opensource/os-cn-hadoop-yarn/}{Yarn from IBM developerworks}
\myenddot
Hadoop 生态系统
\begindot
\item \href{http://www.neevtech.com/blog/2013/03/18/hadoop-ecosystem-at-a-glance/}{Hadoop Ecosystem at a Glance}
\myenddot
SQL on Hadoop
\begindot
\item \href{http://gigaom.com/2013/02/21/sql-is-whats-next-for-hadoop-heres-whos-doing-it/}{SQL is what’s next for Hadoop: Here’s who’s doing it}
\item \href{http://hadapt.com/blog/2013/10/28/all-sql-on-hadoop-solutions-are-missing-the-point-of-hadoop/}{All SQL-on-Hadoop Solutions are missing the point of Hadoop}
\myenddot
Hadoop Summit
\begindot
\item \href{http://hadoopsummit.org/san-jose/}{Hadoop Summit, San Jose}
\myenddot
书籍和Paper:
\begindot
\item “Hadoop: The Definitive Guide”: 里面内容非常好,既有高屋建瓴,又有微观把握,基本适用于1.X版本。比如mapreduce各个子阶段,Join在里面也有代码实现,第三版
\item \href{http://it-ebooks.info/book/635/}{Hadoop: The Definitive Guide, 3rd Edition}
\item \href{https://github.com/tomwhite/hadoop-book}{tomwhite/hadoop-book · GitHub}
\myenddot
Google的三辆马车,GFS, MapReduce, BigTable Google的新三辆马车:Caffeine、Pregel、Dremel
\begindot
\item \href{http://blog.mikiobraun.de/2013/02/big-data-beyond-map-reduce-googles-papers.html}{Big Data beyond MapReduce: Google’s Big Data papers}
\myenddot
SIGMOD, VLDB Top DB conference

入门:

知道MapReduce大致流程,Map, Shuffle, Reduce
知道Combiner, partition作用,设置Compression
搭建Hadoop集群,Master/Slave 都运行那些服务 NameNode, DataNode, JobTracker, TaskTracker
Pig, Hive 简单语法,UDF写法
\begindot
\item When to use Pig Latin versus Hive SQL?
\item Online Feedback Publishing System
\item http://www.slideshare.net/zshao
\item Introduction to Apache Hive Online Training
\item http://i.stanford.edu/~ragho/hive-icde2010.pdf
\myenddot
Hadoop 2.0新知识; HDFS2 HA,Snapshot, ResourceManager,ApplicationsManager, NodeManager

进阶:
\begindot
\item HDFS,Replica如何定位
\item HDFS Design
\item Hadoop 参数调优,性能优化,Cluster level: JVM, Map/Reduce Slots, Job level: Reducer \#, Memory, use Combiner? use Compression?
\item 7 Tips for Improving MapReduce Performance
\item Hadoop Summit 2010 Tuning Hadoop To Deliver Performance To Your Application
\item HBase 搭建,Region server, key如何选取?
\item 数据倾斜怎么办?
\myenddot
算法:
\begindot
\item 字典同位词
\item 翻译SQL语句 select count(x) from a group by b;
\item MapReduce Algorithms
\item Designing algorithms for Map Reduce
\myenddot
Blog
关注Cloudera, Hortonworks, MapR
\begindot
\item 董的博客
\item Hadoop学习资料
\myenddot
相关系统
\begindot
\item 数据流系统: Storm Kafka
\item 内存计算系统: Spark and Shark
\item 交互式实时系统:Cloudera Impala, Apache Drill (Dremel开源实现),Tez (Hortonworks)
\myenddot
公司列表:
\begindot
\item Powered by
\myenddot
其他
\begindot
\item Hadoop进化目标:开发部署傻瓜化,性能更强劲,最后为程序员标配。
\item 核心都是被寡头控制的,记得一边文章说一流的公司卖标准,二流的公司卖技术,三 流的公司卖产品,H和C有最多的committer,自然就影响着整个Hadoop社区。
\item 技术就是日新月异,还是多看看那些公司的博客,关注感兴趣的新产品,Hortonworks Stack
\item 在Hadoop系统中从头裸写MapReduce不现实了,ETL基本靠Hive,Pig, 还有Cascading,Scalding
\item MapReduce并不是最优的,仅适合批处理,很多问题:JVM的启动overhead很大,小 Job更明显,数据必须先存储,不适合迭代计算,延迟高。DB学术圈讨论很久tradeoff 了,MapReduce: 一个巨大的倒退
\myenddot
