\chapter{Debug \& Linux}
\section{Dump Core}
\section{GDB调试}
下面介绍我调试时经常遇到的三种问题,如果大家也有类似的问题交流一下解决方法:
\begindot
\item 情景1:在不中止程序服务的情况下,怎么调试正在运行时的程序
\item 情景2:需要同时看几个变量的值或者批量查看多个core文件的堆栈信息怎么办
\item 情景3:遇到需要查看、队列、链表、树、堆等数据结构里的变量怎么办
\myenddot

\subsection{情景1:在不中止程序服务的情况下,怎么调试正在运行时的程序}
我们在生产环境或者测试环境,会遇到一些异常,我们需要知道程序中的变量或者内存的值来确定程序运行状态.
systemstap可以实现这种功能,但systamstap写起来复杂一些,
还有时候在低内核版本的操作系统上用stap之后,程序或者操作系统都有可能死掉。

用pstack(修改了pstack代码,用gdb实现的,详见http://blog.yufeng.info/archives/873)查看和修改一个正在
执行程序的全局变量,尝试用gdb实现这种功能:

保存下面代码到文件runstack.sh

\begin{Code}
	#!/bin/sh
	if test $# -ne 2; then
	echo "Usage: `basename $0 .sh` <process-id> cmd" 1>&2
	echo "For exampl: `basename $0 .sh` 1000 bt" 1>&2
	exit 1
	fi
	if test ! -r /proc/$1; then
	echo "Process $1 not found." 1>&2
	exit 1
	fi
	result=""
	GDB=${GDB:-/usr/bin/gdb}
	# Run GDB, strip out unwanted noise.
	result=`$GDB –quiet -nx /proc/$1/exe $1 <<EOF 2>&1
	$2
	EOF`
	echo "$result" | egrep -A 1000 -e "^\(gdb\)" | egrep -B 1000 -e "^\(gdb\)"
	
\end{Code}

用于测试runstack.sh调试的c代码

\begin{Code}
	#include <stdio.h>
	#include <stdlib.h>
	#include <string.h>
	#include <unistd.h>
	
	typedef struct slist {
		struct slist *next;
		char data[4096];
	} slist;
	
	slist input_list = {NULL, {‘\0’}};
	int count = 0;
	
	static void stdin_read (int fd)
	{
		char buf[4096];
		int ret;
		
		memset(buf, 0, 4096);
		
		fprintf(stderr, "please input string:");
		
		while (ret = read(fd, buf, 4096)) {
			
			slist *node = calloc(1, sizeof(slist));
			memcpy(node->data, buf, ret);
			node->next = input_list.next;
			input_list.next = node;
			count ++;
			
			if (memcmp(buf, "quit", 4) == 0) {
				fprintf(stdout, "input quit:\n");
				return;
			}
			fprintf(stderr, "ret: %d, there is %d strings, current is %s\nplease input string:", ret, count, buf);
		}
	}
	
	int main()
	{
		fprintf(stderr, "main run!\n");
		
		stdin_read(STDIN_FILENO);
		
		slist *nlist;
		slist *list = input_list.next;
		while (list) {
			fprintf(stderr, "%s\n", list->data);
			nlist = list->next;
			free(list);
			list = nlist;
		}
		
		return 0;
	}
\end{Code}

编译c代码:gcc -g -o read_input read_input.c
执行./read_input 我们开始使用runstack.sh来调试
使用方法:sh ./runstack.sh pid “command”

\subsection{情景2:需要同时看几个变量的值或者批量查看多个core文件的信息}
\subsubsection{多个变量的情景}
我们同时看一下count和input_list里面的值和堆栈信息,我们可以写一个script.gdb
\begin{Code}
	p input_list
	p count
	bt
	f 1
	p buf
\end{Code}
\subsubsection{批处理查看core的情况}
有的时候会出现很多core文件,我们想知道哪些core文件是因为相同的原因,哪些是不相同的,看一个两个的时候还比较轻松

像上面有很多core文件,一个一个用gdb去执行bt去看core在哪里有点麻烦,我们想有把所有的core文件的堆栈和变量信息打印出来
我对runstack稍作修改就可以实现我们的需求,我们起名叫corestack.sh

\begin{Code}
	#!/bin/sh
	
	if test $# -ne 3; then
	echo "Usage: `basename $0 .sh` program core cmd" 1>&2
	echo "For example: `basename $0 .sh` ./main core.1111 bt" 1>&2
	exit 1
	fi
	
	if test ! -r $1; then
	echo "Process $1 not found." 1>&2
	exit 1
	fi
	
	result=""
	GDB=${GDB:-/usr/bin/gdb}
	# Run GDB, strip out unwanted noise.
	result=`$GDB –quiet -nx $1 $2 <<EOF 2>&1
	$3
	EOF`
	echo "$result" | egrep -A 1000 -e "^\(gdb\)" | egrep -B 1000 -e "^\(gdb\)"
\end{Code}

\subsubsection{情景3:遇到需要查看、队列、链表、树、堆等数据结构里的变量怎么办?}
下面介绍链表怎么处理,对其他数据结构感兴趣的同学可以自己尝试编写一些gdb脚本.gdb是支持编写的脚本的 http://sourceware.org/gdb/onlinedocs/gdb/Command-Files.html
我们写个plist.gdb,用while循环来遍历链表
\begin{Code}
	set $list=&input_list
	while($list)
	p *$list
	set $list=$list->next
	end
\end{Code}

\section{如何使用strace+pstack利器分析程序性能}
有时我们需要对程序进行优化、减少程序响应时间。除了一段段地对代码进行时间复杂度分析,我们还有更便捷的方法吗?

若能直接找到影响程序运行时间的函数调用,再有针对地对相关函数进行代码分析和优化,那相比漫无目的地看代码,效率就高多了。

将strace和pstack工具结合起来使用,就可以达到以上目的。strace跟踪程序使用的底层系统调用,可输出系统调用被执行的时间点以及各个调用耗时;pstack工具对指定PID的进程输出函数调用栈。

strace查看系统调用
pstack查看函数堆栈

pstack是一个脚本工具,其核心实现就是使用了gdb以及thread apply all bt命令,下面我们使用pstack查看server进程函数堆栈:

\begin{Code}
	# sh pstack.sh 16739
	#0 0x00002ba1f8152650 in __nanosleep_nocancel () from /lib64/libc.so.6
	#1 0x00002ba1f8152489 in sleep () from /lib64/libc.so.6
	#2 0x00000000004007bb in ha_ha ()
	#3 0x0000000000400a53 in main ()
\end{Code}

\section{Linux下多线程查看工具(pstree、ps、pstack)?}
pstree以树结构显示进程
\begin{Code}
	$ pstree -p user | grep ad
	sshd(22669)---bash(22670)---ad_preprocess(4551)-+-{ad_preprocess}(4552)
				|-{ad_preprocess}(4553)
				|-{ad_preprocess}(4554)
				|-{ad_preprocess}(4555)
				|-{ad_preprocess}(4556)
				`-{ad_preprocess}(4557)
\end{Code}
work为工作用户,-p为显示进程识别码,ad_preprocess共启动了6个子线程,加上主线程共7个线程

ps -Lf显示进程
\begin{Code}
	$ ps -Lf 4551
	UID        PID  PPID   LWP  C NLWP STIME TTY      STAT   TIME CMD
	work      4551 22670  4551  2    7 16:30 pts/2    Sl+    0:02 ./ad_preprocess
	work      4551 22670  4552  0    7 16:30 pts/2    Sl+    0:00 ./ad_preprocess
	work      4551 22670  4553  0    7 16:30 pts/2    Sl+    0:00 ./ad_preprocess
	work      4551 22670  4554  0    7 16:30 pts/2    Sl+    0:00 ./ad_preprocess
	work      4551 22670  4555  0    7 16:30 pts/2    Sl+    0:00 ./ad_preprocess
	work      4551 22670  4556  0    7 16:30 pts/2    Sl+    0:00 ./ad_preprocess
	work      4551 22670  4557  0    7 16:30 pts/2    Sl+    0:00 ./ad_preprocess
\end{Code}


进程共启动了7个线程

pstack显示每个进程的栈跟踪
\begin{Code}
	$ pstack 4551
	Thread 7 (Thread 1084229984 (LWP 4552)):
	#0  0x000000302afc63dc in epoll_wait () from /lib64/tls/libc.so.6
	#1  0x00000000006f0730 in ub::EPollEx::poll ()
	#2  0x00000000006f172a in ub::NetReactor::callback ()
	#3  0x00000000006fbbbb in ub::UBTask::CALLBACK ()
	#4  0x000000302b80610a in start_thread () from /lib64/tls/libpthread.so.0
	#5  0x000000302afc6003 in clone () from /lib64/tls/libc.so.6
	#6  0x0000000000000000 in ?? ()
	Thread 6 (Thread 1094719840 (LWP 4553)):
	#0  0x000000302afc63dc in epoll_wait () from /lib64/tls/libc.so.6
	#1  0x00000000006f0730 in ub::EPollEx::poll ()
	#2  0x00000000006f172a in ub::NetReactor::callback ()
	#3  0x00000000006fbbbb in ub::UBTask::CALLBACK ()
	#4  0x000000302b80610a in start_thread () from /lib64/tls/libpthread.so.0
	#5  0x000000302afc6003 in clone () from /lib64/tls/libc.so.6
	#6  0x0000000000000000 in ?? ()
	Thread 5 (Thread 1105209696 (LWP 4554)):
	#0  0x000000302b80baa5 in __nanosleep_nocancel ()
	#1  0x000000000079e758 in comcm::ms_sleep ()
	#2  0x00000000006c8581 in ub::UbClientManager::healthyCheck ()
	#3  0x00000000006c8471 in ub::UbClientManager::start_healthy_check ()
	#4  0x000000302b80610a in start_thread () from /lib64/tls/libpthread.so.0
	#5  0x000000302afc6003 in clone () from /lib64/tls/libc.so.6
	#6  0x0000000000000000 in ?? ()
	Thread 4 (Thread 1115699552 (LWP 4555)):
	#0  0x000000302b80baa5 in __nanosleep_nocancel ()
	#1  0x0000000000482b0e in armor::armor_check_thread ()
	#2  0x000000302b80610a in start_thread () from /lib64/tls/libpthread.so.0
	#3  0x000000302afc6003 in clone () from /lib64/tls/libc.so.6
	#4  0x0000000000000000 in ?? ()
	Thread 3 (Thread 1126189408 (LWP 4556)):
	#0  0x000000302af8f1a5 in __nanosleep_nocancel () from /lib64/tls/libc.so.6
	#1  0x000000302af8f010 in sleep () from /lib64/tls/libc.so.6
	#2  0x000000000044c972 in Business_config_manager::run ()
	#3  0x0000000000457b83 in Thread::run_thread ()
	#4  0x000000302b80610a in start_thread () from /lib64/tls/libpthread.so.0
	#5  0x000000302afc6003 in clone () from /lib64/tls/libc.so.6
	#6  0x0000000000000000 in ?? ()
	Thread 2 (Thread 1136679264 (LWP 4557)):
	#0  0x000000302af8f1a5 in __nanosleep_nocancel () from /lib64/tls/libc.so.6
	#1  0x000000302af8f010 in sleep () from /lib64/tls/libc.so.6
	#2  0x00000000004524bb in Process_thread::sleep_period ()
	#3  0x0000000000452641 in Process_thread::run ()
	#4  0x0000000000457b83 in Thread::run_thread ()
	#5  0x000000302b80610a in start_thread () from /lib64/tls/libpthread.so.0
	#6  0x000000302afc6003 in clone () from /lib64/tls/libc.so.6
	#7  0x0000000000000000 in ?? ()
	Thread 1 (Thread 182894129792 (LWP 4551)):
	#0  0x000000302af8f1a5 in __nanosleep_nocancel () from /lib64/tls/libc.so.6
	#1  0x000000302af8f010 in sleep () from /lib64/tls/libc.so.6
	#2  0x0000000000420d79 in Ad_preprocess::run ()
	#3  0x0000000000450ad0 in main ()
\end{Code}

\subsection{使用strace,lstrace,truss来跟踪程序的运行过程}
truss和strace用来跟踪一个进程的系统调用或信号产生的情况,而 ltrace用来跟踪进程调用库函数的情况。truss是早期为System V 
R4开发的调试程序,包括Aix、FreeBSD在内的大部分Unix系统都自带了这个工具;而strace最初是为SunOS系统编写的,ltrace 最早出现在GNU/Debian 
Linux中。这两个工具现在也已被移植到了大部分Unix系统中,大多数Linux发行版都自带了strace和ltrace,而FreeBSD也可通 过Ports安装它们。

\subsection{LSOF}
lsof 是 linux 下的一个非常实用的系统级的监控、诊断工具。