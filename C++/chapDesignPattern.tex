\chapter{Design Pattern}
UML的关联(Association), 聚合(Aggregation), 组合(Composition)区别 

【Association】两者之间存在某种关联即可,很弱的关系,如student and course, 每个学生可以选不同的课,每门课上有不同学生;表明某个对象可以向另一个对象通过某种方式发送消息。发送消息的方法可以通过指针成员变量, 
也可以是方法参数、局部变量等等。例如:

\begin{Code}
	class A{
	private:
		B* itsB;
	};
\end{Code}

【Aggregation】它表明了“部分”到“整体”的关系,显著的特点就是不能包含循环的联系 (就是说,部分中不能包含整体)
\begin{Code}
	class A {...} 
	
	class B { 
		A* a; 
		.....
	} 
\end{Code}


【Composition】跟聚合几乎相同,唯一的区别就是“部分”不能脱离“整体”单独存在,就是说, “部分”的生命期不能比“整体”还要长。
\begin{Code}
	class A{...} 
	class B{ 
		A a; 
		......
	}
\end{Code}


\section{Singleton}

There shall be only one object of a class during the execution of a programm

\begin{Code}
	class Singleton{
	public:
		Singleton (const MySingleton&) = delete;
		Singleton& operator=(const MySingleton&) = delete;
		static Singleton * getInstance(){
			std::unique_lock<std::mutex> lock(mx);
			if(theOnlyInstance == nullptr){
				theOnlyInstance = new Singleton();
			}
			return theOnlyInstance;
		}
		void release(); //will be the focus later
	private:
		std::mutex mx;
		static Singleton * theOnlyInstance;
		Singleton();
		~Singleton();
	};
	
	class Singleton{
	public:
		Singleton (const MySingleton&) = delete;
		Singleton& operator=(const Singleton&) = delete;
		static Singleton * getInstance(){
			static Singleton theOnlyInstance;
			return &theOnlyInstance;
		}
	private:
		Singleton();
		~Singleton();
	};
\end{Code}
