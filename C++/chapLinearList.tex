\chapter{线性表}
线性表(Linear List)包含:
\begindot
\item 顺序存储:数组(vector/deque/array/set)
\item 链式存储:单链表,双向链表,循环单链表,循环双向链表
\item 二者结合:静态链表
\myenddot

\section{数组} %%%%%%%%%%%%%%%%%%%%%%%%%%%%%%


\subsection{Remove Duplicates from Sorted Array}
\label{sec:remove-duplicates-from-sorted-array}

Given a sorted array, remove the duplicates in place such that each element
appear only once and return the new length.
Do not allocate extra space for another array, you must do this in place with
constant memory.
For example, given input array $\code{A = [1,1,2]}$,
Your function should return $length = 2$, and $A$ is now $\code{[1,2]}$.

【解题思路】二指针问题,一前一后扫描。

【Algorithm】
\begin{Code}
	// LeetCode, Remove Duplicates from Sorted Array
	// 时间复杂度O(n),空间复杂度O(1)
	class Solution {
		int removeDuplicates(int A[], int n) {
			if (n==0 || A==nullptr) return 0;
			int index = 0;
			for (int i = 1; i < n; i++)
				if (A[index] != A[i])
                    A[++index] = A[i];
			return index + 1;
		}
	};
\end{Code}

\subsubsection{相关题目}

\begindot
\item Remove Duplicates from Sorted Array II,见 \S
\ref{sec:remove-duplicates-from-sorted-array-ii}
\myenddot


\subsection{Remove Duplicates from Sorted Array II}
\label{sec:remove-duplicates-from-sorted-array-ii}



Follow up for "Remove Duplicates":
What if duplicates are allowed at most twice?
For example,
given sorted array \code{A = [1,1,1,2,2,3]},
your function should return length = 5, and A is now \code{[1,1,2,2,3]}

【解题思路】
加一个变量记录一下元素出现的次数即可。这题因为是已经排序的数组,所以一个变量即可解决。如果是没有排序的数组,则需要引入一个hashmap来记录出现次数。

【Sequent Scan】
\begin{Code}
	// LeetCode, Remove Duplicates from Sorted Array II
	// 时间复杂度O(n),空间复杂度O(1)
    #define OCCUR 2
	class Solution {
		int removeDuplicates(int A[], int n) {
			if (n <= OCCUR) return n;
			int index = OCCUR;
			for (int i = OCCUR; i < n; i++){
				if (A[i] != A[index - OCCUR])
					A[index++] = A[i];
			}
			return index;
		}
	};
\end{Code}


【Algorithm2】

上面的Algorithm略长,不过扩展性好一些,例如将\fn{occur < 2}改为\fn{occur
< 3},就变成了允许重复最多3次。
\begin{Code}
	// LeetCode, Remove Duplicates from Sorted Array II
	// 时间复杂度O(n),空间复杂度O(1)
	class Solution {
		int removeDuplicates(int A[], int n) {
			int index = 0;
			for (int i = 0; i < n; ++i) {
				if (i > 0 && i < n - 1 && A[i] == A[i - 1] && A[i] == A[i + 1])
					continue;
				A[index++] = A[i];
			}
			return index;
		}
	};
\end{Code}


\subsubsection{相关题目}

\begindot
\item Remove Duplicates from Sorted Array,见 \S
\ref{sec:remove-duplicates-from-sorted-array}
\myenddot

\subsection{Remove Duplicates in Array}
You're given an array of integers(eg [3,4,7,1,2,9,8]) Find the index of values that satisfy A+B = C+D, where A,B,C and D are integers values in the array.
Eg: Given [3,4,7,1,2,9,8] array
The following
3+7 = 1+ 9 satisfies A+B=C+D
so print (0,2,3,5)

\subsection{Bar Raiser Round}
Divide the array(+ve and -ve numbers) into two parts such that the average of both the parts is equal.

Input:
[1 7 15 29 11 9]

Output:
[15 9 1 7 11 29]

Explanation:
The average of first two elements is (15+9)/2 = 12, average of remaining elements is (1+7 +11 +29)/4 = 12

\subsection{Search in Rotated Sorted Array}
\label{sec:search-in-rotated-sorted-array}


Suppose a sorted array is rotated at some pivot unknown to you beforehand.
(i.e., \code{0 1 2 4 5 6 7} might become \code{4 5 6 7 0 1 2}).
You are given a target value to search. If found in the array return its index,
otherwise return -1.
You may assume no duplicate exists in the array.

【解题思路】
二分查找,难度主要在于左右边界的确定。


【二分查找】
\begin{Code}
	// LeetCode, Search in Rotated Sorted Array
	// 时间复杂度O(log n),空间复杂度O(1)
	class Solution {
		int search(int A[], int n, int target) {
			int first = 0, last = n;
			while (first != last) {
				const int mid = first  + (last - first) / 2;
				if (A[mid] == target) return mid;
				if (A[first] <= A[mid]) {
					if (A[first] <= target && target < A[mid])
						last = mid - 1;
					else	
                        first = mid + 1;
				} else {
					if (A[mid] < target && target <= A[last-1])
						first = mid + 1;
					else	
                        last = mid - 1;
				}
			}
			return -1;
		}
	};
\end{Code}


\subsubsection{相关题目}

\begindot
\item Search in Rotated Sorted Array II,见 \S
\ref{sec:search-in-rotated-sorted-array-ii}
\myenddot


\subsection{Search in Rotated Sorted Array II}
\label{sec:search-in-rotated-sorted-array-ii}



Follow up for "Search in Rotated Sorted Array": What if duplicates are
allowed?

Would this affect the run-time complexity? How and why?

Write a function to determine if a given target is in the array.


【解题思路】

允许重复元素,则上一题中如果\fn{A[m]>=A[l]},那么\fn{[l,m]}为递增序列的假设就不能成立了,比如\code{[1,3,1,1,1]}。

如果\fn{A[m]>=A[l]}不能确定递增,那就把它拆分成两个条件:
\begindot
\item 若\fn{A[m]>A[l]},则区间\fn{[l,m]}一定递增
\item 若\fn{A[m]==A[l]} 确定不了,那就\fn{l++},往下看一步即可。
\myenddot

\subsubsection{Algorithm}
\begin{Code}
	// LeetCode, Search in Rotated Sorted Array II
	// 时间复杂度O(n),空间复杂度O(1)
	class Solution {
		bool search(int A[], int n, int target) {
            if(n<1) return false;
			int first = 0, last = n - 1;
			while (first <= last) {
				const int mid = first  + (last - first) / 2;
				if (A[mid] == target)	return true;
				if (A[first] < A[mid]) {
                    if(target == A[begin]) return true;
					if (A[first] < target && target < A[mid])
						last = mid - 1;
					else	
                        first = mid + 1;
				} else if (A[first] > A[mid]) {
                    if(target == A[end]) return true;
					if (A[mid] < target && target < A[last-1])
						first = mid + 1;
					else	
                        last = mid - 1;
				} else //skip duplicate one
					first++;
			}
			return false;
		}
	};
\end{Code}


\subsubsection{相关题目}

\begindot
\item Search in Rotated Sorted Array,见 \S
\ref{sec:search-in-rotated-sorted-array}
\myenddot

\subsection{Same Gap from Three Arrays}
Given three arrays A,B,C containing unsorted numbers. Find three numbers a, b, c from each of array A, B, C such that |a-b|, |b-c| and |c-a| are minimum
Please provide as efficient code as you can.
Can you better than this?

\subsection{MinIntNotSum}
Given a array of positive integers, you have to find the smallest positive integer that can not be formed from the sum of numbers from array.

\subsection{Consecutive Sub-numbers}
Given a sorted array with some sequenced numbers and some non-sequenced numbers. Write an algorithm that takes this array as an input and returns a list of \{start, end\} of all consecutive numbers. Consecutive numbers have difference of 1 only.
E.g. of array:
[4, 5, 6, 7, 8, 9, 12, 15, 16, 17, 18, 20, 22, 23, 24, 27]

\subsection{Count 0s in Array}
Given a sorted array of 0’s and 1’s. Find out the no. of 0’s in it. Write recursive, iterative versions of the code.

\subsection{Median of Two Sorted Arrays}
\label{sec:median-of-two-sorted-arrays}



There are two sorted arrays A and B of size m and n respectively. Find the
median of the two sorted arrays. The overall run time complexity should be
$O(\log (m+n))$.


【解题思路】

本题通用表述:给定两个已经排序好的数组,找到两者所有元素中第$k$大的元素。

【解法一】$O(m+n)$的解法比较直观,直接merge两个数组,然后求第$k$大的元素。

【解法二】不过我们仅仅需要第$k$大的元素,是不需要“排序”这么复杂的操作的。可以用一个计数器,记录当前已经找到第$m$大的元素了。同时我们使用两个指针\fn{pA}和\fn{pB},分别指向A和B数组的第一个元素,使用类似于merge
sort的原理,如果数组A当前元素小,那么\fn{pA++},同时\fn{m++};如果数组B当前元素小,那么\fn{pB++},同时\fn{m++}。最终当$m$等于$k$的时候,就得到了我们的答案,$O(k)$时间,$O(1)$空间。但是,当$k$很接近$m+n$的时候,这个方法还是$O(m+n)$的。

【解法三】有没有更好的方案呢?我们可以考虑从$k$入手。如果我们每次都能够删除一个一定在第$k$大元素之前的元素,那么我们需要进行$k$次。但是如果每次我们都删除一半呢?由于A和B都是有序的,我们应该充分利用这里面的信息,类似于二分查找,也是充分利用了“有序”。
假设A和B的元素个数都大于$k/2$,我们将A的第$k/2$个元素(即\fn{A[k/2-1]})和B的第$k/2$个元素(即\fn{B[k/2-1]})进行比较,有以下三种情况(为了简化这里先假设$k$为偶数,所得到的结论对于$k$是奇数也是成立的):
\begindot
\item \fn{A[k/2-1] == B[k/2-1]}
\item \fn{A[k/2-1] > B[k/2-1]}
\item \fn{A[k/2-1] < B[k/2-1]}
\myenddot

如果\fn{A[k/2-1] < B[k/2-1]},意味着\fn{A[0]}到\fn{A[k/2-1}的肯定在$A \cup B$的top
k元素的范围内,换句话说,\fn{A[k/2-1}不可能大于$A \cup B$的第$k$大元素。因此,我们可以放心的删除A数组的这$k/2$个元素。同理,当\fn{A[k/2-1] >
B[k/2-1]}时,可以删除B数组的$k/2$个元素。

当\fn{A[k/2-1] ==
B[k/2-1]}时,说明找到了第$k$大的元素,直接返回\fn{A[k/2-1]}或\fn{B[k/2-1]}即可。

因此,我们可以写一个递归函数。那么函数什么时候应该终止呢?
\begindot
\item 当A或B是空时,直接返回\fn{B[k-1]}或\fn{A[k-1]};
\item 当\fn{k=1}是,返回\fn{min(A[0], B[0])};
\item 当\fn{A[k/2-1] == B[k/2-1]}时,返回\fn{A[k/2-1]}或\fn{B[k/2-1]}
\myenddot

【解法三】
\begin{Code}
	// LeetCode, Median of Two Sorted Arrays
	// 时间复杂度O(log(m+n)),空间复杂度O(log(m+n))
	class Solution {
		double findMedianSortedArrays(int A[], int m, int B[], int n) {
			int total = m + n;
			if (total & 0x1)
				return find_kth(A, m, B, n, total / 2 + 1);
			return (find_kth(A, m, B, n, total / 2) + find_kth(A, m, B, n, total / 2 + 1)) / 2.0;
		}
		
		static int find_kth(int A[], int m, int B[], int n, int k) {
			//always assume that m is equal or smaller than n
			if (m > n) return find_kth(B, n, A, m, k);
			if (m == 0) return B[k - 1];
			if (k == 1) return min(A[0], B[0]);
			
			//divide k into two parts
			int ia = min(k / 2, m), ib = k - ia;
			if (A[ia - 1] < B[ib - 1])
				return find_kth(A + ia, m - ia, B, n, k - ia);
			else if (A[ia - 1] > B[ib - 1])
				return find_kth(A, m, B + ib, n - ib, k - ib);
			else
				return A[ia - 1];
		}
	};
\end{Code}


\subsection{Longest Consecutive Sequence} %%%%%%%%%%%%%%%%%%%%%%%%%%%%%%
\label{sec:longest-consecutive-sequence}



Given an unsorted array of integers, find the length of the longest consecutive
elements sequence.
For example,
Given \code{[100, 4, 200, 1, 3, 2]},
The longest consecutive elements sequence is \code{[1, 2, 3, 4]}. Return its
length: 4.
Your algorithm should run in $O(n)$ complexity.


【解题思路】
如果允许$O(n \log n)$的复杂度,那么可以先排序,可是本题要求$O(n)$。

由于序列里的元素是无序的,又要求$O(n)$,首先要想到用哈希表。

用一个哈希表 \fn{unordered_map<int, bool>
used}记录每个元素是否使用,对每个元素,以该元素为中心,往左右扩张,直到不连续为止,记录下最长的长度。


【Hash Cache】
\begin{Code}
	// Leet Code, Longest Consecutive Sequence
	// 时间复杂度O(n),空间复杂度O(n)
	class Solution {
		int longestConsecutive(const vector<int> &num) {
			unordered_map<int, bool> used;
			for (auto i : num) used[i] = false;
			int longest = 0;
			for (auto i : num) {
				if (used[i]) continue;
				int length = 1;
				used[i] = true;
				for (int j = i + 1; used.find(j) != used.end(); ++j) {
					used[j] = true;
					++length;
				}
				for (int j = i - 1; used.find(j) != used.end(); --j) {
					used[j] = true;
					++length;
				}
				longest = max(longest, length);
			}
			return longest;
		}
	};
\end{Code}

【聚类】

直觉是个聚类的操作,应该有union,find的操作。连续序列可以用两端和长度来表示。
本来用两端就可以表示,但考虑到查询的需求,将两端分别暴露出来,用\fn{unordered_map<int, int>
map}来存储。


\begin{Code}
	// Leet Code, Longest Consecutive Sequence
	// 时间复杂度O(n),空间复杂度O(n)
	class Solution {
		int longestConsecutive(vector<int> &num) {
			unordered_map<int, int> map;
			int size = num.size();
			int l = 1;
			for (int i = 0; i < size; i++) {
				if (map.find(num[i]) != map.end()) continue;
				map[num[i]] = 1;
				if (map.find(num[i] - 1) != map.end())
					l = max(l, mergeCluster(map, num[i] - 1, num[i]));
				if (map.find(num[i] + 1) != map.end())
					l = max(l, mergeCluster(map, num[i], num[i] + 1));
			}
			return size == 0 ? 0 : l;
		}
		
		int mergeCluster(unordered_map<int, int> &map, int left, int right) {
			int upper = right + map[right] - 1;
			int lower = left - map[left] + 1;
			int length = upper - lower + 1;
			map[upper] = length;
			map[lower] = length;
			return length;
		}
	};
\end{Code}

【HashMap1】
\begin{Code}
class Solution {
	int longestConsecutive(vector<int> &num) {
        unordered_map<int,int> s;
        int longest=1;
        for(int &i : num){
            if(s.find(i) == s.end()){
             s[i] = 1;
             int left = s.find(i-1) != s.end()? (i-s[i-1]) : i;
             int right = s.find(i+1) != s.end()? (i+s[i+1]) : i;
             s[left] = s[right] = right-left+1;
             longest = s[right]>longest? s[right] : longest;
            }
        }
        return longest;
    }
\end{Code}

【HashMap2】The idea is to build a map on the way. The key is the number, the value means the length of the sequence which has the key as one of its borders.
we iterate each value in the num array, if it resides in the map, just ignore. Else, we find the two adjcent sequences from left and right, and combine them into one seq. after that, we update the value of the borders of the new seq to the length of the seq.
\begin{Code}
    class Solution {
        int longestConsecutive(vector<int> &num) {
            int res = 0;
            unordered_map<int, int> h;
            for (auto& x: num) {
                if (!h[x]) {
                    h[x] = 1 + h[x+1] + h[x-1];
                    if (h[x+1]) h[h[x+1]+x] = h[x];
                    if (h[x-1]) h[x-h[x-1]] = h[x];
                }
                res = max(h[x], res);
            }
            return res;
        }
    };
\end{Code}

\subsection{$k$Sum} %%%%%%%%%%%%%%%%%%%%%%%%%%%%%%
\label{sec:ksum}

Given an array of integers, find $k$ numbers such that they add up to a
specific target number.

Find k elements in set A, where $\sum_{i=0}^k{A_i}=target$


【解题思路】
For even $k$, Time: $O(n^{\frac{k}{2}}log(n))$--
Compute a sorted list $S$ of all sum of $\frac{k}{2}$ elements in $A$. Check whether $S$ contains two elements that sum to target.

For odd $k$, Time: $O(n^{\frac{k+1}{2}})$ --
Compute a sorted list $S$ of all sum of $\frac{k-1}{2}$ elements in $A$. For each input element $a$ in $A$, check whether $S$ contains $s$ and $s′$, where $a+s+s′=target$

\begin{center}
	\begin{tabular}{|c|c|c|c|c|}
		\hline
		\multirow{2}{*}{\textbf{$k≥2$}} & \multirow{2}{*}{\textbf{Brute Force}} & \multirow{2}{*}{\textbf{Sort Find}} & \multicolumn{2}{|c|}{\textbf{Hash}} \\
		\cline{4-5} & & & \textbf{Time} & \textbf{Space}\\
		\hline
		2Sum & $n^2$ & $nlog(n)$ & $n$ & $n$ \\
		\hline
		3Sum & $n^3$ & $n^2$ & $n^2$ & $n$ \\
		\hline
		4Sum & $n^4$ & $n^3$ & $n^2log(n)$ & $n^2$ \\
		\hline
	\end{tabular}
\end{center}



方法1: Brute Force: Find all $k$ pairs of numbers and calculate their sum, if it equals the target return their index in increasing order. Running time = $O(n^k)$.

方法2: Sort + Double Pointer: First, sort the array in $O(nlog(n))$ time. Then use two pointers $p$ and $q$, $p$ scans from left to right, and $q$ scans from right to left.unning
time =
$O(n\log n)$

方法3:Hash Map: We can do even better, by using hash map. First we create a hash map, where
$[key, value] = [target - A[i], i]$.
This requires $O(n)$. Then, we iterate the array. If $A[j]$ contains in map, which means $A[j] = target - A[i]$, we return $i$ and $j$. This solution has a running time $O(n)$,
and its space
complicity is $O(n)$. We used $O(n)$ space to reduce the running time.
Note, one number may be used twice, therefore, we need to check $j != map.get(A[j])$

Note: Handle duplicate elements in the result.

\begin{Code}
	//LeetCode, KSum
	// 方法3:hash。用一个哈希表,存储每个数对应的下标
	// 时间复杂度O(n),空间复杂度O(n)
	class Solution {
		vector< vector<int> > KSum(vector<int> &sortednum, int K, int target, int p) {
			vector< vector<int> > vecResults;
			if (K == 2) { // base case
				vector<int> tuple(2, 0);
				int i = p, j = sortednum.size() - 1;
				while (i < j) {
					if (i > p && sortednum[i] == sortednum[i - 1]) {
						++i;
						continue;
					}
					int sum = sortednum[i] + sortednum[j];
					if (sum == target) {
						tuple[0] = sortednum[i++];
						tuple[1] = sortednum[j--];
						vecResults.push_back(tuple);
					}
					else if (sum > target) {
						--j;
					}
					else {
						++i;
					}
				}
				return vecResults;
			}
			// K > 2
			for (int i = p; i < sortednum.size(); ++i) {
				if (i > p && sortednum[i] == sortednum[i - 1]) continue;
				vector< vector<int> > K1Sum = KSum(sortednum, K - 1, target - sortednum[i], i + 1);
				for (auto it = K1Sum.begin(); it != K1Sum.end(); ++it) {
					vector<int> tuple;
					tuple.push_back(sortednum[i]);
					tuple.insert(tuple.end(), it->begin(), it->end());
					vecResults.push_back(tuple);
				}
			}
			return vecResults;
		}
	};
\end{Code}


\subsubsection{相关题目}
\begindot
\item KSum, 见 \S \ref{sec:2sum}
\item 3Sum, 见 \S \ref{sec:3sum}
\item 3Sum Closest, 见 \S \ref{sec:3sum-closest}
\item 4Sum, 见 \S \ref{sec:4sum}
\myenddot



\subsection{2Sum} %%%%%%%%%%%%%%%%%%%%%%%%%%%%%%
\label{sec:2sum}



Given an array of integers, find two numbers such that they add up to a
specific target number.

The function twoSum should return indices of the two numbers such that they add
up to the target, where index1 must be less than index2. Please note that your
returned answers (both index1 and index2) are not zero-based.

You may assume that each input would have exactly one solution.

Input:  \code{numbers=\{2, 7, 11, 15\}, target=9}

Output: \code{index1=1, index2=2}


【解题思路】

方法1:暴力,复杂度$O(n^2)$,会超时

方法2:hash。用一个哈希表,存储每个数对应的下标,复杂度$O(n)$.

方法3:先排序,然后左右夹逼,注意跳过重复的数,排序$O(n\log n)$,左右夹逼$O(n)$,最终$O(n\log
n)$。但是注意,这题需要返回的是下标,而不是数字本身,因此这个方法行不通。


【方法二】
\begin{Code}
	//LeetCode, Two Sum
	// 方法2:hash。用一个哈希表,存储每个数对应的下标
	// 时间复杂度O(n),空间复杂度O(n)
	class Solution {
		vector<int> twoSum(vector<int> &num, int target) {
			unordered_map<int, int> mapping;
			vector<int> result;
			for (int i = 0; i < num.size(); i++)
				mapping[num[i]] = i;
			for (int i = 0; i < num.size(); i++) {
				const int gap = target - num[i];
				if (mapping.find(gap) != mapping.end() && mapping[gap] > i) {
					result.push_back(i + 1);
					result.push_back(mapping[gap] + 1);
					break;
				}
			}
			return result;
		}
	};
\end{Code}


\subsubsection{相关题目}
\begindot
\item KSum, 见 \S \ref{sec:ksum}
\item 3Sum, 见 \S \ref{sec:3sum}
\item 3Sum Closest, 见 \S \ref{sec:3sum-closest}
\item 4Sum, 见 \S \ref{sec:4sum}
\myenddot

\subsection{2Sum on BSTree}
Given a binary search tree of n nodes, find two nodes whose sum is equal to a given number $k$ in O(n) time and constant space.

【Algorithm】
\begin{Code}
	//Two Sum in BST
	// 时间复杂度O(n),空间复杂度O(n)
	class Solution {
		vector<Node*> twoNodeSum(Node *root, int k) {
			if(root == nullptr) return false;
            vector<Node*> res;
        	if( 2*root->value > sum)
        		twoNodeSum(root->left, root, sum, res);
        	else if(2*root->value < sum)
        		twoNodeSum(root, root->right, sum, res);
        	else // this case occurs when 2*root == value
        		twoNodeSum(root->left, root->right, sum, res);
            return res;
		}

        bool twoNodeSum( Node *left, Node *right, int target, vector<Node*> &res){
	       assert(left && right);
	       if(left.value + right.value > sum){
		      if(twoNodeSum(left->left, right, sum, res))
			     return true;
		      if(twoNodeSum(left, right->left, sum, res))
			     return true;	
	       }
	
	       if(left->value + right->value < sum){
		      if(twoNodeSum(left->right, right, sum, res))
			     return true;
		      if(twoNodeSum(left, right->right, sum, res))
			     return true;
	       }

	       if(left->value + right->value == sum){
                res.push_back(left);
                res.push_back(right);
                return true;
            }
            return false;
        }
	};
\end{Code}


\subsection{3Sum} %%%%%%%%%%%%%%%%%%%%%%%%%%%%%%
\label{sec:3sum}

Given an array $S$ of $n$ integers, are there elements $a, b, c$ in $S$ such
that $a + b + c = 0$? Find all unique triplets in the array which gives the sum
of zero.

Note:
\begindot
\item Elements in a triplet $(a,b,c)$ must be in non-descending order. (ie, $a
\leq b \leq c$)
\item The solution set must not contain duplicate triplets.
\myenddot

For example, given array \code{S = \{-1 0 1 2 -1 -4\}}.

A solution set is:
\begin{Code}
	(-1, 0, 1)
	(-1, -1, 2)
\end{Code}


【解题思路】

先排序,然后左右夹逼,复杂度 $O(n^2)$。

这个方法可以推广到$k$-sum,先排序,然后做$k-2$次循环,在最内层循环左右夹逼,时间复杂度是
$O(\max\{n \log n, n^{k-1}\})$。


\begin{Code}
	// LeetCode, 3Sum
	// 先排序,然后左右夹逼,注意跳过重复的数,时间复杂度O(n^2),空间复杂度O(1)
	class Solution {
		vector<vector<int>> threeSum(vector<int>& num) {
			vector<vector<int>> result;
			if (num.size() < 3) return result;
			sort(num.begin(), num.end());
			const int target = 0;
			auto last = num.end();
			for (auto a = num.begin(); a < prev(last, 2); ++a) {
				auto b = next(a);
				auto c = prev(last);
				while (b < c) {
					if (*a + *b + *c < target) {
						++b;
					} else if (*a + *b + *c > target) {
						--c;
					} else {
						result.push_back({ *a, *b, *c });
						++b;
						--c;
					}
				}
			}
			sort(result.begin(), result.end());
			result.erase(unique(result.begin(), result.end()), result.end());
			return result;
		}
	};
\end{Code}

【DuplicateCase】
\begin{Code}
	// LeetCode, 3Sum
	// 先排序,然后左右夹逼,注意跳过重复的数,时间复杂度O(n^2),空间复杂度O(1)
	class Solution {
		public:
		vector<vector<int>> threeSum(vector<int>& num) {
			vector<vector<int>> result;
			if (num.size() < 3) return result;
			sort(num.begin(), num.end());
			const int target = 0;
			
			auto last = num.end();
			for (auto i = num.begin(); i < last-2; ++i) {
				auto j = i+1;
				if (i > num.begin() && *i == *(i-1)) continue;
				auto k = last-1;
				while (j < k) {
					if (*i + *j + *k < target) {
						++j;
						while(*j == *(j - 1) && j < k) ++j;
					} else if (*i + *j + *k > target) {
					--k;
					while(*k == *(k + 1) && j < k) --k;
				} else {
				result.push_back({ *i, *j, *k });
				++j;
				--k;
				while(*j == *(j - 1) && *k == *(k + 1) && j < k) ++j;
			}
		}
	}
	return result;
}
};
\end{Code}

\subsubsection{相关题目}
\begindot
\item KSum, 见 \S \ref{sec:ksum}
\item 2Sum, 见 \S \ref{sec:2sum}
\item 3Sum Closest, 见 \S \ref{sec:3sum-closest}
\item 4Sum, 见 \S \ref{sec:4sum}
\myenddot

\subsection{3Sum Closest} %%%%%%%%%%%%%%%%%%%%%%%%%%%%%%
\label{sec:3sum-closest}



Given an array $S$ of $n$ integers, find three integers in $S$ such that the
sum is closest to a given number, target. Return the sum of the three integers.
You may assume that each input would have exactly one solution.

For example, given array \code{S = \{-1 2 1 -4\}}, and \code{target = 1}.

The sum that is closest to the target is 2. (\code{-1 + 2 + 1 = 2}).


【解题思路】
先排序,然后左右夹逼,复杂度 $O(n^2)$。


【左右夹逼】
\begin{Code}
	// LeetCode, 3Sum Closest
	// 先排序,然后左右夹逼,时间复杂度O(n^2),空间复杂度O(1)
	class Solution {
		int threeSumClosest(vector<int>& num, int target) {
			int result = 0;
			int min_gap = INT_MAX;
			sort(num.begin(), num.end());
			for (auto a = num.begin(); a != prev(num.end(), 2); ++a) {
				auto b = next(a);
				auto c = prev(num.end());
				while (b < c) {
					const int sum = *a + *b + *c;
					const int gap = abs(sum - target);
					if (gap < min_gap) {
						result = sum;
						min_gap = gap;
					}
					if (sum < target) ++b;
					else              --c;
				}
			}
			return result;
		}
	};
\end{Code}


\subsubsection{相关题目}
\begindot
\item KSum, 见 \S \ref{sec:ksum}
\item 2Sum, 见 \S \ref{sec:2sum}
\item 3Sum, 见 \S \ref{sec:3sum}
\item 4Sum, 见 \S \ref{sec:4sum}
\myenddot


\subsection{4Sum} %%%%%%%%%%%%%%%%%%%%%%%%%%%%%%
\label{sec:4sum}



Given an array $S$ of $n$ integers, are there elements $a, b, c$, and $d$ in
$S$ such that $a + b + c + d = target$? Find all unique quadruplets in the
array which gives the sum of target.

Note:
\begindot
\item Elements in a quadruplet $(a,b,c,d)$ must be in non-descending order.
(ie, $a \leq b \leq c \leq d$)
\item The solution set must not contain duplicate quadruplets.
\myenddot

For example, given array \code{S = \{1 0 -1 0 -2 2\}}, and \code{target = 0}.

A solution set is:
\begin{Code}
	(-1,  0, 0, 1)
	(-2, -1, 1, 2)
	(-2,  0, 0, 2)
\end{Code}


【解题思路】
先排序,然后左右夹逼,复杂度 $O(n^3)$,会超时。

可以用一个hashmap先缓存两个数的和,最终复杂度$O(n^3)$。这个策略也适用于 3Sum 。


【左右夹逼】
\begin{Code}
	// LeetCode, 4Sum
	// 先排序,然后左右夹逼,时间复杂度O(n^3),空间复杂度O(1)
	class Solution {
		vector<vector<int>> fourSum(vector<int>& num, int target) {
			vector<vector<int>> result;
			if (num.size() < 4) return result;
			sort(num.begin(), num.end());
			auto last = num.end();
			for (auto a = num.begin(); a < prev(last, 3); ++a) {
				for (auto b = next(a); b < prev(last, 2); ++b) {
					auto c = next(b);
					auto d = prev(last);
					while (c < d) {
						if (*a + *b + *c + *d < target) {
							++c;
						} else if (*a + *b + *c + *d > target) {
							--d;
						} else {
							result.push_back({ *a, *b, *c, *d });
							++c;
							--d;
						}
					}
				}
			}
			sort(result.begin(), result.end());
			result.erase(unique(result.begin(), result.end()), result.end());
			return result;
		}
	};
\end{Code}


【map做缓存】
\begin{Code}
	// LeetCode, 4Sum
	// 用一个hashmap先缓存两个数的和
	// 时间复杂度,平均O(n^2),最坏O(n^4),空间复杂度O(n^2)
	class Solution {
		vector<vector<int> > fourSum(vector<int> &num, int target) {
			vector<vector<int>> result;
			if (num.size() < 4) return result;
			sort(num.begin(), num.end());
			unordered_map<int, vector<pair<int, int> > > cache;
			for (size_t a = 0; a < num.size(); ++a) {
				for (size_t b = a + 1; b < num.size(); ++b)
					cache[num[a] + num[b]].push_back(pair<int, int>(a, b));
			}
			
			for (int c = 0; c < num.size(); ++c) {
				for (size_t d = c + 1; d < num.size(); ++d) {
					const int key = target - num[c] - num[d];
					if (cache.find(key) == cache.end()) continue;
					const auto& vec = cache[key];
					for (size_t k = 0; k < vec.size(); ++k) {
						if (c <= vec[k].second)
						continue; // 有重叠
						result.push_back( {num[vec[k].first],num[vec[k].second], num[c], num[d]});
					}
				}
			}
			sort(result.begin(), result.end());
			result.erase(unique(result.begin(), result.end()), result.end());
			return result;
		}
	};
\end{Code}


【multimap】
\begin{Code}
	// LeetCode, 4Sum
	// 用一个 hashmap 先缓存两个数的和
	// 时间复杂度O(n^2),空间复杂度O(n^2)
	// @author 龚陆安(http://weibo.com/luangong)
	class Solution {
		vector<vector<int>> fourSum(vector<int>& num, int target) {
			vector<vector<int>> result;
			if (num.size() < 4) return result;
			sort(num.begin(), num.end());
			unordered_multimap<int, pair<int, int>> cache;
			for (int i = 0; i + 1 < num.size(); ++i)
				for (int j = i + 1; j < num.size(); ++j)
					cache.insert(make_pair(num[i] + num[j], make_pair(i, j)));
			
			for (auto i = cache.begin(); i != cache.end(); ++i) {
				int x = target - i->first;
				auto range = cache.equal_range(x);
				for (auto j = range.first; j != range.second; ++j) {
					auto a = i->second.first;
					auto b = i->second.second;
					auto c = j->second.first;
					auto d = j->second.second;
					if (a != c && a != d && b != c && b != d) {
						vector<int> vec = { num[a], num[b], num[c], num[d] };
						sort(vec.begin(), vec.end());
						result.push_back(vec);
					}
				}
			}
			sort(result.begin(), result.end());
			result.erase(unique(result.begin(), result.end()), result.end());
			return result;
		}
	};
\end{Code}


【方法4】
\begin{Code}
	// LeetCode, 4Sum
	// 先排序,然后左右夹逼,时间复杂度O(n^3logn),空间复杂度O(1),会超时
	// 跟方法1相比,表面上优化了,实际上更慢了,切记!
	class Solution {
		vector<vector<int>> fourSum(vector<int>& num, int target) {
			vector<vector<int>> result;
			if (num.size() < 4) return result;
			sort(num.begin(), num.end());
			auto last = num.end();
			for (auto a = num.begin(); a < prev(last, 3);
			a = upper_bound(a, prev(last, 3), *a)) {
				for (auto b = next(a); b < prev(last, 2);
				b = upper_bound(b, prev(last, 2), *b)) {
					auto c = next(b);
					auto d = prev(last);
					while (c < d) {
						if (*a + *b + *c + *d < target) {
							c = upper_bound(c, d, *c);
						} else if (*a + *b + *c + *d > target) {
							d = prev(lower_bound(c, d, *d));
						} else {
							result.push_back({ *a, *b, *c, *d });
							c = upper_bound(c, d, *c);
							d = prev(lower_bound(c, d, *d));
						}
					}
				}
			}
			return result;
		}
	};
\end{Code}


\subsubsection{相关题目}
\begindot
\item Ksum, 见 \S \ref{sec:ksum}
\item 2sum, 见 \S \ref{sec:2sum}
\item 3Sum, 见 \S \ref{sec:3sum}
\item 3Sum Closest, 见 \S \ref{sec:3sum-closest}
\myenddot


\subsection{Remove Element} %%%%%%%%%%%%%%%%%%%%%%%%%%%%%%
\label{sec:remove-element }



Given an array and a value, remove all instances of that value in place and
return the new length.

The order of elements can be changed. It doesn't matter what you leave beyond
the new length.

【Algorithm1】
\begin{Code}
	// LeetCode, Remove Element
	// 时间复杂度O(n),空间复杂度O(1)
	class Solution {
		int removeElement(int A[], int n, int elem) {
			int index = 0;
			for (int i = 0; i < n; ++i) {
				if (A[i] != elem)
					A[index++] = A[i];
			}
			return index;
		}
	};
\end{Code}


【Algorithm2】
\begin{Code}
	// LeetCode, Remove Element
	// 使用remove(),时间复杂度O(n),空间复杂度O(1)
	class Solution {
		int removeElement(int A[], int n, int elem) {
			return distance(A, remove(A, A+n, elem));
		}
	};
\end{Code}


\subsection{Next Permutation} %%%%%%%%%%%%%%%%%%%%%%%%%%%%%%
\label{sec:next-permutation}



Implement next permutation, which rearranges numbers into the lexicographically
next greater permutation of numbers.

If such arrangement is not possible, it must rearrange it as the lowest
possible order (ie, sorted in ascending order).

The replacement must be in-place, do not allocate extra memory.

Here are some examples. Inputs are in the left-hand column and its
corresponding outputs are in the right-hand column.
\begin{Code}
	1,2,3 → 1,3,2
	3,2,1 → 1,2,3
	1,1,5 → 1,5,1
\end{Code}


【解题思路】
算法过程如图~\ref{fig:permutation}所示(来自\myurl{http://fisherlei.blogspot.com/2012/12/leetcode-next-permutation.html})。

\begin{center}
	\includegraphics[width=360pt]{next-permutation.png}\\
	\figcaption{下一个排列算法流程}\label{fig:permutation}
\end{center}


【Algorithm】
\begin{Code}
	// LeetCode, Next Permutation
	// 时间复杂度O(n),空间复杂度O(1)
	class Solution {
		void nextPermutation(vector<int> &num) {
			next_permutation(num.begin(), num.end());
		}
		
		template<typename BidiIt>
		bool next_permutation(BidiIt first, BidiIt last) {
			// Get a reversed range to simplify reversed traversal.
			const auto rfirst = reverse_iterator<BidiIt>(last);
			const auto rlast = reverse_iterator<BidiIt>(first);
			
			// Begin from the second last element to the first element.
			auto pivot = next(rfirst);
			
			// Find `pivot`, which is the first element that is no less than its
			// successor.  `Prev` is used since `pivort` is a `reversed_iterator`.
			while (pivot != rlast && *pivot >= *prev(pivot))
			++pivot;
			
			// No such elemenet found, current sequence is already the largest
			// permutation, then rearrange to the first permutation and return false.
			if (pivot == rlast) {
				reverse(rfirst, rlast);
				return false;
			}
			
			// Scan from right to left, find the first element that is greater than `pivot`.
			auto change = find_if(rfirst, pivot, bind1st(less<int>(), *pivot));
			
			swap(*change, *pivot);
			reverse(rfirst, pivot);
			
			return true;
		}
	};
\end{Code}

\begin{Code}
	void nextPermutation(vector<int> &num) {
        int i=num.size()-1;
        while(i>0 && num[i]<=num[i-1])
            i--;
        if(i){
            i--;
            int j=num.size()-1;
            while(j>=i && num[j]<=num[i])
                j--;
            swap(num[i],num[j]);
            i++;
        }
        reverse(num.begin()+i,num.end());
    }
\end{Code}
\subsubsection{相关题目}
\begindot
\item Permutation Sequence, 见 \S \ref{sec:permutation-sequence}
\item Permutations, 见 \S \ref{sec:permutations}
\item Permutations II, 见 \S \ref{sec:permutations-ii}
\item Combinations, 见 \S \ref{sec:combinations}
\myenddot


\subsection{Permutation Sequence} %%%%%%%%%%%%%%%%%%%%%%%%%%%%%%
\label{sec:permutation-sequence}



The set \fn{[1,2,3,…,n]} contains a total of $n!$ unique permutations.

By listing and labeling all of the permutations in order,
We get the following sequence (ie, for $n = 3$):
\begin{Code}
	"123"
	"132"
	"213"
	"231"
	"312"
	"321"
\end{Code}

Given $n$ and $k$, return the kth permutation sequence.

Note: Given $n$ will be between 1 and 9 inclusive.


【解题思路】
简单的,可以用暴力枚举法,调用 $k-1$ 次 \fn{next_permutation()}。

暴力枚举法把前 $k$个排列都求出来了,比较浪费,而我们只需要第$k$个排列。

利用康托编码的思路,假设有$n$个不重复的元素,第$k$个排列是$a_1, a_2, a_3, ...,
a_n$,那么$a_1$是哪一个位置呢?

我们把$a_1$去掉,那么剩下的排列为
$a_2, a_3, ..., a_n$, 共计$n-1$个元素,$n-1$个元素共有$(n-1)!$个排列,于是就可以知道
$a_1 = k / (n-1)!$。

同理,$a_2, a_3, ..., a_n$的值推导如下:

\begin{eqnarray}
	k_2 &=& k\%(n-1)! \nonumber \\
	a_2 &=& k_2/(n-2)! \nonumber \\
	\quad & \cdots \nonumber \\
	k_{n-1} &=& k_{n-2}\%2! \nonumber \\
	a_{n-1} &=& k_{n-1}/1! \nonumber \\
	a_n &=& 0 \nonumber
\end{eqnarray}


【使用next_permutation()】
\begin{Code}
	// LeetCode, Permutation Sequence
	// 使用next_permutation(),TLE
	class Solution {
		string getPermutation(int n, int k) {
			string s(n, '0');
			for (int i = 0; i < n; ++i)
				s[i] += i+1;
			for (int i = 0; i < k-1; ++i)
				next_permutation(s.begin(), s.end());
			return s;
		}
		
		template<typename BidiIt>
		bool next_permutation(BidiIt first, BidiIt last) {
			// Algorithm见上一题 Next Permutation
		}
	};
\end{Code}


【康托编码】
\begin{Code}
	// LeetCode, Permutation Sequence
	// 康托编码,时间复杂度O(n),空间复杂度O(1)
	class Solution {
		string getPermutation(int n, int k) {
			string s(n, '0');
			string result;
			for (int i = 0; i < n; ++i)
				s[i] += i + 1;
			return kth_permutation(s, k);
		}
		
		int factorial(int n) {
			int result = 1;
			for (int i = 1; i <= n; ++i)
				result *= i;
			return result;
		}
		
		// seq 已排好序,是第一个排列
		template<typename Sequence>
		Sequence kth_permutation(const Sequence &seq, int k) {
			const int n = seq.size();
			Sequence S(seq);
			Sequence result;
			int base = factorial(n - 1);
			--k;  // 康托编码从0开始
			
			for (int i = n - 1; i > 0; k %= base, base /= i, --i) {
				auto a = next(S.begin(), k / base);
			result.push_back(*a);
			S.erase(a);
		}
		
		result.push_back(S[0]); // 最后一个
		return result;
	}
};
\end{Code}


\subsubsection{相关题目}
\begindot
\item Next Permutation, 见 \S \ref{sec:next-permutation}
\item Permutations, 见 \S \ref{sec:permutations}
\item Permutations II, 见 \S \ref{sec:permutations-ii}
\item Combinations, 见 \S \ref{sec:combinations}
\myenddot


\subsection{Valid Sudoku} %%%%%%%%%%%%%%%%%%%%%%%%%%%%%%
\label{sec:valid-sudoku}


Determine if a Sudoku is valid, according to: Sudoku Puzzles - The Rules
\myurl{http://sudoku.com.au/TheRules.aspx} .

The Sudoku board could be partially filled, where empty cells are filled with
the character \fn{'.'}.

\begin{center}
	\includegraphics[width=150pt]{sudoku.png}\\
	\figcaption{A partially filled sudoku which is valid}\label{fig:sudoku}
\end{center}

【解题思路】
细节实现题。


【Algorithm】
\begin{Code}
	// LeetCode, Valid Sudoku
	// 时间复杂度O(n^2),空间复杂度O(1)
	class Solution {
		bool isValidSudoku(const vector<vector<char>>& board) {
			bool used[9];
			
			for (int i = 0; i < 9; ++i) {
				fill(used, used + 9, false);
				
				for (int j = 0; j < 9; ++j) // 检查行
					if (!check(board[i][j], used))
						return false;
				
				fill(used, used + 9, false);
				
				for (int j = 0; j < 9; ++j) // 检查列
					if (!check(board[j][i], used))
						return false;
			}
			
			for (int r = 0; r < 3; ++r) // 检查 9 个子格子
				for (int c = 0; c < 3; ++c) {
					fill(used, used + 9, false);
				
					for (int i = r * 3; i < r * 3 + 3; ++i)
						for (int j = c * 3; j < c * 3 + 3; ++j)
							if (!check(board[i][j], used))
								return false;
				}
				return true;
		}
		
		bool check(char ch, bool used[9]) {
			if (ch == '.') return true;
			if (used[ch - '1']) return false;
			return used[ch - '1'] = true;
		}
	};
\end{Code}


\subsubsection{相关题目}
\begindot
\item Sudoku Solver, 见 \S \ref{sec:sudoku-solver}
\myenddot


\subsection{Trapping Rain Water} %%%%%%%%%%%%%%%%%%%%%%%%%%%%%%
\label{sec:trapping-rain-water}



Given $n$ non-negative integers representing an elevation map where the width
of each bar is 1, compute how much water it is able to trap after raining.

For example,
Given \code{[0,1,0,2,1,0,1,3,2,1,2,1]}, return 6.

\begin{center}
	\includegraphics{trapping-rain-water.png}\\
	\figcaption{Trapping Rain Water}\label{fig:trapping-rain-water}
\end{center}


【解题思路】
对于每个柱子,找到其左右两边最高的柱子,该柱子能容纳的面积就是\code{min(max_left,
max_right) - height}。所以,
\begin{enumerate}
	\item 从左往右扫描一遍,对于每个柱子,求取左边最大值;
	\item 从右往左扫描一遍,对于每个柱子,求最大右值;
	\item 再扫描一遍,把每个柱子的面积并累加。
\end{enumerate}

也可以,
\begin{enumerate}
	\item 扫描一遍,找到最高的柱子,这个柱子将数组分为两半;
	\item 处理左边一半;
	\item 处理右边一半。
\end{enumerate}


【Algorithm1】
\begin{Code}
	// LeetCode, Trapping Rain Water
	// 思路1,时间复杂度O(n),空间复杂度O(n)
	class Solution {
		int trap(int A[], int n) {
			int *max_left = new int[n]();
			int *max_right = new int[n]();
			
			for (int i = 1; i < n; i++) {
				max_left[i] = max(max_left[i - 1], A[i - 1]);
				max_right[n - 1 - i] = max(max_right[n - i], A[n - i]);
			}
			
			int sum = 0;
			for (int i = 0; i < n; i++) {
				int height = min(max_left[i], max_right[i]);
				if (height > A[i])
					sum += height - A[i];
			}
			
			delete[] max_left;
			delete[] max_right;
			return sum;
		}
	};
\end{Code}

【Algorithm2】
\begin{Code}
	// LeetCode, Trapping Rain Water
	// 思路2,时间复杂度O(n),空间复杂度O(1)
	class Solution {
		int trap(int A[], int n) {
			int max = 0; // 最高的柱子,将数组分为两半
			for (int i = 0; i < n; i++)
				if (A[i] > A[max]) max = i;
			
			int water = 0;
			for (int i = 0, peak = 0; i < max; i++)
				if (A[i] > peak) peak = A[i];
				else water += peak - A[i];
			for (int i = n - 1, top = 0; i > max; i--)
				if (A[i] > top) top = A[i];
				else water += top - A[i];
			return water;
		}
	};
\end{Code}


【Algorithm3】
第三种解法,用一个栈辅助,小于栈顶的元素压入,大于等于栈顶就把栈里所有小于或等于当前值的元素全部出栈处理掉。
\begin{Code}
	// LeetCode, Trapping Rain Water
	// 用一个栈辅助,小于栈顶的元素压入,大于等于栈顶就把栈里所有小于或
	// 等于当前值的元素全部出栈处理掉,计算面积,最后把当前元素入栈
	// 时间复杂度O(n),空间复杂度O(n)
	class Solution {
		int trap(int a[], int n) {
			stack<pair<int, int>> s;
			int water = 0;
			for (int i = 0; i < n; ++i) {
				int height = 0;
				while (!s.empty()) { // 将栈里比当前元素矮或等高的元素全部处理掉
					int bar = s.top().first;
					int pos = s.top().second;
					// bar, height, a[i] 三者夹成的凹陷
					water += (min(bar, a[i]) - height) * (i - pos - 1);
					height = bar;
					
					if (a[i] < bar) break; // 碰到了比当前元素高的,跳出循环
					else s.pop(); // 弹出栈顶,因为该元素处理完了,不再需要了
				}
				s.push(make_pair(a[i], i));
			}
			return water;
		}
	};
\end{Code}


\subsubsection{相关题目}
\begindot
\item Container With Most Water, 见 \S \ref{sec:container-with-most-water}
\item Largest Rectangle in Histogram, 见 \S
\ref{sec:largest-rectangle-in-histogram}
\myenddot


\subsection{Rotate Image} %%%%%%%%%%%%%%%%%%%%%%%%%%%%%%
\label{sec:rotate-image}



You are given an $n \times n$ 2D matrix representing an image.

Rotate the image by 90 degrees (clockwise).

Follow up:
Could you do this in-place?


【解题思路】
首先想到,纯模拟,从外到内一圈一圈的转,但这个方法太慢。

如下图,首先沿着副对角线翻转一次,然后沿着水平中线翻转一次。

\begin{center}
	\includegraphics[width=200pt]{rotate-image.png}\\
	\figcaption{Rotate Image}\label{fig:rotate-image}
\end{center}

或者,首先沿着水平中线翻转一次,然后沿着主对角线翻转一次。

【Algorithm1】
\begin{Code}
	// LeetCode, Rotate Image
	// 思路 1,时间复杂度O(n^2),空间复杂度O(1)
	class Solution {
		void rotate(vector<vector<int>>& matrix) {
			const int n = matrix.size();
			for (int i = 0; i < n; ++i)  // 沿着副对角线反转
				for (int j = 0; j < n - i; ++j)
					swap(matrix[i][j], matrix[n - 1 - j][n - 1 - i]);
			for (int i = 0; i < n / 2; ++i) // 沿着水平中线反转
				for (int j = 0; j < n; ++j)
					swap(matrix[i][j], matrix[n - 1 - i][j]);
		}
	};
\end{Code}

【Algorithm2】
\begin{Code}
	// LeetCode, Rotate Image
	// 思路 2,时间复杂度O(n^2),空间复杂度O(1)
	class Solution {
		void rotate(vector<vector<int>>& matrix) {
			const int n = matrix.size();
			for (int i = 0; i < n / 2; ++i) // 沿着水平中线反转
				for (int j = 0; j < n; ++j)
					swap(matrix[i][j], matrix[n - 1 - i][j]);
			for (int i = 0; i < n; ++i)  // 沿着主对角线反转
				for (int j = i + 1; j < n; ++j)
					swap(matrix[i][j], matrix[j][i]);
		}
	};
\end{Code}


\subsubsection{相关题目}
\begindot
\item 无
\myenddot


Given a number represented as an array of digits, plus one to the number.


【解题思路】
高精度加法。


【Algorithm1】
\begin{Code}
	// LeetCode, Plus One
	// 时间复杂度O(n),空间复杂度O(1)
	class Solution {
		vector<int> plusOne(vector<int> &digits) {
			add(digits, 1);
			return digits;
		}

		// 0 <= digit <= 9
		void add(vector<int> &digits, int digit) {
			int c = digit;  // carry, 进位
			for (auto it = digits.rbegin(); it != digits.rend(); ++it) {
				*it += c;
				c = *it / 10;
				*it %= 10;
			}
			if (c > 0) digits.insert(digits.begin(), 1);
		}
	};
\end{Code}


【Algorithm2】
\begin{Code}
	// LeetCode, Plus One
	// 时间复杂度O(n),空间复杂度O(1)
	class Solution {
		vector<int> plusOne(vector<int> &digits) {
			add(digits, 1);
			return digits;
		}

		// 0 <= digit <= 9
		void add(vector<int> &digits, int digit) {
			int c = digit;  // carry, 进位
			for_each(digits.rbegin(), digits.rend(), [&c](int &d){
				d += c;
				c = d / 10;
				d %= 10;
			});
			if (c > 0) digits.insert(digits.begin(), 1);
		}
	};
\end{Code}




\subsection{Climbing Stairs} %%%%%%%%%%%%%%%%%%%%%%%%%%%%%%
\label{sec:climbing-stairs}



You are climbing a stair case. It takes $n$ steps to reach to the top.

Each time you can either climb 1 or 2 steps. In how many distinct ways can you
climb to the top?


【解题思路】
设$f(n)$表示爬$n$阶楼梯的不同方法数,为了爬到第$n$阶楼梯,有两个选择:
\begindot
\item 从第$n-1$阶前进1步;
\item 从第$n-1$阶前进2步;
\myenddot
因此,有$f(n)=f(n-1)+f(n-2)$。

这是一个斐波那契数列。

方法1,递归,太慢;方法2,迭代。

方法3,数学公式。斐波那契数列的通项公式为
$a_n=\dfrac{1}{\sqrt{5}}\left[\left(\dfrac{1+\sqrt{5}}{2}\right)^n-\left(\dfrac{1-\sqrt{5}}{2}\right)^n\right]$。


【迭代】
\begin{Code}
	// LeetCode, Climbing Stairs
	// 迭代,时间复杂度O(n),空间复杂度O(1)
	class Solution {
		int climbStairs(int n) {
			int prev = 0;
			int cur = 1;
			for(int i = 1; i <= n ; ++i){
				int tmp = cur;
				cur += prev;
				prev = tmp;
			}
			return cur;
		}
	};
\end{Code}


【数学公式】
\begin{Code}
	// LeetCode, Climbing Stairs
	// 数学公式,时间复杂度O(1),空间复杂度O(1)
	class Solution {
		int climbStairs(int n) {
			const double s = sqrt(5);
			return floor((pow((1+s)/2, n+1) + pow((1-s)/2, n+1))/s + 0.5);
		}
	};
\end{Code}


\subsubsection{相关题目}
\begindot
\item Decode Ways, 见 \S \ref{sec:decode-ways}
\myenddot


\subsection{Gray Code} %%%%%%%%%%%%%%%%%%%%%%%%%%%%%%
\label{sec:gray-code}



The gray code is a binary numeral system where two successive values differ in
only one bit.

Given a non-negative integer $n$ representing the total number of bits in the
code, print the sequence of gray code. A gray code sequence must begin with 0.

For example, given $n = 2$, return \fn{[0,1,3,2]}. Its gray code sequence is:
\begin{Code}
	00 - 0
	01 - 1
	11 - 3
	10 - 2
\end{Code}

Note:
\begindot
\item For a given $n$, a gray code sequence is not uniquely defined.
\item For example, \fn{[0,2,3,1]} is also a valid gray code sequence according
to the above definition.
\item For now, the judge is able to judge based on one instance of gray code
sequence. Sorry about that.
\myenddot


【解题思路】
格雷码(Gray Code)的定义请参考 \myurl{http://en.wikipedia.org/wiki/Gray_code}

\textbf{自然二进制码转换为格雷码:$g_0=b_0, g_i=b_i \oplus b_{i-1}$}, here, $\oplus\equiv \wedge$.

保留自然二进制码的最高位作为格雷码的最高位,格雷码次高位为二进制码的高位与次高位异或,其余各位与次高位的求法类似。例如,将自然二进制码1001,转换为格雷码的过程是:保留最高位;然后将第1位的1和第2位的0异或,得到1,作为格雷码的第2位;将第2位的0和第3位的0异或,得到0,作为格雷码的第3位;将第3位的0和第4位的1异或,得到1,作为格雷码的第4位,最终,格雷码为1101。

\textbf{格雷码转换为自然二进制码:$b_0=g_0, b_i=g_i \oplus b_{i-1}$}

保留格雷码的最高位作为自然二进制码的最高位,次高位为自然二进制高位与格雷码次高位异或,其余各位与次高位的求法类似。例如,将格雷码1000转换为自然二进制码的过程是:保留最高位1,作为自然二进制码的最高位;然后将自然二进制码的第1位1和格雷码的第2位0异或,得到1,作为自然二进制码的第2位;将自然二进制码的第2位1和格雷码的第3位0异或,得到1,作为自然二进制码的第3位;将自然二进制码的第3位1和格雷码的第4位0异或,得到1,作为自然二进制码的第4位,最终,自然二进制码为1111。

格雷码有\textbf{数学公式},整数$n$的格雷码是$n \oplus (n/2)$。

这题要求生成$n$比特的所有格雷码。

方法1,最简单的方法,利用数学公式,对从 $0\sim2^n-1$的所有整数,转化为格雷码。

方法2,$n$比特的格雷码,可以递归地从$n-1$比特的格雷码生成。如图\S
\ref{fig:gray-code-construction}所示。

\begin{center}
	\includegraphics[width=160pt]{gray-code-construction.png}\\
	\figcaption{The first few steps of the reflect-and-prefix
	method.}\label{fig:gray-code-construction}
\end{center}


【数学公式】
\begin{Code}
	// LeetCode, Gray Code
	// 数学公式,时间复杂度O(2^n),空间复杂度O(1)
	class Solution {
		vector<int> grayCode(int n) {
			vector<int> result;
			const size_t size = 1 << n;  // 2^n
			result.reserve(size);
			for (size_t i = 0; i < size; ++i)
				result.push_back(binary_to_gray(i));
			return result;
		}

		static unsigned int binary_to_gray(unsigned int n) {
			return n ^ (n >> 1);
		}
	};
\end{Code}

【Reflect-and-prefix method】
\begin{Code}
	// LeetCode, Gray Code
	// reflect-and-prefix method
	// 时间复杂度O(2^n),空间复杂度O(1)
	class Solution {
		vector<int> grayCode(int n) {
			vector<int> result;
			result.reserve(1<<n);
			result.push_back(0);
			for (int i = 0; i < n; i++) {
				const int highest_bit = 1 << i;
				for (int j = result.size() - 1; j >= 0; j--) // 要反着遍历,才能对称
					result.push_back(highest_bit | result[j]);
			}
			return result;
		}
	};
\end{Code}

\begin{Code}
	// LeetCode, Gray Code, By Simon Zhang
	// reflect-and-prefix method
	// 时间复杂度O(2^n),空间复杂度O(1),
	vector<int> grayCode(int n) {
		vector<int> res;
		res.push_back(0);
		for(int i=0;i<n;i++){
			int highbit = 1<<i;
			int curlen = res.size();
			for(int j=curlen-1;j>=0;j--)
			res.push_back(highbit | res[j]);
		}
		return res;
	}
\end{Code}



\subsection{Set Matrix Zeroes} %%%%%%%%%%%%%%%%%%%%%%%%%%%%%%
\label{sec:set-matrix-zeroes}



Given a $m \times n$ matrix, if an element is 0, set its entire row and column
to 0. Do it in place.

\textbf{Follow up:}
Did you use extra space?

A straight forward solution using $O(mn)$ space is probably a bad idea.

A simple improvement uses $O(m + n)$ space, but still not the best solution.

Could you devise a constant space solution?


【解题思路】
$O(m+n)$空间的方法很简单,设置两个bool数组,记录每行和每列是否存在0。

想要常数空间,可以复用第一行和第一列。


【Algorithm1】
\begin{Code}
	// LeetCode, Set Matrix Zeroes
	// 时间复杂度O(m*n),空间复杂度O(m+n)
	class Solution {
		void setZeroes(vector<vector<int> > &matrix) {
			const size_t m = matrix.size();
			const size_t n = matrix[0].size();
			vector<bool> row(m, false); // 标记该行是否存在0
			vector<bool> col(n, false); // 标记该列是否存在0
			
			for (size_t i = 0; i < m; ++i) {
				for (size_t j = 0; j < n; ++j) {
					if (matrix[i][j] == 0)
						row[i] = col[j] = true;
				}
			}
			
			for (size_t i = 0; i < m; ++i) {
				if (row[i])
					fill(&matrix[i][0], &matrix[i][0] + n, 0);
			}
			for (size_t j = 0; j < n; ++j)
				if (col[j])
					for (size_t i = 0; i < m; ++i)
						matrix[i][j] = 0;
		}
	};
\end{Code}


【Algorithm2】
\begin{Code}
	// LeetCode, Set Matrix Zeroes
	// 时间复杂度O(m*n),空间复杂度O(1)
	class Solution {
		void setZeroes(vector<vector<int> > &matrix) {
			const size_t m = matrix.size();
			const size_t n = matrix[0].size();
			bool row_has_zero = false; // 第一行是否存在 0
			bool col_has_zero = false; // 第一列是否存在 0
			
			for (size_t i = 0; i < n; i++)
				if (matrix[0][i] == 0) {
					row_has_zero = true;
					break;
				}
			
			for (size_t i = 0; i < m; i++)
				if (matrix[i][0] == 0) {
					col_has_zero = true;
					break;
				}
			
			for (size_t i = 1; i < m; i++)
				for (size_t j = 1; j < n; j++)
					if (matrix[i][j] == 0) {
						matrix[0][j] = 0;
						matrix[i][0] = 0;
					}
			for (size_t i = 1; i < m; i++)
				for (size_t j = 1; j < n; j++)
					if (matrix[i][0] == 0 || matrix[0][j] == 0)
						matrix[i][j] = 0;
			if (row_has_zero)
				for (size_t i = 0; i < n; i++)
					matrix[0][i] = 0;
			if (col_has_zero)
				for (size_t i = 0; i < m; i++)
					matrix[i][0] = 0;
		}
	};
\end{Code}


\subsection{Gas Station} %%%%%%%%%%%%%%%%%%%%%%%%%%%%%%
\label{sec:gas-station}



There are $N$ gas stations along a circular route, where the amount of gas at
station $i$ is \fn{gas[i]}.

You have a car with an unlimited gas tank and it costs \fn{cost[i]} of gas to
travel from station $i$ to its next station ($i$+1). You begin the journey with
an empty tank at one of the gas stations.

Return the starting gas station's index if you can travel around the circuit
once, otherwise return -1.

Note:
The solution is guaranteed to be unique.


【解题思路】
首先想到的是$O(N^2)$的解法,对每个点进行模拟。

$O(N)$的解法是,设置两个变量,\fn{sum}判断当前的指针的有效性;\fn{total}则判断整个数组是否有解,有就返回通过\fn{sum}得到的下标,没有则返回-1。

【Algorithm】
\begin{Code}
	// LeetCode, Gas Station
	// 时间复杂度O(n),空间复杂度O(1)
	class Solution {
		int canCompleteCircuit(vector<int> &gas, vector<int> &cost) {
			int total = 0;
			int j = -1;
			for (int i = 0, sum = 0; i < gas.size(); ++i) {
				sum += gas[i] - cost[i];
				total += gas[i] - cost[i];
				if (sum < 0) {
					j = i;
					sum = 0;
				}
			}
			return total >= 0 ? j + 1 : -1;
		}
	};
\end{Code}


There are $N$ children standing in a line. Each child is assigned a rating
value.

You are giving candies to these children subjected to the following
requirements:
\begindot
\item Each child must have at least one candy.
\item Children with a higher rating get more candies than their neighbors.
\myenddot

What is the minimum candies you must give?


【迭代版】
\begin{Code}
	// LeetCode, Candy
	// 时间复杂度O(n),空间复杂度O(n)
	class Solution {
		int candy(vector<int> &ratings) {
			const int n = ratings.size();
			vector<int> increment(n);
			
			// 左右各扫描一遍
			for (int i = 1, inc = 1; i < n; i++) {
				if (ratings[i] > ratings[i - 1])
					increment[i] = max(inc++, increment[i]);
				else	inc = 1;
			}
			
			for (int i = n - 2, inc = 1; i >= 0; i--) {
				if (ratings[i] > ratings[i + 1])
					increment[i] = max(inc++, increment[i]);
				else	inc = 1;
			}
			// 初始值为n,因为每个小朋友至少一颗糖
			return accumulate(&increment[0], &increment[0]+n, n);
		}
	};
\end{Code}


【递归版】
\begin{Code}
	// LeetCode, Candy
	// 备忘录法,时间复杂度O(n),空间复杂度O(n)
	// @author fancymouse (http://weibo.com/u/1928162822)
	class Solution {
		int candy(const vector<int>& ratings) {
			vector<int> f(ratings.size());
			int sum = 0;
			for (int i = 0; i < ratings.size(); ++i)
				sum += solve(ratings, f, i);
			return sum;
		}
		int solve(const vector<int>& ratings, vector<int>& f, int i) {
			if (f[i] == 0) {
				f[i] = 1;
				if (i > 0 && ratings[i] > ratings[i - 1])
					f[i] = max(f[i], solve(ratings, f, i - 1) + 1);
				if (i < ratings.size() - 1 && ratings[i] > ratings[i + 1])
					f[i] = max(f[i], solve(ratings, f, i + 1) + 1);
			}
			return f[i];
		}
	};
\end{Code}


\subsection{Single Number} %%%%%%%%%%%%%%%%%%%%%%%%%%%%%%
\label{sec:single-number}

Given an array of integers, every element appears twice except for one. Find
that single one.

Note:
Your algorithm should have a linear runtime complexity. Could you implement it
without using extra memory?


【解题思路】
异或,不仅能处理两次的情况,只要出现偶数次,都可以清零。


【Algorithm1】
\begin{Code}
	// LeetCode, Single Number
	// 时间复杂度O(n),空间复杂度O(1)
	class Solution {
		int singleNumber(int A[], int n) {
			int x = 0;
			for (size_t i = 0; i < n; ++i)
				x ^= A[i];
			return x;
		}
	};
\end{Code}


【Algorithm2】
\begin{Code}
	// LeetCode, Single Number
	// 时间复杂度O(n),空间复杂度O(1)
	class Solution {
		int singleNumber(int A[], int n) {
			return accumulate(A, A + n, 0, bit_xor<int>());
		}
	};
\end{Code}


\subsubsection{相关题目}
\begindot
\item  Single Number II, 见 \S \ref{sec:single-number-ii}
\myenddot


\subsection{Single Number II} %%%%%%%%%%%%%%%%%%%%%%%%%%%%%%
\label{sec:single-number-ii}



Given an array of integers, every element appears three times except for one.
Find that single one.

Note:
Your algorithm should have a linear runtime complexity. Could you implement it
without using extra memory?


【解题思路】
本题和上一题 Single Number,考察的是位运算。

方法1:创建一个长度为\fn{sizeof(int)}的数组\fn{count[sizeof(int)]},\fn{count[i]}表示在在$i$位出现的1的次数。如果\fn{count[i]}是3的整数倍,则忽略;否则就把该位取出来组成答案。

方法2:用\fn{one}记录到当前处理的元素为止,二进制1出现“1次”(mod 3 之后的
1)的有哪些二进制位;用\fn{two}记录到当前计算的变量为止,二进制1出现“2次”(mod 3 之后的
2)的有哪些二进制位。当\fn{one}和\fn{two}中的某一位同时为1时表示该二进制位上1出现了3次,此时需要清零。即\textbf{用二进制模拟三进制运算}。最终\fn{one}记录的是最终结果。

【Algorithm1】
\begin{Code}
	// LeetCode, Single Number II
	// 方法1,时间复杂度O(n),空间复杂度O(1)
	class Solution {
		int singleNumber(int A[], int n) {
			const int W = sizeof(int) * 8; // 一个整数的bit数,即整数字长
			int count[W];  // count[i]表示在在i位出现的1的次数
			fill_n(&count[0], W, 0);
			for (int i = 0; i < n; i++) {
				for (int j = 0; j < W; j++) {
					count[j] += (A[i] >> j) & 1;
					count[j] %= 3;
				}
			}
			int result = 0;
			for (int i = 0; i < W; i++) {
				result += (count[i] << i);
			}
			return result;
		}
	};
\end{Code}


【Algorithm2】
\begin{Code}
	// LeetCode, Single Number II
	// 方法2,时间复杂度O(n),空间复杂度O(1)
	class Solution {
		int singleNumber(int A[], int n) {
			int one = 0, two = 0, three = 0;
			for (int i = 0; i < n; ++i) {
				two |= (one & A[i]);
				one ^= A[i];
				three = ~(one & two);
				one &= three;
				two &= three;
			}
			return one;
		}
	};
\end{Code}


\subsubsection{相关题目}
\begindot
\item  Single Number, 见 \S \ref{sec:single-number}
\myenddot

\subsection{Vector Class}

Implement a vector-like data structure from scratch.

This question was to be done in C or C++.

Discussion topics:
1. Dealing with out of bounds accesses.
2. What happens when you need to increase the vector's size?
3. How many copies does the structure perform to insert n elements? That is, n
calls to vector.push_back

\subsection{N Parking Slots for N-1 Cars Sorting}
\label{sec:npark}

There are N parking slots and N-1 cars. Everytime you can move one car. How to move these cars into one given order.
BTW: I got this question from internet but i could not figure it out partially because the description is kind of incomplete to me. Anyone knowing this question or the solution?

【解题思路】

Sorting with O(1) space, e.g., insertsorting, selectsorting

\subsection{Word Search}
\label{sec:wordsearch}

Given a 2D board and a word, find if the word exists in the grid.

The word can be constructed from letters of sequentially adjacent cell, where "adjacent" cells are those horizontally or vertically neighboring. The same letter cell may not be
used more than once.

For example,
Given board =

[

	["ABCE"],

	["SFCS"],

	["ADEE"]

]

word = "ABCCED", -> returns true,

word = "SEE", -> returns true,

word = "ABCB", -> returns false. Anyone knowing this question or the solution?

【解题思路】
The idea of this question is as follows:
\begin{enumerate}
\item Find the 1st element of the word in the board.
\item For each position found where the 1st element lies, recursively do:
\begin{itemize}
\item[i)] Search the around cell to see if the next element exists. ($4$ directions: $(i-1,j),(i+1,j),(i,j-1),(i,j+1)$ )
\item[ii)] If the word ends, return true.
\end{itemize}
\item Return false if no matching found.
\end{enumerate}
Note: A mask matrix is needed to store the positions where have already been visited. Details can be found in code.

【Algorithm】
\begin{Code}
	// LeetCode, Word Search
	class Solution {
		bool search(vector<vector<char> > &board, int i, int j, string &word, int idx){
			if(idx == word.size()) return true;
			if(i<board.size() && j<board[i].size() && i >= 0 && j >= 0 && board[i][j] == word[idx]){
				char c = board[i][j];
				board[i][j] = '#';
				if(search(board, i+1, j, word, idx+1)) return true;
				if(search(board, i-1, j, word, idx+1)) return true;
				if(search(board, i, j+1, word, idx+1)) return true;
				if(search(board, i, j-1, word, idx+1)) return true;
				board[i][j] = c;
			}
			return false;
		}

		bool exist(vector<vector<char> > &board, string word) {
			if(board.empty()||board[0].empty()) return false;
			if(word.empty()) return true;
			for(int i=0; i<board.size();i++){
				for(int j=0;j<board[i].size();j++){
					if (board[i][j] == word[0]){
						vector<vector<char> > tmp (board);
						if(search(tmp, i, j, word, 0))
							return true;
					}
				}
			}
			return false;
		}
	};
\end{Code}


\subsubsection{相关题目}

\begindot
\item Robot Unique Paths, 见 \S \ref{sec:robotuniquepaths}
\myenddot

\subsection{Robot Unique Paths}
\label{sec:robotuniquepaths}

A robot is located at the top-left corner of a $m x n$ grid (marked 'Start' in the diagram below).

The robot can only move either down or right at any point in time. The robot is trying to reach the bottom-right corner of the grid (marked 'Finish' in the diagram below).

How many possible unique paths are there?

\begin{center}
	\includegraphics[width=0.6\linewidth]{robot-maze.png}\\
	\figcaption{Robot Unique Paths}\label{fig:robot-unique-paths}
\end{center}

Above is a 3 x 7 grid. How many possible unique paths are there?

{\bf Note:} $m$ and $n$ will be at most $100$.

【解题思路】
一维DP。 Step[i][j] = Step[i-1][j] + Step[i][j-1];

【Algorithm】
\begin{Code}
	// LeetCode, Robot Unique Paths
	class Solution {
		int uniquePaths(int m, int n) {
			vector<int> dp(n,0);
			dp[0] = 1;
			for(int i=0;i<m;i++){
				for(int j=1;j<n;j++)
					dp[j] = dp[j-1] + dp[j];
			}
			return dp[n-1];
		}
	};
\end{Code}


\subsubsection{相关题目}

\begindot
\item Word Search, 见 \S \ref{sec:wordsearch}
\myenddot

\subsection{FibonacciFun}
Implement a Fibonacci function.

\subsection{A+B=C+D}
You're given an array of integers(eg [3,4,7,1,2,9,8]). Find the index of values that satisfy A+B = C+D, where A,B,C and D are integers values in the array.

Eg: Given [3,4,7,1,2,9,8] array
The following
3+7 = 1+ 9 satisfies A+B=C+D
so print (0,2,3,5)

\subsection{Move Zeros at the rear of Array}
Given a number in an array form, come up with an algorithm to push all the zeros to the end.
Expectation : O(n) solution.

\subsection{Set Zero Row, Col and Depth}
Given a 3-D array, if any m[r][c][d] is <=0 mark all the cells in the entire row,col and depth as zero and return the o/p array

\subsection{Josephus Problem}
Delete every third element of an array until only one element is remaining. Tell the index of that remaining element in O(1) time complexity.

【解题思路】

This is the Josephus problem. There is a formula to calculate result directly only when $k=2$. For other cases, there are solutions with $O(n)$ or $O(klogn)$ time complexities. In the following, $n$ denotes the number of array in the initial circle, and $k$ denotes the count for each step, that is, $k-1$ people are skipped and the $k$-th is deleted. The people in the circle are numbered from $1$ to $n$.

The easiest way to solve this problem in the general case is to use dynamic programming by performing the first step and then using the solution of the remaining problem. When the index starts from one, then the person at $s$ shifts from the first person is in position $(s-1)\%n+1$, where $n$ is the total number of persons. Let $f(n,k)$ denote the position of the survivor. After the $k$-th person is killed, we're left with a circle of $n-1$, and we start the next count with the person whose number in the original problem was $(k\%n)+1$. The position of the survivor in the remaining circle would be $f(n-1,k)$ if we start counting at $1$; shifting this to account for the fact that we're starting at $k\%n+1$ yields the recurrence
$f(n,k)=(f(n-1,k)+k-1)\%n+1$, with $f(1,k)=1$, which takes the simpler form $g(n,k)=(g(n-1,k)+k)\%n$, with $g(1,k)=0$. If we number the position from $0$ to $n-1$ instead.


【Algorithm】
\begin{Code}
    int surviveWithK(int n, int k){
        return (N-pow(k-1,floor(logN/log(k-1))))+1;
    }
\end{Code}

\subsection{Absent Number in Array}
You are given an array of n integers which can contain integers from 1 to n only . Some elements can be repeated multiple times and some other elements can be absent from the array . Write a running code on paper which takes O(1) space apart from the input array and O(n) time to print which elements are not present in the array and the count of every element which is there in the array along with the element number .
NOTE: The array isn't necessarily sorted.

\subsection{Minimal Multiply Sum in Two Arrays}
Given two arrays of same size, arrange the arrays such that a1*b1 + a2*b2 + .... + an*bn should ne minimum.

\subsection{Divide Array into Two Half of Arrays}
Given a array of size n. Divide the array in to two arrays of size n/2,n/2. such that average of two arrays is equal.

\subsection{ maximum-sum subarray}
Find the maximum-sum subarray of an array.

\subsection{Top $k$-th Number}
Find the $k$-th maximal number in the given array.

\subsection{Array Indexing}
Given an array with huge number of elements. Following two operations can be performed on the array at any time\\
1. Find cumulative sum of first x numbers when x is input by user\\
2. Add/subtract value 't' from any index i of the array.

Find the most optimal way such that both the above requirements are optimized.

\subsection{Majority Element}
Find the majority element which occurs more than n/2 times in an array of n size, which contains duplicate elements in minimum time and space complexity.

\subsection{Pair Sum}
Array Pair Sum. Solve it in O(N) time complexity

\begin{Code}
    vector< pair<int, int> > pair_sum(vector<int> arr, int s){
    	vector< pair<int, int> > result;
    	unordered_map< int, bool> present;
    	for(int i=0; i<arr.size(); i++){
    		if(present.count(s-arr[i]) != 0)
    			result.push_back( make_pair(arr[i], s-arr[i]) );
    		present[arr[i]] = true;	
    	}
    	return result;
    }
\end{Code}

\subsection{Gap in 2D Array}
Given an $n x n$ matrix $A(i,j)$ of integers, find maximum value $A(c,d) - A(a,b)$ over all choices of indexes such that both $c > a$ and $d > b$ in $O(n^2)$.

\subsection{2D Array Union and Intersection}
Given two array of integers write two functions that will return an Union and Intersection

\subsection{Interleaving Sorting}
Given an array sort all the elements in even positions in ascending order and odd positions in descending order

\subsection{Pythogorean Triplets}
Given an array of numbers (integers) find all pythogorean triplets ($a^2 + b^2 = c^2$). print a,b an c and the indexes.

\section{单链表} %%%%%%%%%%%%%%%%%%%%%%%%%%%%%%

单链表节点的定义如下:
\begin{Code}
	// 单链表节点
	struct ListNode {
		int val;
		ListNode *next;
		ListNode(int x) : val(x), next(nullptr) { }
	};
\end{Code}

【Fabonacci数列】
Fabonacci数列为:$a_0=1,a_1=1,a_n=a_{n-1}+a_{n-2}$
快速幂优化

\subsection{Add Two Numbers}
\label{sec:add-two-numbers}



You are given two linked lists representing two non-negative numbers. The
digits are stored in reverse order and each of their nodes contain a single
digit. Add the two numbers and return it as a linked list.

Input: {\small \fontspec{Latin Modern Mono} (2 -> 4 -> 3) + (5 -> 6 -> 4)}

Output: {\small \fontspec{Latin Modern Mono} 7 -> 0 -> 8}


【解题思路】
跟Add Binary(见 \S \ref{sec:add-binary})很类似


【Algorithm】
\begin{Code}
	// LeetCode, Add Two Numbers
	// 跟Add Binary 很类似
	// 时间复杂度O(m+n),空间复杂度O(1)
	class Solution {
		ListNode *addTwoNumbers(ListNode *l1, ListNode *l2) {
			ListNode dummy(-1); // 头节点
			int carry = 0;
			ListNode *prev = &dummy;
			for (ListNode *pa = l1, *pb = l2;
				pa != nullptr || pb != nullptr;
				pa = pa == nullptr ? nullptr : pa->next,
				pb = pb == nullptr ? nullptr : pb->next,
				prev = prev->next) {
					const int ai = pa == nullptr ? 0 : pa->val;
					const int bi = pb == nullptr ? 0 : pb->val;
					const int value = (ai + bi + carry) % 10;
					carry = (ai + bi + carry) / 10;
					prev->next = new ListNode(value); // 尾插法
			}
			if (carry > 0)
				prev->next = new ListNode(carry);
			return dummy.next;
		}
	};
\end{Code}


\subsubsection{相关题目}

\begindot
\item Add Binary, 见 \S \ref{sec:add-binary}
\myenddot


\subsection{Reverse Linked List II}
\label{sec:reverse-linked-list-ii}



Reverse a linked list from position $m$ to $n$. Do it in-place and in one-pass.

For example:
Given \code{1->2->3->4->5->nullptr}, $m$ = 2 and $n$ = 4,

return \code{1->4->3->2->5->nullptr}.

Note:
Given m, n satisfy the following condition:
$1 \leq m \leq  n \leq $ length of list.


【解题思路】
这题非常繁琐,有很多边界检查,15分钟内做到bug free很有难度!


【Algorithm】
\begin{Code}
	// LeetCode, Reverse Linked List II
	// 迭代版,时间复杂度O(n),空间复杂度O(1)
	class Solution {
		ListNode *reverseBetween(ListNode *head, int m, int n) {
			ListNode dummy(-1);
			dummy.next = head;
			
			ListNode *prev = &dummy;
			for (int i = 0; i < m-1; ++i)
				prev = prev->next;
			ListNode* const head2 = prev;
			
			prev = head2->next;
			ListNode *cur = prev->next;
			for (int i = m; i < n; ++i) {
				prev->next = cur->next;
				cur->next = head2->next;
				head2->next = cur;  // 头插法
				cur = prev->next;
			}
			
			return dummy.next;
		}
	};
\end{Code}


\subsection{Partition List}
\label{sec:partition-list}



Given a linked list and a value $x$, partition it such that all nodes less than
$x$ come before nodes greater than or equal to $x$.

You should preserve the original relative order of the nodes in each of the two
partitions.

For example,
Given \code{1->4->3->2->5->2} and \code{x = 3}, return \code{1->2->2->4->3->5}.


【Algorithm】
\begin{Code}
	// LeetCode, Partition List
	// 时间复杂度O(n),空间复杂度O(1)
	class Solution {
		ListNode* partition(ListNode* head, int x) {
			ListNode left_dummy(-1); // 头结点
			ListNode right_dummy(-1); // 头结点
			
			auto left_cur = &left_dummy;
			auto right_cur = &right_dummy;
			
			for (ListNode *cur = head; cur; cur = cur->next) {
				if (cur->val < x) {
					left_cur->next = cur;
					left_cur = cur;
				} else {
					right_cur->next = cur;
					right_cur = cur;
				}
			}
		
			left_cur->next = right_dummy.next;
			right_cur->next = nullptr;
		
			return left_dummy.next;
		}
	};
\end{Code}


\subsection{Remove Duplicates from Sorted List}
\label{sec:remove-duplicates-from-sorted-list}



Given a sorted linked list, delete all duplicates such that each element appear
only once.

For example,

Given \code{1->1->2}, return \code{1->2}.

Given \code{1->1->2->3->3}, return \code{1->2->3}.


【递归版】
\begin{Code}
	// LeetCode, Remove Duplicates from Sorted List
	// 递归版,时间复杂度O(n),空间复杂度O(1)
	class Solution {
		ListNode *deleteDuplicates(ListNode *head) {
			if (!head) return head;
				ListNode dummy(head->val + 1); // 值只要跟head不同即可
			dummy.next = head;
			
			recur(&dummy, head);
			return dummy.next;
		}
		private:
		static void recur(ListNode *prev, ListNode *cur) {
			if (cur == nullptr) return;
			
			if (prev->val == cur->val) { // 删除head
				prev->next = cur->next;
				delete cur;
				recur(prev, prev->next);
			} else {
				recur(prev->next, cur->next);
			}
		}
	};
\end{Code}


【迭代版】
\begin{Code}
	// LeetCode, Remove Duplicates from Sorted List
	// 迭代版,时间复杂度O(n),空间复杂度O(1)
	class Solution {
		ListNode *deleteDuplicates(ListNode *head) {
			if (head == nullptr) return nullptr;
			
			for (ListNode *prev = head, *cur = head->next; cur; cur =
				cur->next) {
				if (prev->val == cur->val) {
					prev->next = cur->next;
					delete cur;
				} else {
					prev = cur;
				}
			}
			return head;
		}
	};
\end{Code}


\subsubsection{相关题目}

\begindot
\item Remove Duplicates from Sorted List II,见 \S
\ref{sec:remove-duplicates-from-sorted-list-ii}
\myenddot


\subsection{Remove Duplicates from Sorted List II}
\label{sec:remove-duplicates-from-sorted-list-ii}



Given a sorted linked list, delete all nodes that have duplicate numbers,
leaving only distinct numbers from the original list.

For example,

Given \code{1->2->3->3->4->4->5}, return \code{1->2->5}.

Given \code{1->1->1->2->3}, return \code{2->3}.


【递归版】
\begin{Code}
	// LeetCode, Remove Duplicates from Sorted List II
	// 递归版,时间复杂度O(n),空间复杂度O(1)
	class Solution {
		ListNode *deleteDuplicates(ListNode *head) {
			if (!head || !head->next) return head;
			
			ListNode *p = head->next;
			if (head->val == p->val) {
				while (p && head->val == p->val) {
					ListNode *tmp = p;
					p = p->next;
					delete tmp;
				}
				delete head;
				return deleteDuplicates(p);
			} else {
			head->next = deleteDuplicates(head->next);
			return head;
		}
	}
};
\end{Code}

【迭代版】
\begin{Code}
	// LeetCode, Remove Duplicates from Sorted List II
	// 迭代版,时间复杂度O(n),空间复杂度O(1)
	class Solution {
		ListNode *deleteDuplicates(ListNode *head) {
			if (head == nullptr) return head;
			
			ListNode dummy(INT_MIN); // 头结点
			dummy.next = head;
			ListNode *prev = &dummy, *cur = head;
			while (cur != nullptr) {
				bool duplicated = false;
				while (cur->next != nullptr && cur->val == cur->next->val) {
					duplicated = true;
					ListNode *temp = cur;
					cur = cur->next;
					delete temp;
				}
				if (duplicated) { // 删除重复的最后一个元素
					ListNode *temp = cur;
					cur = cur->next;
					delete temp;
					continue;
				}
				prev->next = cur;
				prev = prev->next;
				cur = cur->next;
			}
			prev->next = cur;
			return dummy.next;
		}
	};
\end{Code}


\subsubsection{相关题目}

\begindot
\item Remove Duplicates from Sorted List,见 \S
\ref{sec:remove-duplicates-from-sorted-list}
\myenddot


\subsection{Rotate List}
\label{sec:rotate-list}



Given a list, rotate the list to the right by $k$ places, where $k$ is
non-negative.

For example:
Given \code{1->2->3->4->5->nullptr} and \code{k = 2}, return
\code{4->5->1->2->3->nullptr}.


【解题思路】
先遍历一遍,得出链表长度$len$,注意$k$可能大于$len$,因此令$k \%=
len$。将尾节点next指针指向首节点,形成一个环,接着往后跑$len-k$步,从这里断开,就是要求的结果了。


【Algorithm】
\begin{Code}
	// LeetCode, Remove Rotate List
	// 时间复杂度O(n),空间复杂度O(1)
	class Solution {
		ListNode *rotateRight(ListNode *head, int k) {
			if (head == nullptr || k == 0) return head;
			
			int len = 1;
			ListNode* p = head;
			while (p->next) { // 求长度
				len++;
				p = p->next;
			}
			k = len - k % len;
			
			p->next = head; // 首尾相连
			for(int step = 0; step < k; step++)
				p = p->next;  //接着往后跑
			
			head = p->next; // 新的首节点
			p->next = nullptr; // 断开环
			return head;
		}
	};
\end{Code}


\subsection{Remove Nth Node From End of List}
\label{sec:remove-nth-node-from-end-of-list}



Given a linked list, remove the $n^{th}$ node from the end of list and return
its head.

For example, Given linked list: \code{1->2->3->4->5}, and $n$ = 2.

After removing the second node from the end, the linked list becomes
\code{1->2->3->5}.

Note:
\begindot
\item Given $n$ will always be valid.
\item Try to do this in one pass.
\myenddot


【解题思路】
设两个指针$p,q$,让$q$先走$n$步,然后$p$和$q$一起走,直到$q$走到尾节点,删除\fn{p->next}即可。


【Algorithm】
\begin{Code}
	// LeetCode, Remove Nth Node From End of List
	// 时间复杂度O(n),空间复杂度O(1)
	class Solution {
		ListNode *removeNthFromEnd(ListNode *head, int n) {
			ListNode dummy{-1, head};
			ListNode *p = &dummy, *q = &dummy;
			
			for (int i = 0; i < n; i++)  // q先走n步
				q = q->next;
			
			while(q->next) { // 一起走
				p = p->next;
				q = q->next;
			}
			ListNode *tmp = p->next;
			p->next = p->next->next;
			delete tmp;
			return dummy.next;
		}
	};
\end{Code}


\subsection{Swap Nodes in Pairs}
\label{sec:swap-nodes-in-pairs}



Given a linked list, swap every two adjacent nodes and return its head.

For example,
Given \code{1->2->3->4}, you should return the list as \code{2->1->4->3}.

Your algorithm should use only constant space. You may \emph{not} modify the
values in the list, only nodes itself can be changed.


【Algorithm】
\begin{Code}
	// LeetCode, Swap Nodes in Pairs
	// 时间复杂度O(n),空间复杂度O(1)
	class Solution {
		ListNode *swapPairs(ListNode *head) {
			if (head == nullptr || head->next == nullptr) return head;
			ListNode dummy(-1);
			dummy.next = head;
			
			for(ListNode *prev = &dummy, *cur = prev->next, *next = cur->next; next;
				prev = cur, cur = cur->next, next = cur ? cur->next: nullptr) {
				prev->next = next;
				cur->next = next->next;
				next->next = cur;
			}
			return dummy.next;
		}
	};
\end{Code}

下面这种写法更简洁,但题目规定了不准这样做。
\begin{Code}
	// LeetCode, Swap Nodes in Pairs
	// 时间复杂度O(n),空间复杂度O(1)
	class Solution {
		ListNode* swapPairs(ListNode* head) {
			ListNode* p = head;
			while (p && p->next) {
				swap(p->val, p->next->val);
				p = p->next->next;
			}
			return head;
		}
	};
\end{Code}

\subsubsection{相关题目}

\begindot
\item Reverse Nodes in k-Group, 见 \S \ref{sec:reverse-nodes-in-k-group}
\myenddot


\subsection{Reverse Nodes in k-Group}
\label{sec:reverse-nodes-in-k-group}



Given a linked list, reverse the nodes of a linked list k at a time and return
its modified list.

If the number of nodes is not a multiple of $k$ then left-out nodes in the end
should remain as it is.

You may not alter the values in the nodes, only nodes itself may be changed.

Only constant memory is allowed.

For example,
Given this linked list: \code{1->2->3->4->5}

For $k = 2$, you should return: \code{2->1->4->3->5}

For $k = 3$, you should return: \code{3->2->1->4->5}

【递归版】
\begin{Code}
	// LeetCode, Reverse Nodes in k-Group
	// 递归版,时间复杂度O(n),空间复杂度O(1)
	class Solution {
		ListNode *reverseKGroup(ListNode *head, int k) {
			if (head == nullptr || head->next == nullptr || k < 2)
				return head;
			
			ListNode *next_group = head;
			for (int i = 0; i < k; ++i) {
				if (next_group)
					next_group = next_group->next;
				else
					return head;
			}
			// next_group is the head of next group
			// new_next_group is the new head of next group after reversion
			ListNode *new_next_group = reverseKGroup(next_group, k);
			ListNode *prev = NULL, *cur = head;
			while (cur != next_group) {
				ListNode *next = cur->next;
				cur->next = prev ? prev : new_next_group;
				prev = cur;
				cur = next;
			}
			return prev; // prev will be the new head of this group
		}
	};
\end{Code}


【迭代版】
\begin{Code}
	// LeetCode, Reverse Nodes in k-Group
	// 迭代版,时间复杂度O(n),空间复杂度O(1)
	class Solution {
		ListNode *reverseKGroup(ListNode *head, int k) {
			if (head == nullptr || head->next == nullptr || k < 2) return head;
			ListNode dummy(-1);
			dummy.next = head;
			
			for(ListNode *prev = &dummy, *end = head; end; end = prev->next) {
				for (int i = 1; i < k && end; i++)
					end = end->next;
				if (end  == nullptr) break;  // 不足 k 个
				prev = reverse(prev, prev->next, end);
			}
			return dummy.next;
		}
		
		// prev 是 first 前一个元素, [begin, end] 闭区间,保证三者都不为 null
		// 返回反转后的倒数第1个元素
		ListNode* reverse(ListNode *prev, ListNode *begin, ListNode *end) {
			ListNode *end_next = end->next;
			for (ListNode *p = begin, *cur = p->next, *next = cur->next; cur != end_next;
				p = cur, cur = next, next = next ? next->next : nullptr) {
				cur->next = p;
			}
			begin->next = end_next;
			prev->next = end;
			return begin;
		}
	};
\end{Code}


\subsubsection{相关题目}
\begindot
\item Swap Nodes in Pairs, 见 \S \ref{sec:swap-nodes-in-pairs}
\myenddot


\subsection{Copy List with Random Pointer}
\label{sec:copy-list-with-random-pointer}



A linked list is given such that each node contains an additional random
pointer which could point to any node in the list or null.

Return a deep copy of the list.


【Algorithm】
\begin{Code}
	// LeetCode, Copy List with Random Pointer
	// 两遍扫描,时间复杂度O(n),空间复杂度O(1)
	class Solution {
		RandomListNode *copyRandomList(RandomListNode *head) {
			for (RandomListNode* cur = head; cur != nullptr; ) {
				RandomListNode* node = new RandomListNode(cur->label);
				node->next = cur->next;
				cur->next = node;
				cur = node->next;
			}
			
			for (RandomListNode* cur = head; cur != nullptr; ) {
				if (cur->random != NULL)
					cur->next->random = cur->random->next;
				cur = cur->next->next;
			}
			
			// 分拆两个单链表
			RandomListNode dummy(-1);
			for (RandomListNode* cur = head, *new_cur = &dummy; cur != nullptr;) {
				new_cur->next = cur->next;
				new_cur = new_cur->next;
				cur->next = cur->next->next;
				cur = cur->next;
			}
			return dummy.next;
		}
	};
\end{Code}


\subsection{Linked List Cycle}
\label{sec:Linked-List-Cycle}



Given a linked list, determine if it has a cycle in it.
Follow up:
Can you solve it without using extra space?


【解题思路】
最容易想到的方法是,用一个哈希表\fn{unordered_map<int, bool>
visited},记录每个元素是否被访问过,一旦出现某个元素被重复访问,说明存在环。空间复杂度$O(n)$,时间复杂度$O(N)$。

最好的方法是时间复杂度$O(n)$,空间复杂度$O(1)$的。设置两个指针,一个快一个慢,快的指针每次走两步,慢的指针每次走一步,如果快指针和慢指针相遇,则说明有环。参考\myurl{
 http://leetcode.com/2010/09/detecting-loop-in-singly-linked-list.html}


【Algorithm】
\begin{Code}
	//LeetCode, Linked List Cycle
	// 时间复杂度O(n),空间复杂度O(1)
	class Solution {
		bool hasCycle(ListNode *head) {
			// 设置两个指针,一个快一个慢
			ListNode *slow = head, *fast = head;
			while (fast && fast->next) {
				slow = slow->next;
				fast = fast->next->next;
				if (slow == fast) return true;
			}
			return false;
		}
	};
\end{Code}


\subsubsection{相关题目}
\begindot
\item Linked List Cycle II, 见 \S \ref{sec:Linked-List-Cycle-II}
\myenddot


\subsection{Linked List Cycle II}
\label{sec:Linked-List-Cycle-II}


Given a linked list, return the node where the cycle begins. If there is no
cycle, return \fn{null}.
Follow up:
Can you solve it without using extra space?

【解题思路】
当fast与slow相遇时,slow肯定没有遍历完链表,而fast已经在环内循环了$n$圈($1 \leq
n$)。假设slow走了$s$步,则fast走了$2s$步(fast步数还等于$s$加上在环上多转的$n$圈),设环长为$r$,则:
\begin{eqnarray}
	2s &=& s + nr \nonumber \\
	s &=& nr \nonumber
\end{eqnarray}

设整个链表长$L$,环入口点与相遇点距离为$a$,起点到环入口点的距离为$x$,则
\begin{eqnarray}
	x + a &=& nr = (n – 1)r +r = (n-1)r + L - x \nonumber \\
	x &=& (n-1)r + (L – x – a) \nonumber
\end{eqnarray}

$L – x –
a$为相遇点到环入口点的距离,由此可知,从链表头到环入口点等于$n-1$圈内环+相遇点到环入口点,于是我们可以从\fn{head}开始另设一个指针\fn{slow2},两个慢指针每次前进一步,它俩一定会在环入口点相遇。

【Algorithm】
\begin{Code}
	//LeetCode, Linked List Cycle II
	// 时间复杂度O(n),空间复杂度O(1)
	class Solution {
		ListNode *detectCycle(ListNode *head) {
			ListNode *slow = head, *fast = head;
			while (fast && fast->next) {
				slow = slow->next;
				fast = fast->next->next;
				if (slow == fast) {
					ListNode *slow2 = head;
					while (slow2 != slow) {
						slow2 = slow2->next;
						slow = slow->next;
					}
					return slow2;
				}
			}
			return nullptr;
		}
	};
\end{Code}


\subsubsection{相关题目}
\begindot
\item Linked List Cycle, 见 \S \ref{sec:Linked-List-Cycle}
\myenddot


\subsection{Reorder List}
\label{sec:Reorder-List}



Given a singly linked list $L: L_0 \rightarrow L_1 \rightarrow \cdots
\rightarrow L_{n-1} \rightarrow L_n$,
reorder it to: $L_0 \rightarrow L_n \rightarrow L_1 \rightarrow L_{n-1}
\rightarrow L_2 \rightarrow L_{n-2} \rightarrow \cdots$

You must do this in-place without altering the nodes' values.

For example,
Given \fn{\{1,2,3,4\}}, reorder it to \fn{\{1,4,2,3\}}.


【解题思路】
题目规定要in-place,也就是说只能使用$O(1)$的空间。

可以找到中间节点,断开,把后半截单链表reverse一下,再合并两个单链表。

【Algorithm】
\begin{Code}
	// LeetCode, Reorder List
	// 时间复杂度O(n),空间复杂度O(1)
	class Solution {
		void reorderList(ListNode *head) {
			if (head == nullptr || head->next == nullptr) return;
			ListNode *slow = head, *fast = head, *prev = nullptr;
			while (fast && fast->next) {
				prev = slow;
				slow = slow->next;
				fast = fast->next->next;
			}
			prev->next = nullptr; // cut at middle
			
			slow = reverse(slow);
			
			// merge two lists
			ListNode *curr = head;
			while (curr->next) {
				ListNode *tmp = curr->next;
				curr->next = slow;
				slow = slow->next;
				curr->next->next = tmp;
				curr = tmp;
			}
			curr->next = slow;
		}
		
		ListNode* reverse(ListNode *head) {
			if (head == nullptr || head->next == nullptr) return head;
			ListNode *prev = head;
			for (ListNode *curr = head->next, *next = curr->next; curr;
				prev = curr, curr = next, next = next ? next->next : nullptr)
				curr->next = prev;
			head->next = nullptr;
			return prev;
		}
	};
\end{Code}


\subsection{LRU Cache}
\label{sec:LRU-Cachet}



Design and implement a data structure for Least Recently Used (LRU) cache. It
should support the following operations: get and set.

\fn{get(key)} - Get the value (will always be positive) of the key if the key
exists in the cache, otherwise return -1.

\fn{set(key, value)} - Set or insert the value if the key is not already
present. When the cache reached its capacity, it should invalidate the least
recently used item before inserting a new item.


【解题思路】
为了使查找、插入和删除都有较高的性能,我们使用一个双向链表(\fn{std::list})和一个哈希表(\fn{std::unordered_map}),因为:
\begin{itemize}
	\item{哈希表保存每个节点的地址,可以基本保证在$O(1)$时间内查找节点}
	\item{双向链表插入和删除效率高,单向链表插入和删除时,还要查找节点的前驱节点}
\end{itemize}

具体实现细节:
\begin{itemize}
	\item{越靠近链表头部,表示节点上次访问距离现在时间最短,尾部的节点表示最近访问最少}
	\item{访问节点时,如果节点存在,把该节点交换到链表头部,同时更新hash表中该节点的地址}
	\item{插入节点时,如果cache的size达到了上限capacity,则删除尾部节点,同时要在hash表中删除对应的项;新节点插入链表头部}
	
\end{itemize}


【Algorithm】
\begin{Code}
	// LeetCode, LRU Cache
	// 时间复杂度O(logn),空间复杂度O(n)
	class LRUCache{
	private:
		struct CacheNode {
			int key;
			int value;
			CacheNode(int k, int v) :key(k), value(v){}
		};
	public:
		LRUCache(int capacity) {
			this->capacity = capacity;
		}
		
		int get(int key) {
			if (cacheMap.find(key) == cacheMap.end()) return -1;
			// 把当前访问的节点移到链表头部,并且更新map中该节点的地址
			cacheList.splice(cacheList.begin(), cacheList, cacheMap[key]);
			cacheMap[key] = cacheList.begin();
			return cacheMap[key]->value;
		}
		
		void set(int key, int value) {
			if (cacheMap.find(key) == cacheMap.end()) {
				if (cacheList.size() == capacity) { //删除链表尾部节点(最少访问的节点)
					cacheMap.erase(cacheList.back().key);
					cacheList.pop_back();
				}
				// 插入新节点到链表头部, 并且在map中增加该节点
				cacheList.push_front(CacheNode(key, value));
				cacheMap[key] = cacheList.begin();
			} else {//更新节点的值,把当前访问的节点移到链表头部,并且更新map中该节点的地址
				cacheMap[key]->value = value;
				cacheList.splice(cacheList.begin(), cacheList, cacheMap[key]);
				cacheMap[key] = cacheList.begin();
			}
		}
	private:
		list<CacheNode> cacheList;
		unordered_map<int, list<CacheNode>::iterator> cacheMap;
		int capacity;
	};
\end{Code}


\subsection{BST2DLL}
Convert a BST to sorted Double linked list


\subsection{Reverse Lists behind $k$-th Node}
1. Given number k, for Single linked list, skip k nodes and then reverse k nodes, till the end.\\
2. Write a program to reverse every K elements of a linked list.
Example: $K = 3$;
Input: $1->2->3->4->5->6->7->NULL$
Output: $3->2->1->6->5->4->7->NULL$

\subsection{Sort Linked List}
Sort a single linked list in place without using an additional node. 