\chapter{数学方法与常见模型}
数学对于算法很重要。没有好的数学基础,很难在算法上达到一定高度。本章介绍算法竞赛中常用的数学方法和模型。

\section{数论} %%%%%%%%%%%%%%%%%%%%%%%%%%%%%%

\subsection{欧几里德算法}

求最大公约数(greates common divisor)有很多方法,最经典的方法是欧几里德算法(Euclidean algorithm\footnote{\myurl{http://en.wikipedia.org/wiki/Euclid_algorithm}}),又称辗转相除法。

\begin{Codex}[label=gcd.c]
/**
 * @brief 求最大公约数,欧几里德算法,也即辗转相除法
 *
 * @param[in] a a
 * @param[in] b b
 * @return a和b的最大公约数
 */
unsigned int gcd(unsigned int a, unsigned int b) {
    if (b == 0) return a;
    return gcd(b, a % b);
}

/**
 * @brief 求最大公约数,欧几里德算法,迭代版本
 *
 * @param[in] a a
 * @param[in] b b
 * @return a和b的最大公约数
 */
unsigned int gcd1(unsigned int a, unsigned int b) {
    while (b != 0) {
        unsigned int tmp = b;
        b = a % b;
        a = tmp;
    }
    return a;
}

/**
 * @brief 求最大公约数,欧几里德算法,迭代版本,基于减法
 *
 * @param[in] a a
 * @param[in] b b
 * @return a和b的最大公约数
 */
unsigned int gcd2(unsigned int a, unsigned int b) {
    while (a != b) {
        if (a > b) {
            a -= b;
        } else {
            b -= a;
        }
    }
    return a;
}
\end{Codex}

求出了最大公约数,可以利用它来求最小公倍数(least common multiple), $\mathrm{lcm}(a,b)=a \times b / \gcd(a,b)$。

\subsubsection{例题}
\begindot
\item wikioi 1212 最大公约数 ,\myurl{http://www.wikioi.com/problem/1212/}
\myenddot


\subsection{扩展欧几里德算法}
定理:对于不完全为$0$的非负整数$a,b$,必然存在整数对 $x,y$ ,使得 $\gcd(a,b)=ax+by$。

这里$x$和$y$不一定是正数。扩展欧几里德算法(Extended Euclidean algorithm\footnote{\myurl{http://en.wikipedia.org/wiki/Extended_Euclidean_algorithm}})就是用来求$x$和$y$的。

\begin{Codex}[label=ex_gcd.c]
/**
 * @brief 扩展欧几里德算法
 * @param[in] a a
 * @param[in] b b
 * @param[out] x
 * @param[out] y
 * @return gcd(a,b)
 */
unsigned int ex_gcd(unsigned int a, unsigned int b, int *x, int *y) {
    if(b == 0) {
        *x = 1; *y = 0; return a;
    } else {
        const unsigned int tmp = ex_gcd(b, a % b, y, x);
        *y -= (*x)*(a/b);
        return tmp;
    }
}
\end{Codex}

扩展欧几里德算法的应用主要有以下三个:
\begindot
\item 求解不定方程;
\item 求解模线性方程(线性同余方程);
\item 求解模的逆元;
\myenddot


\subsection{求解不定方程}
求不定方程$ax+by=c$的整数解$x,y$,其中$a > b \geq c \geq 0$。

设$g=\gcd(a,b)$,方程$ax+by=g$的一组解是$(x_0,y_0)$,则当$c$是$g$的倍数时$ax+by=c$的一组解是$(x_0,y_0)$。证明略。

\begin{Code}
/**
 * @brief 求解不定方程ax+by=c
 * @param[in] a a
 * @param[in] b b
 * @param[in] c c
 * @param[out] x x
 * @param[out] y y
 * @return 是否有解,1表示有解,0表示无解
 */
int linear_equation(unsigned int a, unsigned int b, unsigned int c,
        int *x, int *y) {
    unsigned int k;
    const unsigned int g = ex_gcd(a, b, x, y);
    if (c % g) return 0;

    k = c / g;
    (*x) *= k; (*y) *= k;
    return 1;
}
\end{Code}


\subsection{求解模线性方程}
解方程$ax \equiv b\bmod n$,其中$a,b,n$是正整数。

$ax \equiv b\bmod n$的含义是“$ax$和$b$除以$n$的余数相同”,设这个倍数为$y$,则$ax-b=ny$,这恰好就是前面介绍过的不定方程。接下来的步骤就不用说了吧。

\begin{Code}
/**
 * @brief 求解模线性方程 ax = b mod n
 * @param[in] a
 * @param[in] b
 * @param[in] n n>0
 */
int modular_linear_equation(unsigned int a, unsigned int b, unsigned int n) {
    int x, y, x0, i;
    unsigned int k;
    unsigned int g = ex_gcd(a, n, &x, &y);
    if(b % g) return 0;

    k = b / g;
    x0 = (k * x) % n;   //特解
    for (i = 0; i < g; i++)
        printf("%d\n", (x0 + i * (n/g)) % n);
    return 1;
}
\end{Code}


\subsection{求解模的逆元}
有一个特殊情况要引起读者重视,当$b=1$时,$ax \equiv 1\bmod n$的解称为$a$关于模$n$的逆(inverse),它类似于“倒数”的概念。什么时候$a$的逆存在呢?根据上面的讨论,方程$ax-ny=1$要有解,这样,1必须是$\gcd(a,n)$的倍数,因此$a$和$n$必须互素。

\subsection{素数判定}
素数判定,又称素数测试(Primality test\footnote{\myurl{http://en.wikipedia.org/wiki/Primality_test}}),即给定一个正整数$n$,判断它是否是素数。

\subsubsection{暴力枚举法}
从2到$n$,依次作为除数,让$n$除以它们,只要有一个能整除$n$,则$n$不是素数。

\begin{Code}
/**
 * @brief 判断正整数n是否是素数
 * @param[in] n 正整数
 * @return 是,返回1,否,返回0
 */
int is_prime(unsigned int n) {
    int i;
    if (n < 2) return 0;
    for (i = 2; i < n; i++) {
        if (n % i == 0) return 0;
    }
    return 1;
}
\end{Code}

可以稍微做一点改进,从1到$\sqrt n$,依次作为除数。

\begin{Code}
/**
 * @brief 判断正整数n是否是素数,上界改为sqrt(n)
 * @param[in] n 正整数
 * @return 是,返回1,否,返回0
 */
int is_prime(unsigned int n) {
    int i;
    if (n < 2) return 0;
    const int upper = sqrt(n);

    for (i = 2; i <= upper; i++) {
        if (n % i == 0) return 0;
    }
    return 1;
}

/**
 * @brief 判断正整数n是否是素数,上界改为sqrt(n),但不使用sqrt()函数
 * @param[in] n 正整数
 * @return 是,返回1,否,返回0
 */
int is_prime1(unsigned int n) {
    int i;
    if (n < 2) return 0;
    for (i = 2; i*i <= n; i++) {
        if (n % i == 0) return 0;
    }
    return 1;
}
\end{Code}

\subsubsection{Eratosthenes筛法}
更高效的素数判定方法应该是预先计算出一张素数表,当判断一个数是否是素数时,直接查表即可。

怎样计算?用Eratosthenes筛法(Sieve of Eratosthenes\footnote{\myurl{http://en.wikipedia.org/wiki/Sieve_of_Eratosthenes}})

给出要筛数值的范围$n$,找出$\sqrt{n}$以内的素数$p_{1},p_{2},\dots,p_{k}$。先用2去筛,即把2留下,把2的倍数剔除掉;再用下一个质数,也就是3筛,把3留下,把3的倍数剔除掉;接下去用下一个质数5筛,把5留下,把5的倍数剔除掉;不断重复下去......。

\begin{Code}
#define MAXN 30000
/** prime_table[i]==1表示i是素数,等于0则不是素数 */
int prime_table[MAXN+1];

void compute_prime_table() {
    int i, j;
    const int upper = sqrt(MAXN);

    for (i = 2; i <= MAXN; i++) prime_table[i] = 1;
    prime_table[0] = 0;
    prime_table[1] = 0;

    for (i = 2; i < upper; i++) if(prime_table[i]) {
        for (j = 2; j * i <= MAXN; j++) prime_table[j*i] = 0;
    }
}

int is_prime(unsigned int n) {
    return prime_table[n];
}
\end{Code}

\subsubsection{暴力枚举法优化版}
下面这段代码是从 GNU GMP(\myurl{http://gmplib.org/})的源代码中抠出来的。

\begin{Code}
int is_prime(unsigned int n) {
    unsigned int q, r, d;    /* 商,余数,除数 */

    /* 把这个常数展开成二进制就容易理解了 */
    if (n < 32) return (0xa08a28acUL >> n) & 1;
    if ((n & 1) == 0) return 0;
    if (n % 3 == 0) return 0;
    if (n % 5 == 0) return 0;
    if (n % 7 == 0) return 0;

    for (d = 11;;) {
        q = n / d;
        r = n - q * d;
        if (q < d) return 1;    /* 保证 d 不超过 sqrt(n) */
        if (r == 0) break;

        d += 2;
        q = n / d;
        r = n - q * d;
        if (q < d) return 1;
        if (r == 0) break;

        d += 4;
    }
    return 0;
}
\end{Code}


\subsubsection{例题}
\begindot
\item wikioi 1430 素数判定,\myurl{http://www.wikioi.com/problem/1430/}
\myenddot


\subsection{大整数取模} %%%%%%%%%%%%%%%%%%%%%%%%%%%%%%
【解题思路】
求$a \bmod b, 1 \leq a \leq 10^{1000},1 \leq b \leq 100000$。

\subsubsection{输入}
每行2个整数$a$和$b$

\subsubsection{输出}
对每一行输出结果 $a \bmod b$

\subsubsection{样例输入}
\begin{Code}
2 3
12 7
152455856554521 3250
\end{Code}

\subsubsection{样例输出}
\begin{Code}
2
5
1521
\end{Code}

\subsubsection{代码}
\begin{Codex}[label=bigint_mod.c]
#include<stdio.h>
#include<string.h>

/* 一个数组元素表示4个十进制位,即数组是万进制的 */
#define BIGINT_MOD 10000
#define MOD_LEN 4
#define MAX_LEN (1000/MOD_LEN+1)  /* 整数的最大位数 */

char    a[MAX_LEN * MOD_LEN];
int     x[MAX_LEN], y;

/**
 * @brief 将输入的字符串转化为大整数,用数组表示,低位在低地址.
 * @param[in] s 输入的字符串
 * @param[out] x 大整数
 * @return 无
 */
void bigint_input(const char s[], int x[]) {
    int i, j = 0;
    const int len = strlen(s);
    for (i = 0; i < MAX_LEN; i++) x[i] = 0;

    /* for (i = len - 1; i >= 0; i--) a[j++] = s[i] - '0'; */
    for (i = len; i > 0; i -= MOD_LEN) {  /* [i-MOD_LEN, i) */
        int temp = 0;
        int k;
        const int low = i-MOD_LEN > 0 ? i-MOD_LEN : 0;
        for (k = low; k < i; k++) {
            temp = temp * 10 + s[k] - '0';
        }

        x[j++] = temp;
    }
}

/**
 * @brief 计算 x mod y
 * @param[in] x 大整数
 * @param[in] y 模
 * @return x mod y
 */
int bigint_mod(const int x[MAX_LEN], int y) {
    int ret = 0;
    int i;
    for (i = MAX_LEN-1; i >= 0; i--) {
        ret = (ret * BIGINT_MOD + x[i]) % y;
    }
    return ret;
}

int main() {
    while(scanf("%s%d", a, &y) > 1) {
        bigint_input(a, x);
        printf("%d\n", bigint_mod(x, y));
    }
    return 0;
}
\end{Codex}

\subsubsection{相关的题目}
与本题相同的题目:
\begindot
\item HDU 1212 Big Number, \myurl{http://acm.hdu.edu.cn/showproblem.php?pid=1212}
\myenddot

与本题相似的题目:
\begindot
\item  None
\myenddot


\subsection{单变元模线性方程} %%%%%%%%%%%%%%%%%%%%%%%%%%%%%%
【解题思路】
已知a,b,n, 求x,使得$ax\equiv b(\mod n)$.

\subsubsection{分析}
令$d=gcd(a,n)$,先使用扩展欧几里得求ax+ny=d的解。如果b不能整除d则无解,否则mod n意义下得解有d个,可以通过对某个解不断的加n/d得到。

\subsubsection{代码}
\begin{Codex}[label=line_mod.c]
	//O(logn)
	vector<long long> line_mod_equations(long long a,long long b,long long n){
		long long x,y;
		long long d=gcd(a,n,x,y);
		vector<long long> ans;
		if(d\%d==0){
			x\%=n; x+=n;x\%=n;
			ans.push_back(x*(b/d)\%(n/d));
			for(long long i=1;i<d;++i)
				ans.push_back((ans[0]+i\%n/d)\%n);
		}
		return ans;
	}
\end{Codex}

\subsection{中国剩余定理}
求出方程$x\equiv a_i(mod m_i),0 \leq i<n$的解x.
其中$m_1,m_2,\cdot,m_{n-1}$两两互质。

\subsubsection{分析}
令$M_i=\Pi_{j\neq i}m_j$, 因为$(M_i,m_i)=1$,故存在$p_i,q_i$, 使$M_ip_i+m_iq_i=1$。
令$e_i=M_ip_i$, 故$e_0a_0+e_1a_1+e_2a_2+\cdots+e_{n-1}a_{n-1}$是方程的一个解。在$[0,\Pi_{i=0}^{n-1})$,只有一个解。

\subsubsection{代码}
\begin{Codex}[label=ctr.c]
	//O(nlogm)
	ing ctr(int a[], int m[], int n){
		int M=1;
		for(int i=0;i<n; ++i){
			int x, y;
			int tm=M/m[i];
			extend_gcd(tm,m[i],x,y);
			ret=(ret+tm*x*a[i])\%M;
		}
		return (ret+M)%M;
	}
\end{Codex}

\subsection{欧拉函数}
欧拉函数表示小于或等于n的数中与n互质的数的数目。
欧拉函数求值的方法是:

(1) $\phi(1)=1$

(2)若n是素数p的k次幂,$\phi(n)=p^k-p^{k-1}=(p-1)p^{k-1}$

(3)若m,n互质,$\phi(nm)=\phi(n)\phi(m)$

因此,当p为n的最小质因数,若$p^2|n,\phi(n)=\phi(\frac{n}{p})\times p$;否则$\phi(n)=\phi(\frac{n}{p})\times (p-1)$

\section{组合数学} %%%%%%%%%%%%%%%%%%%%%%%%%%%%%%

\subsection{Sin Average}
【解题思路】
What is the average of $sin^2(x)$?
\subsubsection{思路}
The easiest way to see it is to note that $sin^2(x)+cos^2(x)=1$. Since $cos^2(x)$ is just $sin^2(x)$ shifted over by $90$ degrees, the average of $cos^2(x)$ and the average of 
$sin^2(x)$ over a full period are the same. Since the two functions sum to $1$, the average of each of them must therefore be $\frac{1}{2}$.

\subsection{全排列散列}
对于一个n的全排列,返回一个整数代表的它在所有排列中的排名。同样对于一个排名我们返回原排列,排名从0到n!-1。

\subsubsection{分析}
把排列开成一个多进制数。第i位的进制$(n-i+1)!$。设$a[i]=x$, 前$i-1$个有k个比x小.那么这一位应该用$(i-1-k)\times(n-i+1)!$来做为权值,最后加上所有位的权值即可。

用数求排列的时候,从高位到低位一位一位确定。用这一位的权值除以这一位对应的阶乘,若为k,那么从当前还没有用过的数中找出第k+1小得即可。