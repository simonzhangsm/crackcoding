\chapter{Parallel Programming}

【Note】 Asyn(), thread, ref, mutable, mutex

在进行多线程编程时,难免还要碰到两个问题,那就线程间的互斥与同步:
线程同步是指线程之间所具有的一种制约关系,一个线程的执行依赖另一个线程的消息,当它没有得到另一个线程的消息时应等待,直到消息到达时才被唤醒。
线程互斥是指对于共享的进程系统资源,在各单个线程访问时的排它性。当有若干个线程都要使用某一共享资源时,任何时刻最多只允许一个线程去使用,其它要使用该资源的线程必须等待,直到占用资源者释放该资源。线程互斥可以看成是一种特殊的线程同步(下文统称为同步)。

线程间的同步方法大体可分为两类:用户模式和内核模式。顾名思义,内核模式就是指利用系统内核对象的单一性来进行同步,使用时需要切换内核态与用户态,而用户模式就是不需要切换到内核态,只在用户态完成操作。

用户模式下的方法有:原子操作(例如一个单一的全局变量),临界区。内核模式下的方法有:事件,信号量,互斥量。

下面我们来分别看一下这些方法:

临界区(Critical Section)

保证在某一时刻只有一个线程能访问数据的简便办法。在任意时刻只允许一个线程对共享资源进行访问。如果有多个线程试图同时访问临界区,那么在有一个线程进入后其他所有试图访问此临界区的线程将被挂起,并一直持续到进入临界区的线程离开。临界区在被释放后,其他线程可以继续抢占,并以此达到用原子方式操作共享资源的目的。

临界区包含两个操作原语:
EnterCriticalSection() 进入临界区 
LeaveCriticalSection() 离开临界区

EnterCriticalSection()语句执行后代码将进入临界区以后无论发生什么,必须确保与之匹配的LeaveCriticalSection()都能够被执行到。否则临界区保护的共享资源将永远不会被释放。虽然临界区同步速度很快,但却只能用来同步本进程内的线程,而不可用来同步多个进程中的线程。


事件(Event) 

事件对象也可以通过通知操作的方式来保持线程的同步。并且可以实现不同进程中的线程同步操作。

信号量包含的几个操作原语: 
CreateEvent()    创建一个信号量 
OpenEvent()    打开一个事件 
SetEvent()    回置事件 
WaitForSingleObject()   等待一个事件 
WaitForMultipleObjects()  等待多个事件

WaitForMultipleObjects 函数原型: 
WaitForMultipleObjects( 
IN DWORD nCount, // 等待句柄数 
IN CONST HANDLE *lpHandles, //指向句柄数组 
IN BOOL bWaitAll, //是否完全等待标志 
IN DWORD dwMilliseconds //等待时间 
) 

参数nCount指定了要等待的内核对象的数目,存放这些内核对象的数组由lpHandles来指向。fWaitAll对指定的这nCount个内核对象的两种等待方式进行了指定,为TRUE时当所有对象都被通知时函数才会返回,为FALSE则只要其中任何一个得到通知就可以返回。dwMilliseconds在这里的作用与在WaitForSingleObject()中的作用是完全一致的。如果等待超时,函数将返回WAIT_TIMEOUT。
 

事件可以实现不同进程中的线程同步操作,并且可以方便的实现多个线程的优先比较等待操作,例如写多个WaitForSingleObject来代替WaitForMultipleObjects从而使编程更加灵活。 

互斥量(Mutex) 

互斥量跟临界区很相似,只有拥有互斥对象的线程才具有访问资源的权限,由于互斥对象只有一个,因此就决定了任何情况下此共享资源都不会同时被多个线程所访问。当前占据资源的线程在任务处理完后应将拥有的互斥对象交出,以便其他线程在获得后得以访问资源。互斥量比临界区复杂。因为使用互斥不仅仅能够在同一应用程序不同线程中实现资源的安全共享,而且可以在不同应用程序的线程之间实现对资源的安全共享。
 

互斥量包含的几个操作原语: 
CreateMutex()    创建一个互斥量 
OpenMutex()    打开一个互斥量 
ReleaseMutex()    释放互斥量 
WaitForMultipleObjects() 等待互斥量对象 

信号量(Semaphores)

信号量对象对线程的同步方式与前面几种方法不同,信号允许多个线程同时使用共享资源,这与操作系统中的PV操作相同。它指出了同时访问共享资源的线程最大数目。它允许多个线程在同一时刻访问同一资源,但是需要限制在同一时刻访问此资源的最大线程数目。在用CreateSemaphore()创建信号量时即要同时指出允许的最大资源计数和当前可用资源计数。一般是将当前可用资源计数设置为最大资源计数,每增加一个线程对共享资源的访问,当前可用资源计数就会减1,只要当前可用资源计数是大于0的,就可以发出信号量信号。但是当前可用计数减小到0时则说明当前占用资源的线程数已经达到了所允许的最大数目,不能在允许其他线程的进入,此时的信号量信号将无法发出。线程在处理完共享资源后,应在离开的同时通过ReleaseSemaphore()函数将当前可用资源计数加1。在任何时候当前可用资源计数决不可能大于最大资源计数。

PV操作及信号量的概念都是由荷兰科学家E.W.Dijkstra提出的。信号量S是一个整数,S大于等于零时代表可供并发进程使用的资源实体数,但S小于零时则表示正在等待使用共享资源的进程数。

P操作申请资源: 
(1)S减1; 
(2)若S减1后仍大于等于零,则进程继续执行; 
(3)若S减1后小于零,则该进程被阻塞后进入与该信号相对应的队列中,然后转入进程调度。

V操作释放资源: 
(1)S加1; 
(2)若相加结果大于零,则进程继续执行; 
(3)若相加结果小于等于零,则从该信号的等待队列中唤醒一个等待进程,然后再返回原进程继续执行或转入进程调度。 

信号量包含的几个操作原语: 
CreateSemaphore()  创建一个信号量 
OpenSemaphore()  打开一个信号量 
ReleaseSemaphore()  释放信号量 
WaitForSingleObject()  等待信号量 


信号量的使用特点使其更适用于对Socket(套接字)程序中线程的同步。例如,网络上的HTTP服务器要对同一时间内访问同一页面的用户数加以限制,这时可以为每一个用户对服务器的页面请求设置一个线程,而页面则是待保护的共享资源,通过使用信号量对线程的同步作用可以确保在任一时刻无论有多少用户对某一页面进行访问,只有不大于设定的最大用户数目的线程能够进行访问,而其他的访问企图则被挂起,只有在有用户退出对此页面的访问后才有可能进入。
 

综上所述:当在同一进程中的多线程同步时,临界区是效率最最高,基本不需要什么开销。而内核对象由于要进行用户态和内核态的切换,开销较大,但是内核对象由于可以命名,因此它们同时可以用于进程间的同步。另外,值得一提的是,信号量可以设置允许访问资源的线程或进
程个数,而不仅仅是只允许单个线程或进程访问资源。


\section{Questions}

第一题:线程的基本概念、线程的基本状态及状态之间的关系?

第二题:线程与进程的区别?

1、 线程是进程的一部分,所以线程有的时候被称为是轻权进程或者轻量级进程。\\
2、 一个没有线程的进程是可以被看作单线程的,如果一个进程内拥有多个进程,进程的执行过程不是一条线(线程)的,而是多条线(线程)共同完成的。\\
3、 系统在运行的时候会为每个进程分配不同的内存区域,但是不会为线程分配内存(线程所使用的资源是它所属的进程的资源),线程组只能共享资源。那就是说,
出了CPU之外(线程在运行的时候要占用CPU资源),计算机内部的软硬件资源的分配与线程无关,线程只能共享它所属进程的资源。\\
4、 与进程的控制表PCB相似,线程也有自己的控制表TCB,但是TCB中所保存的线程状态比PCB表中少多了。\\
5、 进程是系统所有资源分配时候的一个基本单位,拥有一个完整的虚拟空间地址,并不依赖线程而独立存在。

第三题:多线程有几种实现方法,都是什么?

1. 继承 Thread 类\\
2. 实现 Runnable 接口再 new Thread(YourRunnableOjbect) 

第四题:多线程同步和互斥有几种实现方法,都是什么?

线程间的同步方法大体可分为两类:用户模式和内核模式。顾名思义,内核模式就是指利用系统内核对象的单一性来进行同步,使用时需要切换内核态与用户态,而用户模式就是不需要切换到内核态,只在用户态完成操作。
用户模式下的方法有:原子操作(例如一个单一的全局变量),临界区。内核模式下的方法有:事件,信号量,互斥量。

第五题:多线程同步和互斥有何异同,在什么情况下分别使用他们?举例说明。

线程同步是指线程之间所具有的一种制约关系,一个线程的执行依赖另一个线程的消息,当它没有得到另一个线程的消息时应等待,直到消息到达时才被唤醒。
线程互斥是指对于共享的进程系统资源,在各单个线程访问时的排它性。当有若干个线程都要使用某一共享资源时,任何时刻最多只允许一个线程去使用,其它要使用该资源的线程必须等待,直到占用资源者释放该资源。线程互斥可以看成是一种特殊的线程同步(下文统称为同步)。

第三题(某培训机构的练习题):

子线程循环 10 次,接着主线程循环 100 次,接着又回到子线程循环 10 次,接着再回到主线程又循环 100 次,如此循环50次,试写出代码。

\begin{Code}
	#include<stdio.h> 
	#include<stdlib.h> 
	#include<errno.h> 
	#include<pthread.h> 
	#include<unistd.h> 
	int num=1; //全局变量 
	static pthread_mutex_t mutex=PTHREAD_MUTEX_INITIALIZER; 
	static pthread_cond_t cond=PTHREAD_COND_INITIALIZER; 
	void *func(); 
	int main() { 
		pthread_t tid; 
		int ret=0,i,j; 
		if((ret=pthread_create(&tid,NULL,func,NULL))!=0) 
			printf("create thread error!\n"); 
			for(i=0;i<3;i++) { 
				pthread_mutex_lock(&mutex); 
				if(num!=0) pthread_cond_wait(&cond,&mutex); 
				num=1; 
				for(j=0;j<10;j++) 
					printf("0"); 
				printf("\n"); 
				pthread_mutex_unlock(&mutex); 
				pthread_cond_signal(&cond); 
			} 
			return 0; 
	} 
	void *func() { 
		int i,j; 
		for(i=0;i<3;i++) { 
			pthread_mutex_lock(&mutex); 
			if(num!=1) pthread_cond_wait(&cond,&mutex); 
			num=0; 
			for(j=0;j<5;j++)
				printf("1"); 
			printf("\n"); 
			pthread_mutex_unlock(&mutex); 
			pthread_cond_signal(&cond); 
		} 
		pthread_exit(0); 
	}
\end{Code}

第四题(迅雷笔试题):

编写一个程序,开启3个线程,这3个线程的ID分别为A、B、C,每个线程将自己的ID在屏幕上打印10遍,要求输出结果必须按ABC的顺序显示;如:ABCABC….依次递推。

\begin{Code}
	#include<stdio.h> 
	#include<stdlib.h> 
	#include<error.h> 
	#include<unistd.h> 
	#include<pthread.h> 
	int num=0; 
	static pthread_mutex_t mutex=PTHREAD_MUTEX_INITIALIZER; 
	static pthread_cond_t cond=PTHREAD_COND_INITIALIZER; 
	void *func(void *); 
	int main() { 
		pthread_t tid[3]; 
		int ret=0,i; 
		for(i=0;i<3;i++) 
		if((ret=pthread_create(&tid[i],NULL,func,(void*)i))!=0) 
			printf("create thread_\%c error\n",i+'A'); 
		for(i=0;i<3;i++) 
			pthread_join(tid[i],NULL); 
		printf("\n"); 
		return 0; 
	} 
	void *func(void *argc) { 
		int i; 
		for(i=0;i<10;i++) { 
			pthread_mutex_lock(&mutex); 
			while(num!=(int)argc) 
				pthread_cond_wait(&cond,&mutex); 
			printf("\%c",num+'A'); 
			num=(num+1)\%3; 
			pthread_mutex_unlock(&mutex); 
			pthread_cond_broadcast(&cond); 
		} pthread_exit(0); 
	}
\end{Code}

第五题(Google面试题)

有四个线程1、2、3、4。线程1的功能就是输出1,线程2的功能就是输出2,以此类推.........现在有四个文件ABCD。初始都为空。现要让四个文件呈如下格式:

A:1 2 3 4 1 2....

B:2 3 4 1 2 3....

C:3 4 1 2 3 4....

D:4 1 2 3 4 1....

请设计程序。

\begin{Code}
	#include <iostream>
	#include <stdlib.h>
	#include <pthread.h>
	using namespace std;
	
	pthread_mutex_t myloack=PTHREAD_MUTEX_INITIALIZER;
	pthread_cond_t mycond=PTHREAD_COND_INITIALIZER;
	int n=0;
	void *ThreadFunc(void *arg){
		int num=(int )arg;
		for (int i = 0; i < 10; ++i){
			pthread_mutex_lock(&myloack);
			while (n!=num)
				pthread_cond_wait(&mycond,&myloack);
			
			if (num==0)	cout<<"1";
			else if(num==1)	cout<<"2";
			else if(num==2)	cout<<"3";
			else cout<<"4"<<endl;
			n=(n+1)\%4;
			pthread_mutex_unlock(&myloack);
			pthread_cond_broadcast(&mycond);
		}
		return (void *)0;
	}
	
	int  main(int argc, char const *argv[]){	
		pthread_t id[4];
		for (int i = 0; i < 4; ++i){
			int err=pthread_create(&id[i],NULL,ThreadFunc,(void *)i);
			if (err!=0){
				cout<<"create err:"<<endl;
				exit(-1);
			}
		}
		for (int i = 0; i < 4; ++i){
			int ret=pthread_join(id[i],NULL);
			if (ret!=0){
				cout<<"join err:"<<endl;
				exit(-1);
			}
		}
		return 0;
	}
\end{Code}

第六题 生产者消费者问题

有一个生产者在生产产品,这些产品将提供给若干个消费者去消费,为了使生产者和消费者能并发执行,在两者之间设置一个有多个缓冲区的缓冲池,生产者将它生产的产品放入一个缓冲区中,消费者可以从缓冲区中取走产品进行消费,所有生产者和消费者都是异步方式运行的,但它们必须保持同步,即不允许消费者到一个空的缓冲区中取产品,也不允许生产者向一个已经装满产品且尚未被取走的缓冲区中投放产品。

\begin{Code}
	#include<iostream>
	#include<thread>
	#include<mutex>
	#include<condition_variable>
	#include<array>
	#include<boost/circular_buffer.hpp>
	using namespace std;
	mutex m;
	condition_variable full,empty;
	boost::circular_buffer<int> Q(10);// 缓冲区大小为10,缓冲区数据为int,这里充当blocking queue
	bool flag=true;//一个简陋的设计,当不再生产时采用flag终止消费者线程
	void put(int x){
		for(int i=0;i<x;i++){
			unique_lock<mutex> lk(m);
			while(Q.full())
			empty.wait(lk);
			assert(!Q.full());
			Q.push_back(i);
			cout<<"@ "<<i<<endl;//生产
			full.notify_all();
		}
		flag=false;
	}
	void take(){
		while(flag){
			unique_lock<mutex> lk(m);
			while(Q.empty())
			full.wait(lk);
			if(flag){
				assert(!Q.empty());
				cout<<"# "<<Q.front()<<endl;//消费
				Q.pop_front();
				empty.notify_all();
			}
		}
	}
	int main(){
		thread one(take);
		thread two(take);
		thread three(take);
		put(100);
		one.join();
		two.join();
		three.join();
		return 0;
	}
\end{Code}

第七题 读者写者问题

有一个写者很多读者,多个读者可以同时读文件,但写者在写文件时不允许有读者在读文件,同样有读者读时写者也不能写。

\begin{Code}
	#include<iostream>
	#include<mutex>
	#include<thread>
	#include<memory>
	#include<vector>
	#include<assert.h>
	using namespace std;
	mutex m;
	shared_ptr<vector<int>> ptr;
	int loop=100;
	void read(){//读者
		while(1){
			{//放在块内可以使临时对象得到及时析构
				shared_ptr<vector<int>> temp_ptr;
				{
					unique_lock<mutex> lk(m);//这里读者和读者之间,读者和写者之间都互斥,但是临界区很小所以不用担心读者和读者间的互斥
					temp_ptr=ptr;//这里会使对象的引用计数加1
					assert(!temp_ptr.unique());
				}
				for(auto it=temp_ptr->begin();it!=temp_ptr->end();it++)//如果存在写者,那么读者访问的是旧的vector
				cout<<*it<<" ";
			}
		}
	}
	void write(){//写者
		for(int i=0;;i++){
			{//在一个块内使临时对象及时得到析构
				unique_lock<mutex> lk(m);//写者和写者之间,写者和读者之间都要互斥
				if(!ptr.unique())//如果存在其它写者或读者,则需要拷贝当前的vector
				ptr.reset(new vector<int>(*ptr));
				assert(ptr.unique());
				ptr->push_back(i);
			}
		}
	}
	int main(){
		ptr.reset(new vector<int>);
		thread one(read);
		thread two(read);
		write();
		one.join();
		two.join();
		return 0;
	}
\end{Code}