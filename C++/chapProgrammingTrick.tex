\chapter{编程技巧}
较大的数组放在main函数外,作为全局变量,这样可以防止栈溢出,因为栈的大小是有限制的。
如果能够预估栈,队列的上限,则不要用\fn{stack, queue},使用数组来模拟,这样速度最快。
输入数据一般放在全局变量,且在运行过程中不要修改这些变量。


在判断两个浮点数 (i.e., 
float/double) 
a和b是否相等时,不应该用\fn{a==b},而是判断二者之差的绝对值\fn{fabs(a-b)}是否小于某个阀值 
\fn{$\epsilon$=1e-9}。

判断一个整数是否是为奇数,用\fn{x \% 2 != 0}或\fn{x \& 1 != 0},不要用\fn{x \% 2 == 
1},因为x可能是负数。

用\fn{char}的值作为数组下标(例如,统计字符串中每个字符出现的次数),要考虑到\fn{char}可能是负数。有的人考虑到了,先强制转型为\fn{unsigned
 int}再用作下标,这仍然是错的。正确的做法是,先强制转型为\fn{unsigned char},再用作下标。这涉及C++整型提升的规则,就不详述了。

以下是关于STL基于《Effective STL》的使用技巧。

\subsubsection{vector和string优先于动态分配的数组}
首先,在性能上,由于\fn{vector}能够保证连续内存,因此一旦分配了后,它的性能跟原始数组相当;

其次,如果用new,意味着你要确保后面进行了delete,一旦忘记了,就会出现BUG,且这样需要都写一行delete,代码不够短;

再次,声明多维数组的话,只能一个一个new,例如:
\begin{Code}
	int** ary = new int*[row_num];
	for(int i = 0; i < row_num; ++i)
		ary[i] = new int[col_num];
\end{Code}

用vector的话一行代码搞定,
\begin{Code}
	vector<vector<int> > ary(row_num, vector<int>(col_num, 0));
\end{Code}
\subsubsection{使用reserve来避免不必要的重新分配}
\subsubsection{用empty来代替检查size()是否为0}
\subsubsection{确保new|delete和malloc|free的成对出现, 
使用智能指针shared_ptr和unique_ptr,替代auto_ptr容器}
\subsubsection{用distance和advance把const_iterator转化成iterator}

\section{数据结构与算法汇总}

1、常见数据结构

线性:数组,链表,队列,堆栈,块状数组(数组+链表),hash表,双端队列,位图(bitmap)\\
树:堆(大顶堆、小顶堆),trie树(字母树or字典树),后缀树,后缀树组,二叉排序/查找树,B+/B-,AVL树,Treap,红黑树,splay树,线段树,树状数组
图:图\\
其它:并查集\\

2、常见算法\\
(1)       基本思想:枚举,递归,分治,模拟,贪心,动态规划,剪枝,回溯\\
(2)       图算法:深度优先遍历与广度优先遍历, 最短路径,最小生成树,拓扑排序\\
(3)       字符串算法:字符串查找,hash算法,KMP算法\\
(4)       排序算法:冒泡,插入,选择,快排,归并排序,堆排序,桶排序\\
(5)       动态规划:背包问题,最长公共子序列,最优二分检索树\\
(6)       数论问题:素数问题,整数问题,进制转换,同余模运算,\\
(7)       排列组合:排列和组合算法\\
(8)       其它:LCA与RMQ问题