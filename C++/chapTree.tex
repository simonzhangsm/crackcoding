\chapter{树}

二叉树的节点定义如下:
\begin{Code}
	// 树的节点
	struct TreeNode {
		int val;
		TreeNode *left;
		TreeNode *right;
		TreeNode(int x) : val(x), left(nullptr), right(nullptr) { }
	};
\end{Code}

\section{BST} %%%%%%%%%%%%%%%%%%%%%%%%%%%%%%
\label{sec:binarySearchTree}

Given a BST and a number x, check whether exists two nodes in the BST whose sum equals to x. You can not use one extra array to serialize the BST and do a 2sum solver on it.

【Difference between hashmap and BST】

\subsubsection{相关题目}
\begindot
\item 2Sum, 见 \S \ref{sec:Two-sum}
\item 3Sum, 见 \S \ref{sec:3sum}
\item 3Sum Closest, 见 \S \ref{sec:3sum-closest}
\item 4Sum, 见 \S \ref{sec:4sum}
\myenddot

\section{二叉树的遍历} %%%%%%%%%%%%%%%%%%%%%%%%%%%%%%
\label{sec:binaryTreeTraversal}

在中序遍历中,一个节点的前驱,是其左子树的最右下角结点,后继,是其右子树的最左下角结点。

在后序遍历中,
\begindot
\item 若结点是根结点,则其后继为空;
\item 若结点是双亲的右子树,或是左子树但双亲无右子树,则其后继为双亲结点;
\item 若结点是双亲的左子树且双亲有右子树,则其后继为右子树按后序遍历的第一个结点
\myenddot


\begin{Codex}[label=binary_tree.cpp]
#include <iostream>
#include <stack>
#include <queue>

/** 结点的数据 */
typedef int tree_node_elem_t;
 /*
  *@struct
  *@brief 二叉树结点
  */
struct binary_tree_node_t {
    binary_tree_node_t *left;   /* 左孩子*/
    binary_tree_node_t *right;   /* 右孩子*/
    tree_node_elem_t elem; /* 结点的数据*/
};

/**
  * @brief 先序遍历,递归.
  * @param[in] root 根结点
  * @param[in] visit 访问数据元素的函数指针
  * @return 无
  */
void pre_order_r(const binary_tree_node_t *root,
                 int (*visit)(const binary_tree_node_t*)) {
    if (root == nullptr) return;

    visit(root);
    pre_order_r(root->left, visit);
    pre_order_r(root->right, visit);
}

/**
  * @brief 中序遍历,递归.
  */
void in_order_r(const binary_tree_node_t *root,
                int (*visit)(const binary_tree_node_t*)) {
    if(root == nullptr) return;

    in_order_r(root->left, visit);
    visit(root);
    in_order_r(root->right, visit);
}

/**
  * @brief 后序遍历,递归.
  */
void post_order_r(const binary_tree_node_t *root,
                  int (*visit)(const binary_tree_node_t*)) {
    if(root == nullptr) return;

    post_order_r(root->left, visit);
    post_order_r(root->right, visit);
    visit(root);
}

/**
 * @brief 先序遍历,非递归.
 */
void pre_order(const binary_tree_node_t *root,
               int (*visit)(const binary_tree_node_t*)) {
    const binary_tree_node_t *p;
    stack<const binary_tree_node_t *> s;

    p = root;

    if(p != nullptr) s.push(p);

    while(!s.empty()) {
        p = s.top();
        s.pop();
        visit(p);

        if(p->right != nullptr) s.push(p->right);
        if(p->left != nullptr) s.push(p->left);
    }
}

/**
 * @brief 中序遍历,非递归.
 */
void in_order(const binary_tree_node_t *root,
              int (*visit)(const binary_tree_node_t*)) {
    const binary_tree_node_t *p;
    stack<const binary_tree_node_t *> s;

    p = root;

    while(!s.empty() || p!=nullptr) {
        if(p != nullptr) {
            s.push(p);
            p = p->left;
        } else {
            p = s.top();
            s.pop();
            visit(p);
            p = p->right;
        }
    }
}

/**
 * @brief 后序遍历,非递归.
 */
void post_order(const binary_tree_node_t *root,
                int (*visit)(const binary_tree_node_t*)) {
    /* p,正在访问的结点,q,刚刚访问过的结点*/
    const binary_tree_node_t *p, *q;
    stack<const binary_tree_node_t *> s;

    p = root;

    do {
        while(p != nullptr) { /* 往左下走*/
            s.push(p);
            p = p->left;
        }
        q = nullptr;
        while(!s.empty()) {
            p = s.top();
            s.pop();
            /* 右孩子不存在或已被访问,访问之*/
            if(p->right == q) {
                visit(p);
                q = p; /* 保存刚访问过的结点*/
            } else {
                /* 当前结点不能访问,需第二次进栈*/
                s.push(p);
                /* 先处理右子树*/
                p = p->right;
                break;
            }
        }
    } while(!s.empty());
}

/**
 * @brief 层次遍历,也即BFS.
 *
 * 跟先序遍历一模一样,唯一的不同是栈换成了队列
 */
void level_order(const binary_tree_node_t *root,
                int (*visit)(const binary_tree_node_t*)) {
    const binary_tree_node_t *p;
    queue<const binary_tree_node_t *> q;

    p = root;

    if(p != nullptr) q.push(p);

    while(!q.empty()) {
        p = q.front();
        q.pop();
        visit(p);

        /*先左后右或先右后左无所谓*/
        if(p->left != nullptr) q.push(p->left);
        if(p->right != nullptr) q.push(p->right);
    }
}
\end{Codex}

\section{线索二叉树} %%%%%%%%%%%%%%%%%%%%%%%%%%%%%%
二叉树中存在很多空指针,可以利用这些空指针,指向其前驱或者后继。这种利用起来的空指针称为线索,这种改进后的二叉树称为线索二叉树(threaded binary tree)。

一棵n个结点的二叉树含有n+1个空指针。这是因为,假设叶子节点数为$n_0$,度为1的节点数为$n_1$,度为2的节点数为$n_2$,每个叶子节点有2个空指针,每个度为1的节点有1个空指针,则空指针的总数为$2n_0+n_1$,又有$n_0=n_2+1$(留给读者证明),因此空指针总数为$2n_0+n_1=n_0+n_2+1+n_1=n_0+n_1+n_2+1=n+1$。

在二叉树线索化过程中,通常规定,若无左子树,令left指向前驱,若无右子树,令rchild指向后继。还需要增加两个标志域表示当前指针是不是线索,例如ltag=1,表示left指向的是前驱,ltag=0,表示left指向的是左孩子,rtag类似。

二叉树的线索化,实质上就是遍历一棵树,只是在遍历的过程中,检查当前节点的左右指针是否为空,若为空,将它们改为指向前驱或后继的线索。

以中序线索二叉树为例,指针pre表示前驱,succ表示后继,如图~\ref{fig:threadedBinaryTree}所示。

\begin{center}
\includegraphics[width=300pt]{threaded-binary-tree.png} \\
\figcaption{中序线索二叉树}\label{fig:threadedBinaryTree}
\end{center}

在中序线索二叉树中,一个节点的前驱,是其左子树的最右下角结点,后继,是其右子树的最左下角结点。

中序线索二叉树的C语言实现如下。
\begin{Codex}[label=theaded_binary_tree.c]
/** @file threaded_binary_tree.c
  * @brief 线索二叉树.
  */
#include <stddef.h>    /* for NULL */
#include <stdio.h>

/* 结点数据的类型. */
typedef int elem_t;

 /**
  *@struct
  *@brief 线索二叉树结点.
  */
typedef struct tbt_node_t {
    int ltag; /** 1表示是线索,0表示是孩子 */
    int rtag; /** 1表示是线索,0表示是孩子 */
    struct tbt_node_t *left; /** 左孩子*/
    struct tbt_node_t *right; /** 右孩子*/
    elem_t elem; /** 结点所存放的数据*/
}tbt_node_t;

/* 内部函数 */
static void in_thread(tbt_node_t *p, tbt_node_t **pre);
static tbt_node_t *first(tbt_node_t *p);
static tbt_node_t *next(const tbt_node_t *p);

 /**
  * @brief 建立中序线索二叉树.
  * @param[in] root 树根
  * @return 无
  */
void create_in_thread(tbt_node_t *root) {
    /* 前驱结点指针*/
    tbt_node_t *pre=NULL;
    if(root != NULL) { /* 非空二叉树,线索化*/
        /* 中序遍历线索化二叉树*/
        in_thread(root, &pre);
        /* 处理中序最后一个结点*/
        pre->right = NULL;
        pre->rtag = 1;
    }
}


/**
  * @brief 在中序线索二叉树上执行中序遍历.
  * @param[in] root 树根
  * @param[in] visit 访问结点的数据的函数
  * @return 无
  */
void in_order(tbt_node_t *root, int(*visit)(tbt_node_t*)) {
    tbt_node_t *p;
    for(p = first(root); p != NULL; p = next(p)) {
        visit(p);
    }
}


 /*
  * @brief 中序线索化二叉树的主过程.
  * @param[in] p 当前要处理的结点
  * @param[inout] pre 当前结点的前驱结点
  * @return 无
  */
static void in_thread(tbt_node_t *p, tbt_node_t **pre) {
    if(p != NULL) {
        in_thread(p->left, pre); /* 线索化左子树 */
        if(p->left == NULL) {  /* 左子树为空,建立前驱 */
            p->left = *pre;
            p->ltag = 1;
        }
        /* 建立前驱结点的后继线索 */
        if((*pre) != NULL &&
            (*pre)->right == NULL) {
            (*pre)->right = p;
            (*pre)->rtag = 1;
        }
        *pre = p; /* 更新前驱 */
        in_thread(p->right, pre); /* 线索化右子树 */
    }
}

 /*
  * @brief 寻找线索二叉树的中序下的第一个结点.
  * @param[in] p 线索二叉树中的任意一个结点
  * @return 此线索二叉树的第一个结点
  */
static tbt_node_t *first(tbt_node_t *p) {
    if(p == NULL)  return NULL;

    while(p->ltag == 0) {
        p = p->left;  /* 最左下结点,不一定是叶结点*/
    }
    return p;
}

 /*
  * @brief 求中序线索二叉树中某结点的后继.
  * @param[in] p 某结点
  * @return p的后继
  */
static tbt_node_t *next(const tbt_node_t *p) {
    if(p->rtag == 0) {
        return first(p->right);
    } else {
        return p->right;
    }
}
\end{Codex}

中序线索二叉树最简单,在中序线索的基础上稍加修改就可以实现先序,后续就要再费点心思了。


\section{Morris Traversal} %%%%%%%%%%%%%%%%%%%%%%%%%%%%%%
通过前面第\S \ref{sec:binaryTreeTraversal}节,我们知道,实现二叉树的前序(preorder)、中序(inorder)、后序(postorder)遍历有两个常用的方法,一是递归(recursive),二是栈(stack+iterative)。这两种方法都是O(n)的空间复杂度。

而Morris Traversal只需要O(1)的空间复杂度。这种算法跟线索二叉树很像,不过Morris Traversal一边建线索,一边访问数据,访问完后销毁线索,保持二叉树不变。

\subsection{Morris中序遍历}
Morris中序遍历的步骤如下:
\begin{enumerate}
\item 初始化当前节点cur为root节点
\item 如果cur没有左孩子,则输出当前节点并将其右孩子作为当前节点,即cur = cur->rchild。
\item 如果cur有左孩子,则寻找cur的前驱,即cur的左子树的最右下角结点。\\
   a) 如果前驱节点的右孩子为空,将它的右孩子指向当前节点,当前节点更新为当前节点的左孩子。\\
   b) 如果前驱节点的右孩子为当前节点,将它的右孩子重新设为空(恢复树的形状),输出当前节点,当前节点更新为当前节点的右孩子。
\item 重复2、3步骤,直到cur为空。
\end{enumerate}
如图~\ref{fig:inorderMorris}所示,cur表示当前节点,深色节点表示该节点已输出。

\begin{center}
\includegraphics[width=360pt]{inorder-morris-traversal.png} \\
\figcaption{Morris中序遍历}\label{fig:inorderMorris}
\end{center}

C语言实现见第\S\ref{sec:morrisTraversalImpl}节。

\subsubsection{相关的题目}
\begindot
\item Leet Code - Binary Tree Inorder Traversal, \myurl{http://leetcode.com/onlinejudge\#question_94}
\myenddot


\subsection{Morris先序遍历}
Morris先序遍历的步骤如下:
\begin{enumerate}
\item 初始化当前节点cur为root节点
\item 如果cur没有左孩子,则输出当前节点并将其右孩子作为当前节点,即cur = cur->rchild。
\item 如果cur有左孩子,则寻找cur的前驱,即cur的左子树的最右下角结点。\\
   a) 如果前驱节点的右孩子为空,将它的右孩子指向当前节点,\textbf{输出当前节点(在这里输出,这是与中序遍历唯一的不同点)}当前节点更新为当前节点的左孩子。\\
   b) 如果前驱节点的右孩子为当前节点,将它的右孩子重新设为空(恢复树的形状),\sout{输出当前节点,}当前节点更新为当前节点的右孩子。
\item 重复2、3步骤,直到cur为空。
\end{enumerate}
如图~\ref{fig:preorderMorris}所示。

\begin{center}
\includegraphics[width=360pt]{preorder-morris-traversal.png} \\
\figcaption{Morris先序遍历}\label{fig:preorderMorris}
\end{center}

C语言实现见第\S\ref{sec:morrisTraversalImpl}节。


\subsection{Morris后序遍历}
Morris后续遍历稍微复杂,需要建立一个临时节点dump,令其左孩子是root,并且还需要一个子过程,就是倒序输出某两个节点之间路径上的所有节点。

Morris后序遍历的步骤如下:
\begin{enumerate}
\item 初始化当前节点cur为root节点
\item 如果cur没有左孩子,则\sout{输出当前节点并}将其右孩子作为当前节点,即cur = cur->rchild。
\item 如果cur有左孩子,则寻找cur的前驱,即cur的左子树的最右下角结点。\\
   a) 如果前驱节点的右孩子为空,将它的右孩子指向当前节点,当前节点更新为当前节点的左孩子。\\
   b) 如果前驱节点的右孩子为当前节点,将它的右孩子重新设为空(恢复树的形状),\sout{输出当前节点,}\textbf{倒序输出从当前节点的左孩子到该前驱节点这条路径上的所有节点。}当前节点更新为当前节点的右孩子。
\item 重复2、3步骤,直到cur为空。
\end{enumerate}
如图~\ref{fig:postorderMorris}所示。

\begin{center}
\includegraphics[width=360pt]{postorder-morris-traversal.png} \\
\figcaption{Morris后序遍历}\label{fig:postorderMorris}
\end{center}

C语言实现见第\S\ref{sec:morrisTraversalImpl}节。


\subsection{C语言实现}
\label{sec:morrisTraversalImpl}
\begin{Codex}[label=morris_traversal.c]
/** @file morris_traversal.c
 * @brief Morris遍历算法.
 */
#include<stdio.h>
#include<stdlib.h>

/* 结点数据的类型. */
typedef int elem_t;

/**
 *@struct
 *@brief 二叉树结点.
 */
typedef struct bt_node_t {
    elem_t elem; /* 节点的数据 */
    struct bt_node_t *left; /* 左孩子 */
    struct bt_node_t *right; /* 右孩子 */
} bt_node_t;

/**
 * @brief 中序遍历,Morris算法.
 * @param[in] root 根节点
 * @param[in] visit 访问函数
 * @return 无
 */
void in_order_morris(bt_node_t *root, int(*visit)(bt_node_t*)) {
    bt_node_t *cur, *prev;

    cur = root;
    while (cur != NULL ) {
        if (cur->left == NULL ) {
            visit(cur);
            prev = cur;
            cur = cur->right;
        } else {
            /* 查找前驱 */
            bt_node_t *node = cur->left;
            while (node->right != NULL && node->right != cur)
                node = node->right;

            if (node->right == NULL ) { /* 还没线索化,则建立线索 */
                node->right = cur;
                /* prev = cur; 不能有这句,cur还没有被访问 */
                cur = cur->left;
            } else {    /* 已经线索化,则访问节点,并删除线索  */
                visit(cur);
                node->right = NULL;
                prev = cur;
                cur = cur->right;
            }
        }
    }
}

/**
 * @brief 先序遍历,Morris算法.
 * @param[in] root 根节点
 * @param[in] visit 访问函数
 * @return 无
 */
void pre_order_morris(bt_node_t *root, int (*visit)(bt_node_t*)) {
    bt_node_t *cur, *prev;

    cur = root;
    while (cur != NULL ) {
        if (cur->left == NULL ) {
            visit(cur);
            prev = cur; /* cur刚刚被访问过 */
            cur = cur->right;
        } else {
            /* 查找前驱 */
            bt_node_t *node = cur->left;
            while (node->right != NULL && node->right != cur)
                node = node->right;

            if (node->right == NULL ) { /* 还没线索化,则建立线索 */
                visit(cur); /* 仅这一行的位置与中序不同 */
                node->right = cur;
                prev = cur; /* cur刚刚被访问过 */
                cur = cur->left;
            } else {    /* 已经线索化,则删除线索  */
                node->right = NULL;
                /* prev = cur; 不能有这句,cur已经被访问 */
                cur = cur->right;
            }
        }
    }
}


static void reverse(bt_node_t *from, bt_node_t *to);
static void visit_reverse(bt_node_t* from, bt_node_t *to,
        int (*visit)(bt_node_t*));
/**
 * @brief 后序遍历,Morris算法.
 * @param[in] root 根节点
 * @param[in] visit 访问函数
 * @return 无
 */
void post_order_morris(bt_node_t *root, int (*visit)(bt_node_t*)) {
    bt_node_t dummy = { 0, NULL, NULL };
    bt_node_t *cur, *prev = NULL;

    dummy.left = root;
    cur = &dummy;
    while (cur != NULL ) {
        if (cur->left == NULL ) {
            prev = cur; /* 必须要有 */
            cur = cur->right;
        } else {
            bt_node_t *node = cur->left;
            while (node->right != NULL && node->right != cur)
                node = node->right;

            if (node->right == NULL ) { /* 还没线索化,则建立线索 */
                node->right = cur;
                prev = cur; /* 必须要有 */
                cur = cur->left;
            } else { /* 已经线索化,则访问节点,并删除线索  */
                visit_reverse(cur->left, prev, visit);  // call print
                prev->right = NULL;
                prev = cur; /* 必须要有 */
                cur = cur->right;
            }
        }
    }
}

/*
 * @brief 逆转路径.
 * @param[in] from from
 * @param[to] to to
 * @return 无
 */
static void reverse(bt_node_t *from, bt_node_t *to) {
    bt_node_t *x = from, *y = from->right, *z;
    if (from == to) return;

    while (x != to) {
        z = y->right;
        y->right = x;
        x = y;
        y = z;
    }
}

/*
 * @brief  访问逆转后的路径上的所有结点.
 * @param[in] from from
 * @param[to] to to
 * @return 无
 */
static void visit_reverse(bt_node_t* from, bt_node_t *to,
        int (*visit)(bt_node_t*)) {
    bt_node_t *p = to;
    reverse(from, to);

    while (1) {
        visit(p);
        if (p == from)
            break;
        p = p->right;
    }

    reverse(to, from);
}

/*
 * @brief 分配一个新节点.
 * @param[in] e 新节点的数据
 * @return 新节点
 */
bt_node_t* new_node(int e) {
    bt_node_t* node = (bt_node_t*) malloc(sizeof(bt_node_t));
    node->elem = e;
    node->left = NULL;
    node->right = NULL;

    return (node);
}

static int print(bt_node_t *node) {
    printf(" %d ", node->elem);
    return 0;
}

/* test */
int main() {
    /* 构造的二叉树如下
       1
     /   \
    2      3
  /  \
4     5
     */
    bt_node_t *root = new_node(1);
    root->left = new_node(2);
    root->right = new_node(3);
    root->left->left = new_node(4);
    root->left->right = new_node(5);

    in_order_morris(root, print);
    printf("\n");
    pre_order_morris(root, print);
    printf("\n");
    post_order_morris(root, print);
    printf("\n");

    return 0;
}
\end{Codex}


\section{重建二叉树} %%%%%%%%%%%%%%%%%%%%%%%%%%%%%%
\begin{Codex}[label=binary_tree_rebuild.c]
#include <stdio.h>
#include <stdlib.h>
#include <string.h>
#include <stddef.h>
/**
 * @brief 给定前序遍历和中序遍历,输出后序遍历.
 *
 * @param[in] pre 前序遍历的序列
 * @param[in] in 中序遍历的序列
 * @param[in] n 序列的长度
 * @param[out] post 后续遍历的序列
 * @return 无
 */
void build_post(const char * pre, const char *in, const int n, char *post) {
    int left_len = strchr(in, pre[0]) - in;
    if(n <= 0) return;
    
    build_post(pre + 1, in, left_len, post);
    build_post(pre + left_len + 1, in + left_len + 1,
            n - left_len - 1, post + left_len);
    post[n - 1] = pre[0];
}

#define MAX  64
// 测试
// BCAD CBAD,输出 CDAB
// DBACEGF ABCDEFG,输出 ACBFGED
void build_post_test() {
    char pre[MAX] = {0};
    char in[MAX] = {0};
    char post[MAX] = {0};
    int n;

    scanf("%s%s", pre, in);
    n = strlen(pre);

    build_post(pre, in, n, post);
    printf("%s\n", post);
}

/* 结点数据的类型. */
typedef char elem_t;

/**
 *@struct
 *@brief 二叉树结点.
 */
typedef struct bt_node_t {
    elem_t elem; /* 节点的数据 */
    struct bt_node_t *left; /* 左孩子 */
    struct bt_node_t *right; /* 右孩子 */
} bt_node_t;

/**
 * @brief 给定前序遍历和中序遍历,重建二叉树.
 *
 * @param[in] pre 前序遍历的序列
 * @param[in] in 中序遍历的序列
 * @param[in] n 序列的长度
 * @param[out] root 根节点
 * @return 无
 */
void rebuild(const char *pre, const char *in, int n, bt_node_t **root) {
    int left_len;
    // 检查终止条件
    if (n <= 0 || pre == NULL || in == NULL)
        return;
    //获得前序遍历的第一个结点
    *root = (bt_node_t*) malloc(sizeof(bt_node_t));
    (*root)->elem = *pre;
    (*root)->left = NULL;
    (*root)->right = NULL;

    left_len = strchr(in, pre[0]) - in;
    //重建左子树
    rebuild(pre + 1, in, left_len, &((*root)->left));
    //重建右子树
    rebuild(pre + left_len + 1, in + left_len + 1, n - left_len - 1,
            &((*root)->right));
}

void print_post_order(const bt_node_t *root) {
    if(root != NULL) {
        print_post_order(root->left);
        print_post_order(root->right);
        printf("%c", root->elem);
    }
}

void rebuild_test() {
    char pre[MAX] = { 0 };
    char in[MAX] = { 0 };
    int n;
    bt_node_t *root;
    scanf("%s%s", pre, in);
    n = strlen(pre);
    
    rebuild(pre, in, n, &root);
    print_post_order(root);
}

int main() {
    build_post_test();
    rebuild_test();
    return 0;
}
\end{Codex}

\section{BST Num}
We are given N Nodes ,each having unique values in[1,N] how many different Binary search tree are possible using all of them.

\section{二叉树的遍历} %%%%%%%%%%%%%%%%%%%%%%%%%%%%%%

树的遍历有两类:深度优先遍历和宽度优先遍历。深度优先遍历又可分为两种:先根(次序)遍历和后根(次序)遍历。

树的先根遍历是:先访问树的根结点,然后依次先根遍历根的各棵子树。树的先跟遍历的结果与对应二叉树(孩子兄弟表示法)的先序遍历的结果相同。

树的后根遍历是:先依次后根遍历树根的各棵子树,然后访问根结点。树的后跟遍历的结果与对应二叉树的中序遍历的结果相同。

二叉树的先根遍历有:\textbf{先序遍历}(root->left->right),root->right->left;后根遍历有:\textbf{后序遍历}(left->right->root),right->left->root;二叉树还有个一般的树没有的遍历次序,\textbf{中序遍历}(left->root->right)。


\subsection{Binary Tree Preorder Traversal}
\label{sec:binary-tree-preorder-traversal}


\subsubsection{描述}
Given a binary tree, return the \emph{preorder} traversal of its nodes' values.

For example:
Given binary tree \code{\{1,\#,2,3\}},
\begin{Code}
	1
	\
	2
	/
	3
\end{Code}
return \code{[1,2,3]}.

Note: Recursive solution is trivial, could you do it iteratively?


\subsubsection{分析}
用栈或者Morris遍历。


\subsubsection{栈}
\begin{Code}
	// LeetCode, Binary Tree Preorder Traversal
	// 使用栈,时间复杂度O(n),空间复杂度O(n)
	class Solution {
		public:
		vector<int> preorderTraversal(TreeNode *root) {
			vector<int> result;
			const TreeNode *p;
			stack<const TreeNode *> s;
			
			p = root;
			if (p != nullptr) s.push(p);
			
			while (!s.empty()) {
				p = s.top();
				s.pop();
				result.push_back(p->val);
				
				if (p->right != nullptr) s.push(p->right);
				if (p->left != nullptr) s.push(p->left);
			}
			return result;
		}
	};
\end{Code}


\subsubsection{Morris先序遍历}
\begin{Code}
	// LeetCode, Binary Tree Preorder Traversal
	// Morris先序遍历,时间复杂度O(n),空间复杂度O(1)
	class Solution {
		public:
		vector<int> preorderTraversal(TreeNode *root) {
			vector<int> result;
			TreeNode *cur, *prev;
			
			cur = root;
			while (cur != nullptr) {
				if (cur->left == nullptr) {
					result.push_back(cur->val);
					prev = cur; /* cur刚刚被访问过 */
					cur = cur->right;
				} else {
				/* 查找前驱 */
				TreeNode *node = cur->left;
				while (node->right != nullptr && node->right != cur)
				node = node->right;
				
				if (node->right == nullptr) { /* 还没线索化,则建立线索 */
					result.push_back(cur->val); /* 仅这一行的位置与中序不同 */
					node->right = cur;
					prev = cur; /* cur刚刚被访问过 */
					cur = cur->left;
				} else {    /* 已经线索化,则删除线索  */
				node->right = nullptr;
				/* prev = cur; 不能有这句,cur已经被访问 */
				cur = cur->right;
			}
		}
	}
	return result;
}
};
\end{Code}


\subsubsection{相关题目}
\begindot
\item Binary Tree Inorder Traversal,见 \S \ref{sec:binary-tree-inorder-traversal}
\item Binary Tree Postorder Traversal,见 \S \ref{sec:binary-tree-postorder-traversal}
\item Recover Binary Search Tree,见 \S \ref{sec:recover-binary-search-tree}
\myenddot


\subsection{Binary Tree Inorder Traversal}
\label{sec:binary-tree-inorder-traversal}


\subsubsection{描述}
Given a binary tree, return the \emph{inorder} traversal of its nodes' values.

For example:
Given binary tree \code{\{1,\#,2,3\}},
\begin{Code}
	1
	\
	2
	/
	3
\end{Code}
return \code{[1,3,2]}.

Note: Recursive solution is trivial, could you do it iteratively?


\subsubsection{分析}
用栈或者Morris遍历。


\subsubsection{栈}
\begin{Code}
	// LeetCode, Binary Tree Inorder Traversal
	// 使用栈,时间复杂度O(n),空间复杂度O(n)
	class Solution {
		public:
		vector<int> inorderTraversal(TreeNode *root) {
			vector<int> result;
			const TreeNode *p = root;
			stack<const TreeNode *> s;
			
			while (!s.empty() || p != nullptr) {
				if (p != nullptr) {
					s.push(p);
					p = p->left;
				} else {
				p = s.top();
				s.pop();
				result.push_back(p->val);
				p = p->right;
			}
		}
		return result;
	}
};
\end{Code}


\subsubsection{Morris中序遍历}
\begin{Code}
	// LeetCode, Binary Tree Inorder Traversal
	// Morris中序遍历,时间复杂度O(n),空间复杂度O(1)
	class Solution {
		public:
		vector<int> inorderTraversal(TreeNode *root) {
			vector<int> result;
			TreeNode *cur, *prev;
			
			cur = root;
			while (cur != nullptr) {
				if (cur->left == nullptr) {
					result.push_back(cur->val);
					prev = cur;
					cur = cur->right;
				} else {
				/* 查找前驱 */
				TreeNode *node = cur->left;
				while (node->right != nullptr && node->right != cur)
				node = node->right;
				
				if (node->right == nullptr) { /* 还没线索化,则建立线索 */
					node->right = cur;
					/* prev = cur; 不能有这句,cur还没有被访问 */
					cur = cur->left;
				} else {    /* 已经线索化,则访问节点,并删除线索  */
				result.push_back(cur->val);
				node->right = nullptr;
				prev = cur;
				cur = cur->right;
			}
		}
	}
	return result;
}
};
\end{Code}


\subsubsection{相关题目}
\begindot
\item Binary Tree Preorder Traversal,见 \S \ref{sec:binary-tree-preorder-traversal}
\item Binary Tree Postorder Traversal,见 \S \ref{sec:binary-tree-postorder-traversal}
\item Recover Binary Search Tree,见 \S \ref{sec:recover-binary-search-tree}
\myenddot


\subsection{Binary Tree Postorder Traversal}
\label{sec:binary-tree-postorder-traversal}


\subsubsection{描述}
Given a binary tree, return the \emph{postorder} traversal of its nodes' values.

For example:
Given binary tree \code{\{1,\#,2,3\}},
\begin{Code}
	1
	\
	2
	/
	3
\end{Code}
return \code{[3,2,1]}.

Note: Recursive solution is trivial, could you do it iteratively?


\subsubsection{分析}
用栈或者Morris遍历。


\subsubsection{栈}
\begin{Code}
	// LeetCode, Binary Tree Postorder Traversal
	// 使用栈,时间复杂度O(n),空间复杂度O(n)
	class Solution {
		public:
		vector<int> postorderTraversal(TreeNode *root) {
			vector<int> result;
			/* p,正在访问的结点,q,刚刚访问过的结点*/
			const TreeNode *p, *q;
			stack<const TreeNode *> s;
			
			p = root;
			
			do {
				while (p != nullptr) { /* 往左下走*/
					s.push(p);
					p = p->left;
				}
				q = nullptr;
				while (!s.empty()) {
					p = s.top();
					s.pop();
					/* 右孩子不存在或已被访问,访问之*/
					if (p->right == q) {
						result.push_back(p->val);
						q = p; /* 保存刚访问过的结点*/
					} else {
					/* 当前结点不能访问,需第二次进栈*/
					s.push(p);
					/* 先处理右子树*/
					p = p->right;
					break;
				}
			}
		} while (!s.empty());
		
		return result;
	}
};
\end{Code}


\subsubsection{Morris后序遍历}
\begin{Code}
	// LeetCode, Binary Tree Postorder Traversal
	// Morris后序遍历,时间复杂度O(n),空间复杂度O(1)
	class Solution {
		public:
		vector<int> postorderTraversal(TreeNode *root) {
			vector<int> result;
			TreeNode dummy(-1);
			TreeNode *cur, *prev = nullptr;
			std::function < void(const TreeNode*)> visit = 
			[&result](const TreeNode *node){
				result.push_back(node->val); 
			};
			
			dummy.left = root;
			cur = &dummy;
			while (cur != nullptr) {
				if (cur->left == nullptr) {
					prev = cur; /* 必须要有 */
					cur = cur->right;
				} else {
				TreeNode *node = cur->left;
				while (node->right != nullptr && node->right != cur)
				node = node->right;
				
				if (node->right == nullptr) { /* 还没线索化,则建立线索 */
					node->right = cur;
					prev = cur; /* 必须要有 */
					cur = cur->left;
				} else { /* 已经线索化,则访问节点,并删除线索  */
				visit_reverse(cur->left, prev, visit);
				prev->right = nullptr;
				prev = cur; /* 必须要有 */
				cur = cur->right;
			}
		}
	}
	return result;
}
private:
// 逆转路径
static void reverse(TreeNode *from, TreeNode *to) {
	TreeNode *x = from, *y = from->right, *z;
	if (from == to) return;
	
	while (x != to) {
		z = y->right;
		y->right = x;
		x = y;
		y = z;
	}
}

// 访问逆转后的路径上的所有结点
static void visit_reverse(TreeNode* from, TreeNode *to, 
std::function< void(const TreeNode*) >& visit) {
	TreeNode *p = to;
	reverse(from, to);
	
	while (true) {
		visit(p);
		if (p == from)
		break;
		p = p->right;
	}
	
	reverse(to, from);
}
};
\end{Code}


\subsubsection{相关题目}
\begindot
\item Binary Tree Preorder Traversal,见 \S \ref{sec:binary-tree-preorder-traversal}
\item Binary Tree Inorder Traversal,见 \S \ref{sec:binary-tree-inorder-traversal}
\item Recover Binary Search Tree,见 \S \ref{sec:recover-binary-search-tree}
\myenddot


\subsection{Binary Tree Level Order Traversal}
\label{sec:binary-tree-tevel-order-traversal}


\subsubsection{描述}
Given a binary tree, return the level order traversal of its nodes' values. (ie, from left to right, level by level).

For example:
Given binary tree \code{\{3,9,20,\#,\#,15,7\}},
\begin{Code}
	3
	/ \
	9  20
	/  \
	15   7
\end{Code}
return its level order traversal as:
\begin{Code}
	[
	[3],
	[9,20],
	[15,7]
	]
\end{Code}


\subsubsection{分析}
无


\subsubsection{递归版}
\begin{Code}
	// LeetCode, Binary Tree Level Order Traversal
	// 递归版,时间复杂度O(n),空间复杂度O(n)
	class Solution {
		public:
		vector<vector<int> > levelOrder(TreeNode *root) {
			vector<vector<int>> result;
			traverse(root, 1, result);
			return result;
		}
		
		void traverse(TreeNode *root, size_t level, vector<vector<int>> &result) {
			if (!root) return;
			
			if (level > result.size())
			result.push_back(vector<int>());
			
			result[level-1].push_back(root->val);
			traverse(root->left, level+1, result);
			traverse(root->right, level+1, result);
		}
	};
\end{Code}


\subsubsection{迭代版}
\begin{Code}
	// LeetCode, Binary Tree Level Order Traversal
	// 迭代版,时间复杂度O(n),空间复杂度O(1)
	class Solution {
		public:
		vector<vector<int> > levelOrder(TreeNode *root) {
			vector<vector<int> > result;
			if(root == nullptr) return result;
			
			queue<TreeNode*> current, next;
			vector<int> level; // elments in level level
			
			current.push(root);
			while (!current.empty()) {
				while (!current.empty()) {
					TreeNode* node = current.front();
					current.pop();
					level.push_back(node->val);
					if (node->left != nullptr) next.push(node->left);
					if (node->right != nullptr) next.push(node->right);
				}
				result.push_back(level);
				level.clear();
				swap(next, current);
			}
			return result;
		}
	};
\end{Code}


\subsubsection{相关题目}
\begindot
\item Binary Tree Level Order Traversal II,见 \S \ref{sec:binary-tree-tevel-order-traversal-ii}
\item Binary Tree Zigzag Level Order Traversal,见 \S \ref{sec:binary-tree-zigzag-level-order-traversal}
\myenddot


\subsection{Binary Tree Level Order Traversal II}
\label{sec:binary-tree-tevel-order-traversal-ii}


\subsubsection{描述}
Given a binary tree, return the bottom-up level order traversal of its nodes' values. (ie, from left to right, level by level from leaf to root).

For example:
Given binary tree \code{\{3,9,20,\#,\#,15,7\}},
\begin{Code}
	3
	/ \
	9  20
	/  \
	15   7
\end{Code}
return its bottom-up level order traversal as:
\begin{Code}
	[
	[15,7]
	[9,20],
	[3],
	]
\end{Code}


\subsubsection{分析}
在上一题(见\S \ref{sec:binary-tree-tevel-order-traversal})的基础上,\fn{reverse()}一下即可。


\subsubsection{递归版}
\begin{Code}
	// LeetCode, Binary Tree Level Order Traversal II
	// 递归版,时间复杂度O(n),空间复杂度O(n)
	class Solution {
		public:
		vector<vector<int> > levelOrderBottom(TreeNode *root) {
			vector<vector<int>> result;
			traverse(root, 1, result);
			std::reverse(result.begin(), result.end()); // 比上一题多此一行
			return result;
		}
		
		void traverse(TreeNode *root, size_t level, vector<vector<int>> &result) {
			if (!root) return;
			
			if (level > result.size())
			result.push_back(vector<int>());
			
			result[level-1].push_back(root->val);
			traverse(root->left, level+1, result);
			traverse(root->right, level+1, result);
		}
	};
\end{Code}


\subsubsection{迭代版}
\begin{Code}
	// LeetCode, Binary Tree Level Order Traversal II
	// 迭代版,时间复杂度O(n),空间复杂度O(1)
	class Solution {
		public:
		vector<vector<int> > levelOrderBottom(TreeNode *root) {
			vector<vector<int> > result;
			if(root == nullptr) return result;
			
			queue<TreeNode*> current, next;
			vector<int> level; // elments in level level
			
			current.push(root);
			while (!current.empty()) {
				while (!current.empty()) {
					TreeNode* node = current.front();
					current.pop();
					level.push_back(node->val);
					if (node->left != nullptr) next.push(node->left);
					if (node->right != nullptr) next.push(node->right);
				}
				result.push_back(level);
				level.clear();
				swap(next, current);
			}
			reverse(result.begin(), result.end()); // 比上一题多此一行
			return result;
		}
	};
\end{Code}


\subsubsection{相关题目}
\begindot
\item Binary Tree Level Order Traversal,见 \S \ref{sec:binary-tree-tevel-order-traversal}
\item Binary Tree Zigzag Level Order Traversal,见 \S \ref{sec:binary-tree-zigzag-level-order-traversal}
\myenddot


\subsection{Binary Tree Zigzag Level Order Traversal}
\label{sec:binary-tree-zigzag-level-order-traversal}


\subsubsection{描述}
Given a binary tree, return the zigzag level order traversal of its nodes' values. (ie, from left to right, then right to left for the next level and alternate between).

For example:
Given binary tree \code{{3,9,20,\#,\#,15,7}},
\begin{Code}
	3
	/ \
	9  20
	/  \
	15   7
\end{Code}
return its zigzag level order traversal as:
\begin{Code}
	[
	[3],
	[20,9],
	[15,7]
	]
\end{Code}


\subsubsection{分析}
广度优先遍历,用一个bool记录是从左到右还是从右到左,每一层结束就翻转一下。


\subsubsection{递归版}
\begin{Code}
	// LeetCode, Binary Tree Zigzag Level Order Traversal
	// 递归版,时间复杂度O(n),空间复杂度O(n)
	class Solution {
		public:
		vector<vector<int> > zigzagLevelOrder(TreeNode *root) {
			vector<vector<int>> result;
			traverse(root, 1, result, true);
			return result;
		}
		
		void traverse(TreeNode *root, size_t level, vector<vector<int>> &result,
		bool left_to_right) {
			if (!root) return;
			
			if (level > result.size())
			result.push_back(vector<int>());
			
			if (left_to_right)
			result[level-1].push_back(root->val);
			else
			result[level-1].insert(result[level-1].begin(), root->val);
			
			traverse(root->left, level+1, result, !left_to_right);
			traverse(root->right, level+1, result, !left_to_right);
		}
	};
\end{Code}

\subsubsection{迭代版}
\begin{Code}
	//LeetCode, Binary Tree Zigzag Level Order Traversal
	//广度优先遍历,用一个bool记录是从左到右还是从右到左,每一层结束就翻转一下。
	// 迭代版,时间复杂度O(n),空间复杂度O(n)
	class Solution {
		public:
		vector<vector<int> > zigzagLevelOrder(TreeNode *root) {
			vector<vector<int> > result;
			if (nullptr == root) return result;
			
			queue<TreeNode*> q;
			bool left_to_right = true;  //left to right
			vector<int> level;  // one level's elements
			
			q.push(root);
			q.push(nullptr);  // level separator
			while (!q.empty()) {
				TreeNode *cur = q.front();
				q.pop();
				if (cur) {
					level.push_back(cur->val);
					if (cur->left) q.push(cur->left);
					if (cur->right) q.push(cur->right);
				} else {
				if (left_to_right) {
					result.push_back(level);
				} else {
				reverse(level.begin(), level.end());
				result.push_back(level);
			}
			level.clear();
			left_to_right = !left_to_right;
			
			if (q.size() > 0) q.push(nullptr);
		}
	}
	
	return result;
}
};
\end{Code}


\subsubsection{相关题目}
\begindot
\item Binary Tree Level Order Traversal,见 \S \ref{sec:binary-tree-tevel-order-traversal}
\item Binary Tree Level Order Traversal II,见 \S \ref{sec:binary-tree-tevel-order-traversal-ii}
\myenddot


\subsection{Recover Binary Search Tree}
\label{sec:recover-binary-search-tree}


\subsubsection{描述}
Two elements of a binary search tree (BST) are swapped by mistake.

Recover the tree without changing its structure.

Note: A solution using $O(n)$ space is pretty straight forward. Could you devise a constant space solution?


\subsubsection{分析}
$O(n)$空间的解法是,开一个指针数组,中序遍历,将节点指针依次存放到数组里,然后寻找两处逆向的位置,先从前往后找第一个逆序的位置,然后从后往前找第二个逆序的位置,交换这两个指针的值。

中序遍历一般需要用到栈,空间也是$O(n)$的,如何才能不使用栈?Morris中序遍历。


\subsubsection{代码}

\begin{Code}
	// LeetCode, Recover Binary Search Tree
	// Morris中序遍历,时间复杂度O(n),空间复杂度O(1)
	class Solution {
		public:
		void recoverTree(TreeNode* root) {
			pair<TreeNode*, TreeNode*> broken;
			TreeNode* prev = nullptr;
			TreeNode* cur = root;
			
			while (cur != nullptr) {
				if (cur->left == nullptr) {
					detect(broken, prev, cur);
					prev = cur;
					cur = cur->right;
				} else {
				auto node = cur->left;
				
				while (node->right != nullptr && node->right != cur)
				node = node->right;
				
				if (node->right == nullptr) {
					node->right = cur;
					//prev = cur; 不能有这句!因为cur还没有被访问
					cur = cur->left;
				} else {
				detect(broken, prev, cur);
				node->right = nullptr;
				prev = cur;
				cur = cur->right;
			}
		}
	}
	
	swap(broken.first->val, broken.second->val);
}

void detect(pair<TreeNode*, TreeNode*>& broken, TreeNode* prev,
TreeNode* current) {
	if (prev != nullptr && prev->val > current->val) {
		if (broken.first == nullptr) {
			broken.first = prev;
		} //不能用else,例如 {0,1},会导致最后 swap时second为nullptr,
		//会 Runtime Error
		broken.second = current;
	}
}
};
\end{Code}


\subsubsection{相关题目}
\begindot
\item Binary Tree Inorder Traversal,见 \S \ref{sec:binary-tree-inorder-traversal}
\myenddot


\subsection{Same Tree}
\label{sec:same-tree}


\subsubsection{描述}
Given two binary trees, write a function to check if they are equal or not.

Two binary trees are considered equal if they are structurally identical and the nodes have the same value.


\subsubsection{分析}
无


\subsubsection{递归版}
递归版
\begin{Code}
	// LeetCode, Same Tree
	// 递归版,时间复杂度O(n),空间复杂度O(logn)
	class Solution {
		public:
		bool isSameTree(TreeNode *p, TreeNode *q) {
			if (!p && !q) return true;   // 终止条件
			if (!p || !q) return false;  // 剪枝
			return p->val == q->val      // 三方合并
			&& isSameTree(p->left, q->left)
			&& isSameTree(p->right, q->right);
		}
	};
\end{Code}


\subsubsection{迭代版}
\begin{Code}
	// LeetCode, Same Tree
	// 迭代版,时间复杂度O(n),空间复杂度O(logn)
	class Solution {
		public:
		bool isSameTree(TreeNode *p, TreeNode *q) {
			stack<TreeNode*> s;
			s.push(p);
			s.push(q);
			
			while(!s.empty()) {
				p = s.top(); s.pop();
				q = s.top(); s.pop();
				
				if (!p && !q) continue;
				if (!p || !q) return false;
				if (p->val != q->val) return false;
				
				s.push(p->left);
				s.push(q->left);
				
				s.push(p->right);
				s.push(q->right);
			}
			return true;
		}
	};
\end{Code}


\subsubsection{相关题目}
\begindot
\item Symmetric Tree,见 \S \ref{sec:symmetric-tree}
\myenddot


\subsection{Symmetric Tree}
\label{sec:symmetric-tree}


\subsubsection{描述}
Given two binary trees, write a function to check if they are equal or not.

Two binary trees are considered equal if they are structurally identical and the nodes have the same value.


\subsubsection{分析}
无


\subsubsection{递归版}
\begin{Code}
	// LeetCode, Symmetric Tree
	// 递归版,时间复杂度O(n),空间复杂度O(logn)
	class Solution {
		public:
		bool isSymmetric(TreeNode *root) {
			return root ? isSymmetric(root->left, root->right) : true;
		}
		bool isSymmetric(TreeNode *left, TreeNode *right) {
			if (!left && !right) return true;   // 终止条件
			if (!left || !right) return false;  // 终止条件
			return left->val == right->val      // 三方合并
			&& isSymmetric(left->left, right->right)
			&& isSymmetric(left->right, right->left);
		}
	};
\end{Code}


\subsubsection{迭代版}
\begin{Code}
	// LeetCode, Symmetric Tree
	// 迭代版,时间复杂度O(n),空间复杂度O(logn)
	class Solution {
		public:
		bool isSymmetric (TreeNode* root) {
			if (!root) return true;
			
			stack<TreeNode*> s;
			s.push(root->left);
			s.push(root->right);
			
			while (!s.empty ()) {
				auto p = s.top (); s.pop();
				auto q = s.top (); s.pop();
				
				if (!p && !q) continue;
				if (!p || !q) return false;
				if (p->val != q->val) return false;
				
				s.push(p->left);
				s.push(q->right);
				
				s.push(p->right);
				s.push(q->left);
			}
			
			return true;
		}
	};
\end{Code}


\subsubsection{相关题目}
\begindot
\item Same Tree,见 \S \ref{sec:same-tree}
\myenddot


\subsection{Balanced Binary Tree}
\label{sec:balanced-binary-tree}


\subsubsection{描述}
Given a binary tree, determine if it is height-balanced.

For this problem, a height-balanced binary tree is defined as a binary tree in which the depth of the two subtrees of every node never differ by more than 1.


\subsubsection{分析}
无


\subsubsection{代码}
\begin{Code}
	// LeetCode, Balanced Binary Tree
	// 时间复杂度O(n),空间复杂度O(logn)
	class Solution {
		public:
		bool isBalanced (TreeNode* root) {
			return balancedHeight (root) >= 0;
		}
		
		/**
		* Returns the height of `root` if `root` is a balanced tree,
		* otherwise, returns `-1`.
		*/
		int balancedHeight (TreeNode* root) {
			if (root == nullptr) return 0;  // 终止条件
			
			int left = balancedHeight (root->left);
			int right = balancedHeight (root->right);
			
			if (left < 0 || right < 0 || abs(left - right) > 1) return -1;  // 剪枝
			
			return max(left, right) + 1; // 三方合并
		}
	};
\end{Code}


\subsubsection{相关题目}
\begindot
\item 无
\myenddot


\subsection{Flatten Binary Tree to Linked List}
\label{sec:flatten-binary-tree-to-linked-list}


\subsubsection{描述}
Given a binary tree, flatten it to a linked list in-place.

For example, Given
\begin{Code}
	1
	/ \
	2   5
	/ \   \
	3   4   6
\end{Code}

The flattened tree should look like:
\begin{Code}
	1
	\
	2
	\
	3
	\
	4
	\
	5
	\
	6
\end{Code}

\subsubsection{分析}
无


\subsubsection{递归版1}
\begin{Code}
	// LeetCode, Flatten Binary Tree to Linked List
	// 递归版1,时间复杂度O(n),空间复杂度O(logn)
	class Solution {
		public:
		void flatten(TreeNode *root) {
			if (root == nullptr) return;  // 终止条件
			
			flatten(root->left);
			flatten(root->right);
			
			if (nullptr == root->left) return;
			
			// 三方合并,将左子树所形成的链表插入到root和root->right之间
			TreeNode *p = root->left;
			while(p->right) p = p->right; //寻找左链表最后一个节点
			p->right = root->right;
			root->right = root->left;
			root->left = nullptr;
		}
	};
\end{Code}


\subsubsection{递归版2}
\begin{Code}
	// LeetCode, Flatten Binary Tree to Linked List
	// 递归版2
	// @author 王顺达(http://weibo.com/u/1234984145)
	// 时间复杂度O(n),空间复杂度O(logn)
	class Solution {
		public:
		void flatten(TreeNode *root) {
			flatten(root, NULL);
		}
		private:
		// 把root所代表树变成链表后,tail跟在该链表后面
		TreeNode *flatten(TreeNode *root, TreeNode *tail) {
			if (NULL == root) return tail;
			
			root->right = flatten(root->left, flatten(root->right, tail));
			root->left = NULL;
			return root;
		}
	};
\end{Code}


\subsubsection{迭代版}
\begin{Code}
	// LeetCode, Flatten Binary Tree to Linked List
	// 迭代版,时间复杂度O(n),空间复杂度O(logn)
	class Solution {
		public:
		void flatten(TreeNode* root) {
			if (root == nullptr) return;
			
			stack<TreeNode*> s;
			s.push(root);
			
			while (!s.empty()) {
				auto p = s.top();
				s.pop();
				
				if (p->right)
				s.push(p->right);
				if (p->left)
				s.push(p->left);
				
				p->left = nullptr;
				if (!s.empty())
				p->right = s.top();
			}
		}
	};
\end{Code}


\subsubsection{相关题目}
\begindot
\item 无
\myenddot


\subsection{Populating Next Right Pointers in Each Node II} %%%%%%%%%%%%%%%%%%%%%%%%%%%%%%
\label{sec:populating-next-right-pointers-in-each-node-ii}


\subsubsection{描述}
Follow up for problem "Populating Next Right Pointers in Each Node".

What if the given tree could be any binary tree? Would your previous solution still work?

Note: You may only use constant extra space.

For example,
Given the following binary tree,
\begin{Code}
	1
	/  \
	2    3
	/ \    \
	4   5    7
\end{Code}

After calling your function, the tree should look like:
\begin{Code}
	1 -> NULL
	/  \
	2 -> 3 -> NULL
	/ \    \
	4-> 5 -> 7 -> NULL
\end{Code}


\subsubsection{分析}
要处理一个节点,可能需要最右边的兄弟节点,首先想到用广搜。但广搜不是常数空间的,本题要求常数空间。

注意,这题的代码原封不动,也可以解决 Populating Next Right Pointers in Each Node I.


\subsubsection{递归版}
\begin{Code}
	// LeetCode, Populating Next Right Pointers in Each Node II
	// 时间复杂度O(n),空间复杂度O(1)
	class Solution {
		public:
		void connect(TreeLinkNode *root) {
			if (root == nullptr) return;
			
			TreeLinkNode dummy(-1);
			for (TreeLinkNode *curr = root, *prev = &dummy; 
			curr; curr = curr->next) {
				if (curr->left != nullptr){
					prev->next = curr->left;
					prev = prev->next;
				}
				if (curr->right != nullptr){
					prev->next = curr->right;
					prev = prev->next;
				}
			}
			connect(dummy.next);
		}
	};
\end{Code}


\subsubsection{迭代版}
\begin{Code}
	// LeetCode, Populating Next Right Pointers in Each Node II
	// 时间复杂度O(n),空间复杂度O(1)
	class Solution {
		public:
		void connect(TreeLinkNode *root) {
			while (root) {
				TreeLinkNode * next = nullptr; // the first node of next level
				TreeLinkNode * prev = nullptr; // previous node on the same level
				for (; root; root = root->next) {
					if (!next) next = root->left ? root->left : root->right;
					
					if (root->left) {
						if (prev) prev->next = root->left;
						prev = root->left;
					}
					if (root->right) {
						if (prev) prev->next = root->right;
						prev = root->right;
					}
				}
				root = next; // turn to next level
			}
		}
	};
\end{Code}


\subsubsection{相关题目}
\begindot
\item Populating Next Right Pointers in Each Node,见 \S \ref{sec:populating-next-right-pointers-in-each-node}
\myenddot


\section{二叉树的构建} %%%%%%%%%%%%%%%%%%%%%%%%%%%%%%


\subsection{Construct Binary Tree from Preorder and Inorder Traversal}
\label{sec:construct-binary-tree-from-preorder-and-inorder-traversal}


\subsubsection{描述}
Given preorder and inorder traversal of a tree, construct the binary tree.

Note:
You may assume that duplicates do not exist in the tree.


\subsubsection{分析}
无


\subsubsection{代码}
\begin{Code}
	// LeetCode, Construct Binary Tree from Preorder and Inorder Traversal
	// 递归,时间复杂度O(n),空间复杂度O(\logn)
	class Solution {
		public:
		TreeNode* buildTree(vector<int>& preorder, vector<int>& inorder) {
			return buildTree(begin(preorder), end(preorder),
			begin(inorder), end(inorder));
		}
		
		template<typename InputIterator>
		TreeNode* buildTree(InputIterator pre_first, InputIterator pre_last,
		InputIterator in_first, InputIterator in_last) {
			if (pre_first == pre_last) return nullptr;
			if (in_first == in_last) return nullptr;
			
			auto root = new TreeNode(*pre_first);
			
			auto inRootPos = find(in_first, in_last, *pre_first);
			auto leftSize = distance(in_first, inRootPos);
			
			root->left = buildTree(next(pre_first), next(pre_first,
			leftSize + 1), in_first, next(in_first, leftSize));
			root->right = buildTree(next(pre_first, leftSize + 1), pre_last,
			next(inRootPos), in_last);
			
			return root;
		}
	};
\end{Code}


\subsubsection{相关题目}
\begindot
\item Construct Binary Tree from Inorder and Postorder Traversal,见 \S \ref{sec:construct-binary-tree-from-inorder-and-postorder-traversal}
\myenddot


\subsection{Construct Binary Tree from Inorder and Postorder Traversal}
\label{sec:construct-binary-tree-from-inorder-and-postorder-traversal}


\subsubsection{描述}
Given inorder and postorder traversal of a tree, construct the binary tree.

Note:
You may assume that duplicates do not exist in the tree.


\subsubsection{分析}
无


\subsubsection{代码}
\begin{Code}
	// LeetCode, Construct Binary Tree from Inorder and Postorder Traversal
	// 递归,时间复杂度O(n),空间复杂度O(\logn)
	class Solution {
		public:
		TreeNode* buildTree(vector<int>& inorder, vector<int>& postorder) {
			return buildTree(begin(inorder), end(inorder),
			begin(postorder), end(postorder));
		}
		
		template<typename BidiIt>
		TreeNode* buildTree(BidiIt in_first, BidiIt in_last,
		BidiIt post_first, BidiIt post_last) {
			if (in_first ==in_last) return nullptr;
			if (post_first == post_last) return nullptr;
			
			const auto val = *prev(post_last);
			TreeNode* root = new TreeNode(val);
			
			auto in_root_pos = find(in_first, in_last, val);
			auto left_size = distance(in_first, in_root_pos);
			auto post_left_last = next(post_first, left_size);
			
			root->left = buildTree(in_first, in_root_pos, post_first, post_left_last);
			root->right = buildTree(next(in_root_pos), in_last, post_left_last,
			prev(post_last));
			
			return root;
		}
	};
\end{Code}


\subsubsection{相关题目}
\begindot
\item Construct Binary Tree from Preorder and Inorder Traversal,见 \S \ref{sec:construct-binary-tree-from-preorder-and-inorder-traversal}
\myenddot


\section{二叉查找树} %%%%%%%%%%%%%%%%%%%%%%%%%%%%%%


\subsection{Unique Binary Search Trees}
\label{sec:unique-binary-search-trees}


\subsubsection{描述}
Given $n$, how many structurally unique BST's (binary search trees) that store values $1...n$?

For example,
Given $n = 3$, there are a total of 5 unique BST's.
\begin{Code}
	1         3     3      2      1
	\       /     /      / \      \
	3     2     1      1   3      2
	/     /       \                 \
	2     1         2                 3
\end{Code}

\subsubsection{分析}
如果把上例的顺序改一下,就可以看出规律了。
\begin{Code}
	1       1           2          3       3
	\       \         / \        /       / 
	3       2       1   3      2       1
	/         \                /         \
	2            3              1           2
\end{Code}

比如,以1为根的树的个数,等于左子树的个数乘以右子树的个数,左子树是0个元素的树,右子树是2个元素的树。以2为根的树的个数,等于左子树的个数乘以右子树的个数,左子树是1个元素的树,右子树也是1个元素的树。依此类推。

当数组为 $1,2,3,...,n$时,基于以下原则的构建的BST树具有唯一性:
\textbf{以i为根节点的树,其左子树由[1, i-1]构成, 其右子树由[i+1, n]构成。}

定义$f(i)$为以$[1,i]$能产生的Unique Binary Search Tree的数目,则

如果数组为空,毫无疑问,只有一种BST,即空树,$f(0)=1$。

如果数组仅有一个元素{1},只有一种BST,单个节点,$f(1)=1$。

如果数组有两个元素{1,2}, 那么有如下两种可能
\begin{Code}
	1             2
	\          /
	2      1
\end{Code}

\begin{eqnarray}
	f(2) &=& f(0) * f(1)   \text{ ,1为根的情况} \nonumber \\
	&+& f(1) * f(0)   \text{ ,2为根的情况} \nonumber
\end{eqnarray}

再看一看3个元素的数组,可以发现BST的取值方式如下:
\begin{eqnarray}
	f(3) &=& f(0) * f(2)   \text{ ,1为根的情况} \nonumber \\
	&+& f(1) * f(1)   \text{ ,2为根的情况} \nonumber \\
	&+& f(2) * f(0)   \text{ ,3为根的情况} \nonumber
\end{eqnarray}

所以,由此观察,可以得出$f$的递推公式为
$$
f(i) = \sum_{k=1}^{i} f(k-1) \times f(i-k)
$$
至此,问题划归为一维动态规划。


\subsubsection{代码}

\begin{Code}
	// LeetCode, Unique Binary Search Trees
	// 时间复杂度O(n^2),空间复杂度O(n)
	class Solution {
		public:
		int numTrees(int n) {
			vector<int> f(n + 1, 0);
			
			f[0] = 1;
			f[1] = 1;
			for (int i = 2; i <= n; ++i) {
				for (int k = 1; k <= i; ++k)
				f[i] += f[k-1] * f[i - k];
			}
			
			return f[n];
		}
	};
\end{Code}


\subsubsection{相关题目}
\begindot
\item Unique Binary Search Trees II,见 \S \ref{sec:unique-binary-search-trees-ii}
\myenddot


\subsection{Unique Binary Search Trees II}
\label{sec:unique-binary-search-trees-ii}


\subsubsection{描述}
Given $n$, generate all structurally unique BST's (binary search trees) that store values 1...n.

For example,
Given $n = 3$, your program should return all 5 unique BST's shown below.
\begin{Code}
	1         3     3      2      1
	\       /     /      / \      \
	3     2     1      1   3      2
	/     /       \                 \
	2     1         2                 3
\end{Code}


\subsubsection{分析}
见前面一题。


\subsubsection{代码}

\begin{Code}
	// LeetCode, Unique Binary Search Trees II
	// 时间复杂度TODO,空间复杂度TODO
	class Solution {
		public:
		vector<TreeNode *> generateTrees(int n) {
			if (n == 0) return generate(1, 0);
			return generate(1, n);
		}
		private:
		vector<TreeNode *> generate(int start, int end) {
			vector<TreeNode*> subTree;
			if (start > end) {
				subTree.push_back(nullptr);
				return subTree;
			}
			for (int k = start; k <= end; k++) {
				vector<TreeNode*> leftSubs = generate(start, k - 1);
				vector<TreeNode*> rightSubs = generate(k + 1, end);
				for (auto i : leftSubs) {
					for (auto j : rightSubs) {
						TreeNode *node = new TreeNode(k);
						node->left = i;
						node->right = j;
						subTree.push_back(node);
					}
				}
			}
			return subTree;
		}
	};
\end{Code}


\subsubsection{相关题目}
\begindot
\item Unique Binary Search Trees,见 \S \ref{sec:unique-binary-search-trees}
\myenddot


\subsection{Validate Binary Search Tree}
\label{sec:validate-binary-search-tree}


\subsubsection{描述}
Given a binary tree, determine if it is a valid binary search tree (BST).

Assume a BST is defined as follows:
\begindot
\item The left subtree of a node contains only nodes with keys less than the node's key.
\item The right subtree of a node contains only nodes with keys greater than the node's key.
\item Both the left and right subtrees must also be binary search trees.
\myenddot


\subsubsection{分析}


\subsubsection{代码}

\begin{Code}
	// LeetCode, Validate Binary Search Tree
	// 时间复杂度O(n),空间复杂度O(\logn)
	class Solution {
		public:
		bool isValidBST(TreeNode* root) {
			return isValidBST(root, INT_MIN, INT_MAX);
		}
		
		bool isValidBST(TreeNode* root, int lower, int upper) {
			if (root == nullptr) return true;
			
			return root->val > lower && root->val < upper
			&& isValidBST(root->left, lower, root->val)
			&& isValidBST(root->right, root->val, upper);
		}
	};
\end{Code}


\subsubsection{相关题目}
\begindot
\item Validate Binary Search Tree,见 \S \ref{sec:validate-binary-search-tree}
\myenddot


\subsection{Convert Sorted Array to Binary Search Tree}
\label{sec:convert-sorted-array-to-binary-search-tree}


\subsubsection{描述}
Given an array where elements are sorted in ascending order, convert it to a height balanced BST.


\subsubsection{分析}
二分法。


\subsubsection{代码}

\begin{Code}
	// LeetCode, Convert Sorted Array to Binary Search Tree
	// 分治法,时间复杂度O(n),空间复杂度O(logn)
	class Solution {
		public:
		TreeNode* sortedArrayToBST (vector<int>& num) {
			return sortedArrayToBST(num.begin(), num.end());
		}
		
		template<typename RandomAccessIterator>
		TreeNode* sortedArrayToBST (RandomAccessIterator first,
		RandomAccessIterator last) {
			const auto length = distance(first, last);
			
			if (length <= 0) return nullptr;  // 终止条件
			
			// 三方合并
			auto mid = first + length / 2;
			TreeNode* root = new TreeNode (*mid);
			root->left = sortedArrayToBST(first, mid);
			root->right = sortedArrayToBST(mid + 1, last);
			
			return root;
		}
	};
\end{Code}


\subsubsection{相关题目}
\begindot
\item Convert Sorted List to Binary Search Tree,见 \S \ref{sec:convert-sorted-list-to-binary-search-tree}
\myenddot


\subsection{Convert Sorted List to Binary Search Tree}
\label{sec:convert-sorted-list-to-binary-search-tree}


\subsubsection{描述}
Given a singly linked list where elements are sorted in ascending order, convert it to a height balanced BST.


\subsubsection{分析}
这题与上一题类似,但是单链表不能随机访问,而自顶向下的二分法必须需要RandomAccessIterator,因此前面的方法不适用本题。

存在一种自底向上(bottom-up)的方法,见\myurl{http://leetcode.com/2010/11/convert-sorted-list-to-balanced-binary.html}


\subsubsection{分治法,自顶向下}
分治法,类似于 Convert Sorted Array to Binary Search Tree,自顶向下,复杂度$O(n\log n)$。
\begin{Code}
	// LeetCode, Convert Sorted List to Binary Search Tree
	// 分治法,类似于 Convert Sorted Array to Binary Search Tree,
	// 自顶向下,时间复杂度O(n^2),空间复杂度O(logn)
	class Solution {
		public:
		TreeNode* sortedListToBST (ListNode* head) {
			return sortedListToBST (head, listLength (head));
		}
		
		TreeNode* sortedListToBST (ListNode* head, int len) {
			if (len == 0) return nullptr;
			if (len == 1) return new TreeNode (head->val);
			
			TreeNode* root = new TreeNode (nth_node (head, len / 2 + 1)->val);
			root->left = sortedListToBST (head, len / 2);
			root->right = sortedListToBST (nth_node (head, len / 2 + 2), 
			(len - 1) / 2);
			
			return root;
		}
		
		int listLength (ListNode* node) {
			int n = 0;
			
			while(node) {
				++n;
				node = node->next;
			}
			
			return n;
		}
		
		ListNode* nth_node (ListNode* node, int n) {
			while (--n)
			node = node->next;
			
			return node;
		}
	};
\end{Code}


\subsubsection{自底向上}
\begin{Code}
	// LeetCode, Convert Sorted List to Binary Search Tree
	// bottom-up,时间复杂度O(n),空间复杂度O(logn)
	class Solution {
		public:
		TreeNode *sortedListToBST(ListNode *head) {
			int len = 0;
			ListNode *p = head;
			while (p) {
				len++;
				p = p->next;
			}
			return sortedListToBST(head, 0, len - 1);
		}
		private:
		TreeNode* sortedListToBST(ListNode*& list, int start, int end) {
			if (start > end) return nullptr;
			
			int mid = start + (end - start) / 2;
			TreeNode *leftChild = sortedListToBST(list, start, mid - 1);
			TreeNode *parent = new TreeNode(list->val);
			parent->left = leftChild;
			list = list->next;
			parent->right = sortedListToBST(list, mid + 1, end);
			return parent;
		}
	};
\end{Code}


\subsubsection{相关题目}
\begindot
\item Convert Sorted Array to Binary Search Tree,见 \S \ref{sec:convert-sorted-array-to-binary-search-tree}
\myenddot


\section{二叉树的递归} %%%%%%%%%%%%%%%%%%%%%%%%%%%%%%
二叉树是一个递归的数据结构,因此是一个用来考察递归思维能力的绝佳数据结构。

递归一定是深搜(见 \S \ref{sec:dfs-vs-recursion}节 “深搜与递归的区别”),由于在二叉树上,递归的味道更浓些,因此本节用“二叉树的递归”作为标题,而不是“二叉树的深搜”,尽管本节所有的算法都属于深搜。

二叉树的先序、中序、后序遍历都可以看做是DFS,此外还有其他顺序的深度优先遍历,共有$3!=6$种。其他3种顺序是 \fn{root->r->l,r->root->l, r->l->root}。


\subsection{Minimum Depth of Binary Tree}
\label{sec:minimum-depth-of-binary-tree}


\subsubsection{描述}
Given a binary tree, find its minimum depth.

The minimum depth is the number of nodes along the shortest path from the root node down to the nearest leaf node.


\subsubsection{分析}
无


\subsubsection{递归版}
\begin{Code}
	// LeetCode, Minimum Depth of Binary Tree
	// 递归版,时间复杂度O(n),空间复杂度O(logn)
	class Solution {
		public:
		int minDepth(const TreeNode *root) {
			return minDepth(root, false);
		}
		private:
		static int minDepth(const TreeNode *root, bool hasbrother) {
			if (!root) return hasbrother ? INT_MAX : 0;
			
			return 1 + min(minDepth(root->left, root->right != NULL),
			minDepth(root->right, root->left != NULL));
		}
	};
\end{Code}


\subsubsection{迭代版}
\begin{Code}
	// LeetCode, Minimum Depth of Binary Tree
	// 迭代版,时间复杂度O(n),空间复杂度O(logn)
	class Solution {
		public:
		int minDepth(TreeNode* root) {
			if (root == nullptr)
			return 0;
			
			int result = INT_MAX;
			
			stack<pair<TreeNode*, int>> s;
			s.push(make_pair(root, 1));
			
			while (!s.empty()) {
				auto node = s.top().first;
				auto depth = s.top().second;
				s.pop();
				
				if (node->left == nullptr && node->right == nullptr)
				result = min(result, depth);
				
				if (node->left && result > depth) // 深度控制,剪枝
				s.push(make_pair(node->left, depth + 1));
				
				if (node->right && result > depth) // 深度控制,剪枝
				s.push(make_pair(node->right, depth + 1));
			}
			
			return result;
		}
	};
\end{Code}

\subsubsection{相关题目}
\begindot
\item Maximum Depth of Binary Tree,见 \S \ref{sec:maximum-depth-of-binary-tree}
\myenddot


\subsection{Maximum Depth of Binary Tree}
\label{sec:maximum-depth-of-binary-tree}


\subsubsection{描述}
Given a binary tree, find its maximum depth.

The maximum depth is the number of nodes along the longest path from the root node down to the farthest leaf node.


\subsubsection{分析}
无

\subsubsection{代码}

\begin{Code}
	// LeetCode, Maximum Depth of Binary Tree
	// 时间复杂度O(n),空间复杂度O(logn)
	class Solution {
		public:
		int maxDepth(TreeNode *root) {
			if (root == nullptr) return 0;
			
			return max(maxDepth(root->left), maxDepth(root->right)) + 1;
		}
	};
\end{Code}


\subsubsection{相关题目}
\begindot
\item Minimum Depth of Binary Tree,见 \S \ref{sec:minimum-depth-of-binary-tree}
\myenddot


\subsection{Path Sum}
\label{sec:path-sum}


\subsubsection{描述}
Given a binary tree and a sum, determine if the tree has a root-to-leaf path such that adding up all the values along the path equals the given sum.

For example:
Given the below binary tree and \code{sum = 22},
\begin{Code}
	5
	/ \
	4   8
	/   / \
	11  13  4
	/  \      \
	7    2      1
\end{Code}
return true, as there exist a root-to-leaf path \code{5->4->11->2} which sum is 22.


\subsubsection{分析}
题目只要求返回\fn{true}或者\fn{false},因此不需要记录路径。

由于只需要求出一个结果,因此,当左、右任意一棵子树求到了满意结果,都可以及时return。

由于题目没有说节点的数据一定是正整数,必须要走到叶子节点才能判断,因此中途没法剪枝,只能进行朴素深搜。

\subsubsection{代码}

\begin{Code}
	// LeetCode, Path Sum
	// 时间复杂度O(n),空间复杂度O(logn)
	class Solution {
		public:
		bool hasPathSum(TreeNode *root, int sum) {
			if (root == nullptr) return false;
			
			if (root->left == nullptr && root->right == nullptr) // leaf
			return sum == root->val;
			
			return hasPathSum(root->left, sum - root->val)
			|| hasPathSum(root->right, sum - root->val);
		}
	};
\end{Code}


\subsubsection{相关题目}
\begindot
\item Path Sum II,见 \S \ref{sec:path-sum-ii}
\myenddot


\subsection{Path Sum II}
\label{sec:path-sum-ii}


\subsubsection{描述}
Given a binary tree and a sum, find all root-to-leaf paths where each path's sum equals the given sum.

For example:
Given the below binary tree and \code{sum = 22},
\begin{Code}
	5
	/ \
	4   8
	/   / \
	11  13  4
	/  \    / \
	7    2  5   1
\end{Code}
return
\begin{Code}
	[
	[5,4,11,2],
	[5,8,4,5]
	]
\end{Code}


\subsubsection{分析}
跟上一题相比,本题是求路径本身。且要求出所有结果,左子树求到了满意结果,不能return,要接着求右子树。

\subsubsection{代码}

\begin{Code}
	// LeetCode, Path Sum II
	// 时间复杂度O(n),空间复杂度O(logn)
	class Solution {
		public:
		vector<vector<int> > pathSum(TreeNode *root, int sum) {
			vector<vector<int> > result;
			vector<int> cur; // 中间结果
			pathSum(root, sum, cur, result);
			return result;
		}
		private:
		void pathSum(TreeNode *root, int gap, vector<int> &cur,
		vector<vector<int> > &result) {
			if (root == nullptr) return;
			
			cur.push_back(root->val);
			
			if (root->left == nullptr && root->right == nullptr) { // leaf
				if (gap == root->val)
				result.push_back(cur);
			}
			pathSum(root->left, gap - root->val, cur, result);
			pathSum(root->right, gap - root->val, cur, result);
			
			cur.pop_back();
		}
	};
\end{Code}


\subsubsection{相关题目}
\begindot
\item Path Sum,见 \S \ref{sec:path-sum}
\myenddot


\subsection{Binary Tree Maximum Path Sum}
\label{sec:binary-tree-maximum-path-sum}


\subsubsection{描述}
Given a binary tree, find the maximum path sum.

The path may start and end at any node in the tree.
For example:
Given the below binary tree,
\begin{Code}
	1
	/ \
	2   3
\end{Code}
Return $6$.


\subsubsection{分析}
这题很难,路径可以从任意节点开始,到任意节点结束。

可以利用“最大连续子序列和”问题的思路,见第\S \ref{sec:maximum-subarray}节。如果说Array只有一个方向的话,那么Binary Tree其实只是左、右两个方向而已,我们需要比较两个方向上的值。

不过,Array可以从头到尾遍历,那么Binary Tree怎么办呢,我们可以采用Binary Tree最常用的dfs来进行遍历。先算出左右子树的结果L和R,如果L大于0,那么对后续结果是有利的,我们加上L,如果R大于0,对后续结果也是有利的,继续加上R。

\subsubsection{代码}

\begin{Code}
	// LeetCode, Binary Tree Maximum Path Sum
	// 时间复杂度O(n),空间复杂度O(logn)
	class Solution {
		public:
		int maxPathSum(TreeNode *root) {
			max_sum = INT_MIN;
			dfs(root);
			return max_sum;
		}
		private:
		int max_sum;
		int dfs(const TreeNode *root) {
			if (root == nullptr) return 0;
			int l = dfs(root->left);
			int r = dfs(root->right);
			int sum = root->val;
			if (l > 0) sum += l;
			if (r > 0) sum += r;
			max_sum = max(max_sum, sum);
			return max(r, l) > 0 ? max(r, l) + root->val : root->val;
		}
	};
\end{Code}

注意,最后return的时候,只返回一个方向上的值,为什么?这是因为在递归中,只能向父节点返回,不可能存在L->root->R的路径,只可能是L->root或R->root。


\subsubsection{相关题目}
\begindot
\item Maximum Subarray,见 \S \ref{sec:maximum-subarray}
\myenddot


\subsection{Populating Next Right Pointers in Each Node} %%%%%%%%%%%%%%%%%%%%%%%%%%%%%%
\label{sec:populating-next-right-pointers-in-each-node}


\subsubsection{描述}
Given a binary tree
\begin{Code}
	struct TreeLinkNode {
		int val;
		TreeLinkNode *left, *right, *next;
		TreeLinkNode(int x) : val(x), left(NULL), right(NULL), next(NULL) {}
	};
\end{Code}

Populate each next pointer to point to its next right node. If there is no next right node, the next pointer should be set to \fn{NULL}.

Initially, all next pointers are set to \fn{NULL}.

Note:
\begindot
\item You may only use constant extra space.
\item You may assume that it is a perfect binary tree (ie, all leaves are at the same level, and every parent has two children).
\myenddot

For example,
Given the following perfect binary tree,
\begin{Code}
	1
	/  \
	2    3
	/ \  / \
	4  5  6  7
\end{Code}

After calling your function, the tree should look like:
\begin{Code}
	1 -> NULL
	/  \
	2 -> 3 -> NULL
	/ \  / \
	4->5->6->7 -> NULL
\end{Code}


\subsubsection{分析}
无

\subsubsection{代码}

\begin{Code}
	// LeetCode, Populating Next Right Pointers in Each Node
	// 时间复杂度O(n),空间复杂度O(logn)
	class Solution {
		public:
		void connect(TreeLinkNode *root) {
			connect(root, NULL);
		}
		private:
		void connect(TreeLinkNode *root, TreeLinkNode *sibling) {
			if (root == nullptr)
			return;
			else
			root->next = sibling;
			
			connect(root->left, root->right);
			if (sibling)
			connect(root->right, sibling->left);
			else
			connect(root->right, nullptr);
		}
	};
\end{Code}


\subsubsection{相关题目}
\begindot
\item Populating Next Right Pointers in Each Node II,见 \S \ref{sec:populating-next-right-pointers-in-each-node-ii}
\myenddot


\subsection{Sum Root to Leaf Numbers} %%%%%%%%%%%%%%%%%%%%%%%%%%%%%%
\label{sec:sum-root-to-leaf-numbers}


\subsubsection{描述}
Given a binary tree containing digits from \fn{0-9} only, each root-to-leaf path could represent a number.

An example is the root-to-leaf path \fn{1->2->3} which represents the number \fn{123}.

Find the total sum of all root-to-leaf numbers.

For example,
\begin{Code}
	1
	/ \
	2   3
\end{Code}

The root-to-leaf path \fn{1->2} represents the number \fn{12}.
The root-to-leaf path \fn{1->3} represents the number \fn{13}.

Return the sum = \fn{12 + 13 = 25}.


\subsubsection{分析}
无

\subsubsection{代码}

\begin{Code}
	// LeetCode, Decode Ways
	// 时间复杂度O(n),空间复杂度O(logn)
	class Solution {
		public:
		int sumNumbers(TreeNode *root) {
			return dfs(root, 0);
		}
		private:
		int dfs(TreeNode *root, int sum) {
			if (root == nullptr) return 0;
			if (root->left == nullptr && root->right == nullptr)
			return sum * 10 + root->val;
			
			return dfs(root->left, sum * 10 + root->val) +
			dfs(root->right, sum * 10 + root->val);
		}
	};
\end{Code}


\subsubsection{相关题目}
\begindot
\item 无
\myenddot


\section{堆} %%%%%%%%%%%%%%%%%%%%%%%%%%%%%%

\subsection{原理和实现}
C++可以直接使用\fn{priority_queue}。

\begin{Codex}[label=heap.c]
/** @file heap.c
 * @brief 堆,默认为小根堆,即堆顶为最小.
 * @author soulmachine@gmail.com
 */
#include <stdlib.h>  /* for malloc() */
#include <string.h>  /* for memcpy() */

typedef int heap_elem_t; // 元素的类型

/**
 * @struct
 * @brief 堆的结构体
 */
typedef struct heap_t {
    int     size;   /** 实际元素个数 */
    int     capacity; /** 容量,以元素为单位 */
    heap_elem_t  *elems;   /** 堆的数组 */
    int (*cmp)(const heap_elem_t*, const heap_elem_t*);   /** 元素的比较函数 */
}heap_t;


/** 基本类型(如int, long, float, double)的比较函数 */
int cmp_int(const int *x, const int *y) {
    const int sub = *x - *y;
    if(sub > 0) {
        return 1;
    } else if(sub < 0) {
        return -1;
    } else {
        return 0;
    }
}

/**
 * @brief 创建堆.
 * @param[out] capacity 初始容量
 * @param[in] cmp cmp 比较函数,小于返回-1,等于返回0
 *            大于返回1,反过来则是大根堆
 * @return 成功返回堆对象的指针,失败返回 NULL
 */
heap_t* heap_create(const int capacity,
        int (*cmp)(const heap_elem_t*, const heap_elem_t*)) {
    heap_t *h = (heap_t*)malloc(sizeof(heap_t));
    h->size = 0;
    h->capacity = capacity;
    h->elems = (heap_elem_t*)malloc(capacity * sizeof(heap_elem_t));
    h->cmp = cmp;

    return h;
}

/**
 * @brief 销毁堆.
 * @param[inout] h 堆对象的指针
 * @return 无
 */
void heap_destroy(heap_t *h) {
    free(h->elems);
    free(h);
}


/**
 * @brief 判断堆是否为空.
 * @param[in] h 堆对象的指针
 * @return 是空,返回 1,否则返回 0
 */
int heap_empty(const heap_t *h) {
    return h->size == 0;
}

/**
 * @brief 获取元素个数.
 * @param[in] s 堆对象的指针
 * @return 元素个数
 */
int heap_size(const heap_t *h) {
    return h->size;
}

/*
 * @brief 小根堆的自上向下筛选算法.
 * @param[in] h 堆对象的指针
 * @param[in] start 开始结点
 * @return 无
 */
void heap_sift_down(const heap_t *h, const int start) {
    int i = start;
    int j;
    const heap_elem_t tmp = h->elems[start];

    for(j = 2 * i + 1; j < h->size; j = 2 * j + 1) {
        if(j < (h->size - 1) &&
            // h->elems[j] > h->elems[j + 1]
            h->cmp(&(h->elems[j]), &(h->elems[j + 1])) > 0) {
                j++; /* j 指向两子女中小者*/
        }
        // tmp <= h->data[j]
        if(h->cmp(&tmp, &(h->elems[j])) <= 0) {
            break;
        } else {
            h->elems[i] = h->elems[j];
            i = j;
        }
    }
    h->elems[i] = tmp;
}

/*
 * @brief 小根堆的自下向上筛选算法.
 * @param[in] h 堆对象的指针
 * @param[in] start 开始结点
 * @return 无
 */
void heap_sift_up(const heap_t *h, const int start) {
    int j = start;
    int i= (j - 1) / 2;
    const heap_elem_t tmp = h->elems[start];

    while(j > 0) {
        // h->data[i] <= tmp
        if(h->cmp(&(h->elems[i]), &tmp) <= 0) {
            break;
        } else {
            h->elems[j] = h->elems[i];
            j = i;
            i = (i - 1) / 2;
        }
    }
    h->elems[j] = tmp;
}

/**
 * @brief 添加一个元素.
 * @param[in] h 堆对象的指针
 * @param[in] x 要添加的元素
 * @return 无
 */
void heap_push(heap_t *h, const heap_elem_t x) {
    if(h->size == h->capacity) { /*已满,重新分配内存*/
        heap_elem_t* tmp =
            (heap_elem_t*)realloc(h->elems, h->capacity * 2 * sizeof(heap_elem_t));
        h->elems = tmp;
        h->capacity *= 2;
    }

    h->elems[h->size] = x;
    h->size++;

    heap_sift_up(h, h->size - 1);
}

/**
 * @brief 弹出堆顶元素.
 * @param[in] h 堆对象的指针
 * @return 无
 */
void heap_pop(heap_t *h) {
    h->elems[0] = h->elems[h->size - 1];
    h->size --;
    heap_sift_down(h, 0);
}

/**
 * @brief 获取堆顶元素.
 * @param[in] h 堆对象的指针
 * @return 堆顶元素
 */
heap_elem_t heap_top(const heap_t *h) {
    return h->elems[0];
}
\end{Codex}


\subsection{最小的N个和} %%%%%%%%%%%%%%%%%%%%%%%%%%%%%%
\subsubsection{描述}
有两个长度为$N$的序列 A 和 B,在 A 和 B 中各任取一个数可以得到 $N^2$ 个和,求这$N^2$ 个和中最小的$N$个。

\subsubsection{输入}
第一行输入一个正整数$N$;第二行N个整数$A_i$ 且$A_i \leq 10^9$;第三行$N$个整数$B_i$,且$Bi \leq 10^9$。

\subsubsection{输出}
输出仅一行,包含$N$个整数,从小到大输出这$N$个最小的和,相邻数字之间用空格隔开。

\subsubsection{样例输入}
\begin{Code}
5
1 3 2 4 5 
6 3 4 1 7
\end{Code}

\subsubsection{样例输出}
\begin{Code}
2 3 4 4 5
\end{Code}

\subsubsection{分析}
由于数据太大,有$N^2$个和,不能通过先求和再排序的方式来求解,这个时候就要用到堆了。

首先将A,B两数组排序,我们可以建立这样一个有序表:
\begin{eqnarray}
A_1+B_1<A_1+B_2<A_1+B_3< &...& <A_1+B_N \nonumber \\
A_2+B_1<A_2+B_2<A_2+B_3< &...& <A_2+B_N \nonumber \\
& ...  \nonumber \\
A_N+B_1<A_N+B_2<A_N+B_3< &...& <A_N+B_N \nonumber
\end{eqnarray}

首先将\fn{A[i] + B[0]}压入堆中,设每次出堆的元素为\fn{sum=A[a]+B[b]},则将\fn{A[a]+B[b+1]}入堆,这样可以保证前$N$个出堆的元素为最小的前$N$项。在实现的时候,可以不用保存B数组的下标,通过\fn{sum-B[b]+B[b+1]}来替换\fn{A[a]+B[b+1]}来节省空间。

\subsubsection{代码}
\begin{Codex}[label=sequence.cpp]
/* wikioi 1245 最小的N个和,http://www.wikioi.com/problem/1245/  */
#include <cstdio>
#include <queue>
#include <algorithm>

const int MAXN = 100000;

int N;
int a[MAXN], b[MAXN];

typedef struct node_t {
    int sum;
    int b; /* sum=a[i]+b[b] */
    bool operator>(const node_t &other) const {
        return sum > other.sum;
    }
} node_t;


void k_merge() {
    sort(a, a+N);
    sort(b, b+N);
    priority_queue<node_t, vector<node_t>,
                                greater<node_t> > q;

    for (int i = 0; i < N; i++) {
        node_t tmp;
        tmp.sum = a[i]+b[0];
        tmp.b = 0;
        q.push(tmp);
    }

    for (int i = 0; i < N; i++) {
        node_t tmp = q.top(); q.pop();
        printf("%d ", tmp.sum);
        tmp.sum = tmp.sum - b[tmp.b] + b[tmp.b + 1];
        tmp.b++;
        q.push(tmp);
    }

    return;
}

int main() {
    scanf("%d", &N);
    for (int i = 0; i < N; i++) {
        scanf("%d", &a[i]);
    }
    for (int i = 0; i < N; i++) {
        scanf("%d", &b[i]);
    }

    k_merge();
    return 0;
}
\end{Codex}

\subsubsection{相关的题目}
与本题相同的题目:
\begindot
\item wikioi 1245 最小的N个和, \myurl{http://www.wikioi.com/problem/1245/}
\myenddot

与本题相似的题目:
\begindot
\item  POJ 2442 Sequence, \myurl{http://poj.org/problem?id=2442}
\myenddot


\section{并查集} %%%%%%%%%%%%%%%%%%%%%%%%%%%%%%

\subsection{原理和实现}
通常用树双亲表示作为并查集的存储结构。每个集合以一棵树表示,数组元素的下标代表元素名,根结点的双亲指针为一个负数,表示集合的元素的个数。如图~\ref{fig:ufs1}、图~\ref{fig:ufs2}和图~\ref{fig:ufs3}所示。

\begin{center}
\includegraphics[width=280pt]{ufs1.png}\\
\figcaption{并查集的初始化}\label{fig:ufs1}
\end{center}

\begin{center}
\includegraphics[width=280pt]{ufs2.png}\\
\figcaption{用树表示并查集}\label{fig:ufs2}
\end{center}

\begin{center}
\includegraphics[width=380pt]{ufs3.png}\\
\figcaption{两个集合的并}\label{fig:ufs3}
\end{center}

并查集的C语言实现如下。

\begin{Codex}[label=ufs.c]
#include <stdlib.h>

/** 并查集. */
typedef struct ufs_t {
    int *p;     /** 树的双亲表示法 */
    int size;   /** 大小. */
} ufs_t;

/**
 * @brief 创建并查集.
 * @param[in] n 数组的容量
 * @return 并查集
 */
ufs_t* ufs_create(int n) {
    ufs_t *ufs = (ufs_t*)malloc(sizeof(ufs_t));
    int i;
    ufs->p = (int*)malloc(n * sizeof(int));
    for(i = 0; i < n; i++)
        ufs->p[i] = -1;
    return ufs;
}

/**
 * @brief 销毁并查集.
 * @param[in] ufs 并查集
 * @return 无
 */
void ufs_destroy(ufs_t *ufs) {
    free(ufs->p);
    free(ufs);
}

/**
 * @brief Find操作,带路径压缩,递归版.
 * @param[in] s 并查集
 * @param[in] x 要查找的元素
 * @return 包含元素x的树的根
 */
int ufs_find(ufs_t *ufs, int x) {
    if (ufs->p[x] < 0) return x; // 终止条件

    return ufs->p[x] = ufs_find(ufs, ufs->p[x]); /* 回溯时的压缩路径 */
}

/** Find操作,朴素版, deprecated. */
static int ufs_find_naive(ufs_t *ufs, int x) {
    while (ufs->p[x] >= 0) {
        x = ufs->p[x];
    }
    return x;
}

/** Find操作,带路径压缩,迭代版. */
static int ufs_find_iterative(ufs_t *ufs, int x) {
    int oldx = x; /* 记录原始x */
    while (ufs->p[x] >= 0) {
        x = ufs->p[x];
    }
    while (oldx != x) {
        int temp = ufs->p[oldx];
        ufs->p[oldx] = x;
        oldx = temp;
    }
    return x;
}

/**
 * @brief Union操作,将y并入到x所在的集合.
 * @param[in] s 并查集
 * @param[in] x 一个元素
 * @param[in] y 另一个元素
 * @return 如果二者已经在同一集合,并失败,返回-1,否则返回0
 */
int ufs_union(ufs_t *ufs, int x, int y) {
    const int rx = ufs_find(ufs, x);
    const int ry = ufs_find(ufs, y);
    if(rx == ry) return -1;

    ufs->p[rx] += ufs->p[ry];
    ufs->p[ry] = rx;
    return 0;
}

/**
 * @brief 获取元素所在的集合的大小
 * @param[in] ufs 并查集
 * @param[in] x 元素
 * @return 元素所在的集合的大小
 */
int ufs_set_size(ufs_t *ufs, int x) {
    const int rx = ufs_find(ufs, x);
    return -ufs->p[rx];
}
\end{Codex}


\subsection{病毒感染者} %%%%%%%%%%%%%%%%%%%%%%%%%%%%%%
\subsubsection{描述}
一个学校有$n$个社团,一个学生能同时加入不同的社团。由于社团内的同学们交往频繁,如果一个学生感染了病毒,该社团的所有学生都会感染病毒。现在0号学生感染了病毒,问一共有多少个人感染了病毒。

\subsubsection{输入}
输入包含多组测试用例。每个测试用例,第一行包含两个整数$n$,$m$,$n$表示学生个数,$m$表示社团个数。假设$0 < n \leq 30000, 0 \leq m \leq 500$。每个学生从0到$n-1$编号。接下来是$m$行,每行开头是一个整数k,表示该社团的学生个数,接着是$k$个整数表示该社团的学生编号。最后一个测试用例,$n=0,m=0$,表示输入结束。

\subsubsection{输出}
对每个测试用例,输出感染了病毒的学生数目。

\subsubsection{样例输入}
\begin{Code}
100 4
2 1 2
5 10 13 11 12 14
2 0 1
2 99 2
200 2
1 5
5 1 2 3 4 5
1 0
0 0
\end{Code}

\subsubsection{样例输出}
\begin{Code}
4
1
1
\end{Code}

\subsubsection{分析}
非常简单的并查集题目。

\subsubsection{代码}
\begin{Codex}[label=suspects.c]
/* POJ 1611 The Suspects, http://poj.org/problem?id=1611 */
#include <stdio.h>

#define MAXN 30000

/* 等价于复制粘贴,这里为了节约篇幅,使用include,在OJ上提交时请用复制粘贴 */
#include "ufs.c"  /* 见“树->并查集”这节 */

int main() {
    int n, m, k;
    while (scanf("%d%d", &n, &m) && n > 0) {
        ufs_t *ufs = ufs_create(MAXN);
        while (m--) {
            int x, y; /* 两个学生 */
            int rx, ry; /* x, y 所属的集合的根 */
            scanf("%d", &k);

            k--;
            scanf("%d", &x);
            rx = ufs_find(ufs, x);
            while (k--) {
                scanf("%d", &y);
                ry = ufs_find(ufs, y);
                ufs_union(ufs, rx, ry);  /* 只要是跟x同一个集合的都并进去 */
            }
        }
        /* 最后搜索0属于哪个集合,这个集合有多少人 */
        printf("%d\n", ufs_set_size(ufs, 0));
        ufs_destroy(ufs);
    }
    return 0;
}
\end{Codex}

\subsubsection{相关的题目}
与本题相同的题目:
\begindot
\item POJ 1611 The Suspects, \myurl{http://poj.org/problem?id=1611}
\myenddot

与本题相似的题目:
\begindot
\item  None
\myenddot


\subsection{两个黑帮} %%%%%%%%%%%%%%%%%%%%%%%%%%%%%%
\subsubsection{描述}
Tadu城市有两个黑帮帮派,已知有$N$黑帮分子,从1到$N$编号,每个人至少属于一个帮派。每个帮派至少有一个人。给你$M$条信息,有两类信息:
\begindot
\item D a b,明确告诉你,a和b属于不同的帮派 
\item A a b,问你,a和b是否属于不同的帮派
\myenddot

\subsubsection{输入}
第一行是一个整数$T$,表示有$T$组测试用例。每组测试用例的第一行是两个整数$N$和$M$,接下来是$M$行,每行包含一条消息。

\subsubsection{输出}
对每条消息"A a b",基于当前获得的信息,输出判断。答案是"In the same gang.", "In different gangs." 和 "Not sure yet."中的一个。

\subsubsection{样例输入}
\begin{Code}
1
5 5
A 1 2
D 1 2
A 1 2
D 2 4
A 1 4
\end{Code}

\subsubsection{样例输出}
\begin{Code}
Not sure yet.
In different gangs.
In the same gang.
\end{Code}

\subsubsection{分析}
把不在一个集合的节点直接用并查集合并在一起。这样的话,如果询问的2个节点在同一个并查集里面,那么它们之间的关系是确定的,否则无法确定它们的关系。

现在还有一个问题是,在同一个集合里面的2个节点是敌对关系还是朋友关系?可以给每个节点另外附加个信息,记录其距离集合根节点的距离。如果,询问的2个节点距离其根节点的距离都是奇数或者都是偶数,那么这2个节点是朋友关系,否则是敌对关系。

\subsubsection{代码}
\begin{Codex}[label=two_gangs.c]
/* POJ 1703 Find them, Catch them, http://poj.org/problem?id=1703 */
#include <stdio.h>
#include <stdlib.h>

#define MAXN 1000001

/** 并查集. */
typedef struct ufs_t {
    int *p;     /** 树的双亲表示法 */
    int *dist;  /** 到根节点的距离的奇偶性 */
    int size;   /** 大小. */
} ufs_t;

/**
 * @brief 创建并查集.
 * @param[in] ufs 并查集
 * @param[in] ufs 并查集
 * @param[in] n 数组的容量
 * @return 并查集
 */
ufs_t* ufs_create(int n) {
    int i;
    ufs_t *ufs = (ufs_t*)malloc(sizeof(ufs_t));
    ufs->p = (int*)malloc(n * sizeof(int));
    ufs->dist = (int*)malloc(n * sizeof(int));
    for(i = 0; i < n; i++) {
        ufs->p[i] = -1;
        ufs->dist[i] = 0;
    }
    return ufs;
}

/**
 * @brief 销毁并查集.
 * @param[in] ufs 并查集
 * @return 无
 */
void ufs_destroy(ufs_t *ufs) {
    free(ufs->p);
    free(ufs->dist);
    free(ufs);
}

/**
 * @brief Find操作,带路径压缩,递归版.
 * @param[in] s 并查集
 * @param[in] x 要查找的元素
 * @return 包含元素x的树的根
 */
int ufs_find(ufs_t *ufs, int x) {
    if (ufs->p[x] < 0) return x; // 终止条件

    const int parent = ufs->p[x];
    ufs->p[x] = ufs_find(ufs, ufs->p[x]); /* 回溯时的压缩路径 */
    ufs->dist[x] = (ufs->dist[x] + ufs->dist[parent]) % 2;
    return ufs->p[x];
}

/**
 * @brief Union操作,将root2并入到root1.
 * @param[in] s 并查集
 * @param[in] root1 一棵树的根
 * @param[in] root2 另一棵树的根
 * @return 如果二者已经在同一集合,并失败,返回-1,否则返回0
 */
int ufs_union(ufs_t *ufs, int root1, int root2) {
    if(root1 == root2) return -1;
    ufs->p[root1] += ufs->p[root2];
    ufs->p[root2] = root1;
    return 0;
}

/**
 * @brief 添加一对敌人.
 * @param[inout] s 并查集
 * @param[in] x 一对敌人的一个
 * @param[in] y 一对敌人的另一个
 * @return 无
 */
void ufs_add_opponent(ufs_t *ufs, int x, int y) {
    const int rx = ufs_find(ufs, x);
    const int ry = ufs_find(ufs, y);
    ufs_union(ufs, rx, ry);
    /* ry与y关系 + y与x的关系 + x与rx的关系 = ry与rx的关系 */
    ufs->dist[ry] = (ufs->dist[y] + 1 + ufs->dist[x]) % 2;
}

int main() {
    int T;

    scanf("%d", &T);
    while (T--) {
        ufs_t *ufs = ufs_create(MAXN);
        int n, m;
        char c;
        int x, y, rx, ry;
        scanf("%d%d%*c", &n, &m);

        while (m--) {
            scanf("%c%d%d%*c", &c, &x, &y); //注意输入
            rx = ufs_find(ufs, x);
            ry = ufs_find(ufs, y);

            if (c == 'A') {
                if (rx == ry) { //如果根节点相同,则表示能判断关系
                    if (ufs->dist[x] != ufs->dist[y])
                        printf("In different gangs.\n");
                    else
                        printf("In the same gang.\n");
                } else
                    printf("Not sure yet.\n");
            } else if (c == 'D') {
                ufs_add_opponent(ufs, x, y);
            }
        }
        ufs_destroy(ufs);
    }
    return 0;
}
\end{Codex}

\subsubsection{相关的题目}
与本题相同的题目:
\begindot
\item POJ 1703 Find them, Catch them, \myurl{http://poj.org/problem?id=1703}
\myenddot

与本题相似的题目:
\begindot
\item  None
\myenddot


\subsection{食物链} %%%%%%%%%%%%%%%%%%%%%%%%%%%%%%
\subsubsection{描述}
动物王国中有三类动物A,B,C,这三类动物的食物链构成了有趣的环形。A吃B, B吃C,C吃A。 现有$N$个动物,从1到$N$编号。每个动物都是A,B,C中的一种,但是我们并不知道它到底是哪一种。 

有人用两种说法对这N个动物所构成的食物链关系进行描述:
\begindot
\item 第一种说法是"1 X Y",表示X和Y是同类。 
\item 第二种说法是"2 X Y",表示X吃Y。 
\myenddot

此人对$N$个动物,用上述两种说法,一句接一句地说出$K$句话,这$K$句话有的是真的,有的是假的。当一句话满足下列三条之一时,这句话就是假话,否则就是真话。 
\begindot
\item 当前的话与前面的某些真的话冲突,就是假话; 
\item 当前的话中X或Y比N大,就是假话; 
\item 当前的话表示X吃X,就是假话。 
\myenddot

你的任务是根据给定的$N(1 \leq N \leq 50,000)$和$K$句话($0 \leq K \leq 100,000$),输出假话的总数。 

\subsubsection{输入}
第一行是两个整数$N$和$K$,以一个空格分隔。 

以下$K$行每行是三个正整数D,X,Y,两数之间用一个空格隔开,其中D表示说法的种类。
\begindot
\item 若D=1,则表示X和Y是同类。 
\item 若D=2,则表示X吃Y。
\myenddot

\subsubsection{输出}
只有一个整数,表示假话的数目。

\subsubsection{样例输入}
\begin{Code}
100 7
1 101 1 
2 1 2
2 2 3 
2 3 3 
1 1 3 
2 3 1 
1 5 5
\end{Code}

\subsubsection{样例输出}
\begin{Code}
3
\end{Code}

\subsubsection{分析}


\subsubsection{代码}
\begin{Codex}[label=food_chain.c]
/* POJ 1182 食物链, Catch them, http://poj.org/problem?id=1182 */
#include <stdio.h>
#include <stdlib.h>

/** 并查集. */
typedef struct ufs_t {
    int *p;     /** 树的双亲表示法 */
    int *dist;  /** 表示x与父节点p[x]的关系,0表示x与p[x]是同类,
                    1表示x吃p[x],2表示p[x]吃x */
    int size;   /** 大小. */
} ufs_t;

/**
 * @brief 创建并查集.
 * @param[in] ufs 并查集
 * @param[in] n 数组的容量
 * @return 并查集
 */
ufs_t* ufs_create(int n) {
    int i;
    ufs_t *ufs = (ufs_t*)malloc(sizeof(ufs_t));
    ufs->p = (int*)malloc(n * sizeof(int));
    ufs->dist = (int*)malloc(n * sizeof(int));
    for(i = 0; i < n; i++) {
        ufs->p[i] = -1;
        ufs->dist[i] = 0; // 自己与自己是同类
    }
    return ufs;
}

/**
 * @brief 销毁并查集.
 * @param[in] ufs 并查集
 * @return 无
 */
void ufs_destroy(ufs_t *ufs) {
    free(ufs->p);
    free(ufs->dist);
    free(ufs);
}

/**
 * @brief Find操作,带路径压缩,递归版.
 * @param[in] s 并查集
 * @param[in] x 要查找的元素
 * @return 包含元素x的树的根
 */
int ufs_find(ufs_t *ufs, int x) {
    if (ufs->p[x] < 0) return x; // 终止条件

    const int parent = ufs->p[x];
    ufs->p[x] = ufs_find(ufs, ufs->p[x]); /* 回溯时的压缩路径 */
    /* 更新关系 */
    ufs->dist[x] = (ufs->dist[x] + ufs->dist[parent]) % 3;
    return ufs->p[x];
}

/**
 * @brief Union操作,将root2并入到root1.
 * @param[in] s 并查集
 * @param[in] root1 一棵树的根
 * @param[in] root2 另一棵树的根
 * @return 如果二者已经在同一集合,并失败,返回-1,否则返回0
 */
int ufs_union(ufs_t *ufs, int root1, int root2) {
    if(root1 == root2) return -1;
    ufs->p[root1] += ufs->p[root2];
    ufs->p[root2] = root1;
    return 0;
}

/**
 * @brief 添加一对关系.
 * @param[inout] s 并查集
 * @param[in] x 一个
 * @param[in] y 另一个
 * @param[in] len
 * @return 无
 */
void ufs_add_relation(ufs_t *ufs, int x, int y, int relation) {
    const int rx = ufs_find(ufs, x);
    const int ry = ufs_find(ufs, y);
    ufs_union(ufs, ry, rx); /* 注意顺序! */
    /* rx与x关系 + x与y的关系 + y与ry的关系 = rx与ry的关系 */
    ufs->dist[rx] = (ufs->dist[y] - ufs->dist[x] + 3 + relation) % 3;
}

int main() {
    int n, k;
    int result = 0; /* 假话的数目 */
    ufs_t *ufs;

    scanf("%d%d", &n, &k);
    ufs = ufs_create(n + 1);

    while(k--) {
        int d, x, y;
        scanf("%d%d%d", &d, &x, &y);

        if (x > n || y > n || (d == 2 && x == y)) {
            result++;
        } else {
            const int rx = ufs_find(ufs, x);
            const int ry = ufs_find(ufs, y);

            if (rx == ry) { /* 若在同一个集合则可确定x和y的关系 */
                if((ufs->dist[x] - ufs->dist[y] + 3) % 3 != d - 1)
                    result++;
            } else {
                ufs_add_relation(ufs, x, y, d-1);
            }
        }
    }

    printf("%d\n", result);

    ufs_destroy(ufs);
    return 0;
}
\end{Codex}

\subsubsection{相关的题目}
与本题相同的题目:
\begindot
\item POJ 1182 食物链, \myurl{http://poj.org/problem?id=1182}
\item wikioi 1074 食物链, \myurl{http://www.wikioi.com/problem/1074/}
\myenddot

与本题相似的题目:
\begindot
\item  None
\myenddot


\section{线段树} %%%%%%%%%%%%%%%%%%%%%%%%%%%%%%

\subsection{原理和实现}
\textbf{线段树},也叫区间树(interval tree),它在各个节点保存一条线段(即子数组)。设数列$A$包含$N$个元素,则线段树的根节点表示整个区间$A[1,N]$,左孩子表示区间$A[1, (1+N)/2]$,右孩子表示区间$A[(1+N)/2+1, N]$,不断递归,直到叶子节点,叶子节点只包含一个元素。

线段树有如下特征:
\begindot
\item 线段树是一棵完全二叉树
\item 线段树的深度不超过$\log L$, $L$是区间的长度
\item 线段树把一个长度为L的区间分成不超过$2\log L$条线段
\myenddot

线段树的基本操作有构造线段树、区间查询和区间修改。

线段树通常用于解决和区间统计有关的问题。比如某些数据可以按区间进行划分,按区间动态进行修改,而且还需要按区间多次进行查询,那么使用线段树可以达到较快的查询速度。

用线段树解题,关键是要想清楚每个节点要存哪些信息(当然区间起点和终点,以及左右孩子指针是必须的),以及这些信息如何高效查询,更新。不要一更新就更新到叶子节点,那样更新操作的效率最坏有可能$O(N)$的。

\subsection{Balanced Lineup} %%%%%%%%%%%%%%%%%%%%%%%%%%%%%%
\subsubsection{描述}
给定$N(1 \leq N \leq 50,000)$ 个数, $A_1, A_2, ... , A_N$,求任意区间中最大数和最小数的差。

\subsubsection{输入}
第一行包含两个整数,$N$和$Q$。$Q$表示查询次数。

第2到N+1行,每行包含一个整数$A_i$。

第N+2到N+Q+1行,每行包含两个整数$a$和$b(1 \leq a \leq b \leq N)$,表示区间$A[a,b]$。

\subsubsection{输出}
对每个查询进行回应,输出该区间内最大数和最小数的差

\subsubsection{样例输入}
\begin{Code}
6 3
1
7
3
4
2
5
1 5
4 6
2 2
\end{Code}

\subsubsection{样例输出}
\begin{Code}
6
3
0
\end{Code}

\subsubsection{分析}
本题是“区间求和”,只需要“线段树构造”和“区间查询”两个操作。

\subsubsection{代码}
\begin{Codex}[label=balanced_lineup.c]
#include <stdio.h>
#include <stdlib.h>
#include <string.h>
#include <limits.h>

#define MAXN 50001
#define INF INT_MAX
#define max(a,b) ((a)>(b)?(a):(b))
#define min(a,b) ((a)<(b)?(a):(b))
#define L(a) ((a)<<1)
#define R(a) (((a)<<1)+1)

typedef struct node_t {
    int left, right;  /* 区间  */
    int max, min;  /* 本区间里的最大值和最小值 */
} node_t;

int A[MAXN]; /* 输入数据,0位置未用 */

/* 完全二叉树,结点编号从1开始,层次从1开始.
 * 用一维数组存储完全二叉树,空间约为4N,
 * 参考 http://comzyh.tk/blog/archives/479/
 */
node_t node[MAXN * 4];

int minx, maxx; /* 存放查询的结果 */

void init() {
    memset(node, 0, sizeof(node));
}

/* 以t为根结点,为区间A[l,r]建立线段树 */
void build(int t, int l, int r) {
    node[t].left = l, node[t].right = r;
    if (l == r) {
        node[t].max = node[t].min = A[l];
        return;
    }
    const int mid = (l + r) / 2;
    build(L(t), l, mid);
    build(R(t), mid + 1, r);
    node[t].max = max(node[L(t)].max,node[R(t)].max);
    node[t].min = min(node[L(t)].min,node[R(t)].min);
}

/* 查询根结点为t,区间为A[l,r]的最大值和最小值 */
void query(int t, int l, int r) {
    if (node[t].left == l && node[t].right == r) {
        if (maxx < node[t].max)
            maxx = node[t].max;
        if (minx > node[t].min)
            minx = node[t].min;
        return;
    }
    const int mid = (node[t].left + node[t].right) / 2;
    if (l > mid) {
        query(R(t), l, r);
    } else if (r <= mid) {
        query(L(t), l, r);
    } else {
        query(L(t), l, mid);
        query(R(t), mid + 1, r);
    }
}

int main() {
    int n, q, i;

    scanf("%d%d", &n, &q);
    for (i = 1; i <= n; i++) scanf("%d", &A[i]);

    init();
    /* 建立以tree[1]为根结点,区间为A[1,n]的线段树 */
    build(1, 1, n);

    while (q--) {
        int a, b;
        scanf("%d%d", &a, &b);
        maxx = 0;
        minx = INF;
        query(1, a, b); /* 查询区间A[a,b]的最大值和最小值 */
        printf("%d\n", maxx - minx);
    }
    return 0;
}
\end{Codex}

\subsubsection{相关的题目}
与本题相同的题目:
\begindot
\item POJ 3264 Balanced Lineup, \myurl{http://poj.org/problem?id=3264}
\myenddot

与本题相似的题目:
\begindot
\item  None
\myenddot


\subsection{线段树练习 1} %%%%%%%%%%%%%%%%%%%%%%%%%%%%%%
\subsubsection{描述}
一行$N(1\leq N < 100000)$个方格,开始每个格子里都有一个整数。现在动态地提出一些命令请求,有两种命令,查询和增加:求某一个特定的子区间$[a,b]$中所有元素的和;指定某一个格子$x$,加上一个特定的值A。现在要求你能对每个请求作出正确的回答。

\subsubsection{输入}
输入文件第一行为一个整数$N$,接下来是$n$行每行1个整数,表示格子中原来的整数。接下来是一个正整数$Q$,再接下来有$Q$行,表示$Q$个询问,第一个整数表示命令代号,命令代号1表示增加,后面的两个数$a$和$x$表示给位置$a$上的数值增加$x$,命令代号2表示区间求和,后面两个整数a和b,表示要求[a,b]之间的区间和。

\subsubsection{输出}
共$Q$行,每个整数

\subsubsection{样例输入}
\begin{Code}
6
4 
5 
6 
2 
1 
3
4
1 3 5
2 1 4
1 1 9
2 2 6
\end{Code}

\subsubsection{样例输出}
\begin{Code}
22
22
\end{Code}

\subsubsection{分析}
单点更新+区间求和

\subsubsection{代码}
\begin{Codex}[label=interval_tree1.c]
/* wikioi 1080 线段树练习 , http://www.wikioi.com/problem/1080/ */
#include <stdio.h>
#include <string.h>

#define L(a) ((a)<<1)
#define R(a) (((a)<<1)+1)
#define MAXN 100001

typedef long long int64_t;

typedef struct node_t {
    int left, right;
    int64_t sum;
} node_t;

int A[MAXN]; /* 输入数据,0位置未用 */

/* 完全二叉树,结点编号从1开始,层次从1开始.
 * 用一维数组存储完全二叉树,空间约为4N,
 * 参考 http://comzyh.tk/blog/archives/479/
 */
node_t node[MAXN * 4];

void init() {
    memset(node, 0, sizeof(node));
}

/* 以t为根结点,为区间A[l,r]建立线段树 */
void build(int t, int l, int r) {
    node[t].left = l;
    node[t].right = r;
    if (l == r) {
        node[t].sum = A[l];
        return;
    }
    const int mid = (l + r) / 2;
    build(L(t), l, mid);
    build(R(t), mid + 1, r);
    node[t].sum = node[L(t)].sum + node[R(t)].sum;
}

/* 给区间A[l,r]里的pos位置加delta */
void update(int t, int l, int r, int pos, int64_t delta) {
    if (node[t].left > pos || node[t].right < pos) return;
    if (node[t].left == node[t].right) {
        node[t].sum += delta;
        return;
    }

    const int mid = (node[t].left + node[t].right) / 2;
    if (l > mid) update(R(t), l, r, pos, delta);
    else if (r <= mid) update(L(t), l, r, pos, delta);
    else {
        update(L(t), l, mid, pos, delta);
        update(R(t), mid + 1, r, pos, delta);
    }
    node[t].sum = node[L(t)].sum + node[R(t)].sum;
}

/* 查询根结点为t,区间为A[l,r]的和 */
int64_t query(int t, int l, int r) {
    if (node[t].left == l && node[t].right == r)
        return node[t].sum;
    const int mid = (node[t].left + node[t].right) / 2;
    if (l > mid) return query(R(t), l, r);
    else if (r <= mid) return query(L(t), l, r);
    else return query(L(t), l, mid) + query(R(t), mid + 1, r);
}

int main() {
    int i, n, q;
    scanf("%d", &n);
    for (i = 1; i <= n; i++) scanf("%d", &A[i]);

    init();
    /* 建立以tree[1]为根结点,区间为A[1,n]的线段树 */
    build(1, 1, n);

    scanf("%d", &q);
    while (q--) {
        int cmd;
        scanf("%d", &cmd);
        if (cmd == 2) {
            int a, b;
            scanf("%d%d", &a, &b);
            printf("%lld\n", query(1, a, b)); /* 查询区间A[a,b]的和 */
        } else {
            int a;
            int64_t x;
            scanf("%d%lld", &a, &x);
            if (x != 0) update(1, 1, n, a, x);
        }
    }
    return 0;
}
\end{Codex}

\subsubsection{相关的题目}
与本题相同的题目:
\begindot
\item wikioi 1080 线段树练习 1, \myurl{http://www.wikioi.com/problem/1080/}
\myenddot

与本题相似的题目:
\begindot
\item  wikioi 1081 线段树练习 2, \myurl{http://www.wikioi.com/problem/1081/} 。本题是“区间更新+单点查询”,可以转化为线段树练习1。设原数组为$A[N]$,将其转化为差分数列,然后在数组上维护一棵线段树。“区间更新”操作转化为两个“单点更新”操作:将$A[a]$加上$x$,并将$A[b+1]$减去$x$(也就是加上$-x$)。“单点查询”操作转化为“区间求和”操作:求$A$数组$[1..i]$范围内所有数的和。这样就转化成与线段树练习1完全相同了。标程 \myurl{https://gist.github.com/soulmachine/6449609}
\myenddot


\subsection{A Simple Problem with Integers} %%%%%%%%%%%%%%%%%%%%%%%%%%%%%%
\subsubsection{描述}
You have $N$ integers, $A_1, A_2, ... , A_N$. You need to deal with two kinds of operations. One type of operation is to add some given number to each number in a given interval. The other is to ask for the sum of numbers in a given interval.

\subsubsection{输入}
The first line contains two numbers $N$ and $Q$. $1 \leq N,Q \leq 100000$.

The second line contains $N$ numbers, the initial values of $A_1, A_2, ... , A_N$. $-1000000000 \leq A_i \leq 1000000000$.

Each of the next $Q$ lines represents an operation.
"C a b c" means adding $c$ to each of $A_a, A_{a+1}, ... , A_b$. $-10000 ≤ c ≤ 10000$.
"Q a b" means querying the sum of $A_a, A_{a+1}, ... , A_b$.

\subsubsection{输出}
You need to answer all $Q$ commands in order. One answer in a line.

\subsubsection{样例输入}
\begin{Code}
10 5
1 2 3 4 5 6 7 8 9 10
Q 4 4
Q 1 10
Q 2 4
C 3 6 3
Q 2 4
\end{Code}

\subsubsection{样例输出}
\begin{Code}
4
55
9
15
\end{Code}

\subsubsection{提示}
The sums may exceed the range of 32-bit integers.

\subsubsection{分析}
区间更新+区间求和。

树节点要存哪些信息?只存该区间的和,行不行?只存和,会导致每次加数的时候都要更新到叶子节点,速度太慢。本题节点的结构如下:
\begin{Code}
typedef struct node_t {
    int left, right;
    int64_t sum;  /* 本区间的和实际上是sum+inc*[right-left+1] */
    int64_t inc;  /* 增量c的累加 */
} node_t;
\end{Code}

\subsubsection{代码}
\begin{Codex}[label=poj3468.c]
#include <stdio.h>
#include <string.h>

#define L(a) ((a)<<1)
#define R(a) (((a)<<1)+1)
#define MAXN 100001

typedef long long int64_t;

typedef struct node_t {
    int left, right;
    int64_t sum;  /* 本区间的和实际上是sum+inc*[right-left+1] */
    int64_t inc;  /* 增量c的累加 */
} node_t;

int A[MAXN]; /* 输入数据,0位置未用 */

/* 完全二叉树,结点编号从1开始,层次从1开始.
 * 用一维数组存储完全二叉树,空间约为4N,
 * 参考 http://comzyh.tk/blog/archives/479/
 */
node_t node[MAXN * 4];

void init() {
    memset(node, 0, sizeof(node));
}

/* 以t为根结点,为区间A[l,r]建立线段树 */
void build(int t, int l, int r) {
    node[t].left = l;
    node[t].right = r;
    if (l == r) {
        node[t].sum = A[l];
        return;
    }
    const int mid = (l + r) / 2;
    build(L(t), l, mid);
    build(R(t), mid+1, r);
    node[t].sum = node[L(t)].sum + node[R(t)].sum;
}

/* 给区间A[l,r]里的每个元素都加c */
void update(int t, int l, int r, int64_t c) {
    if (node[t].left == l && node[t].right == r) {
        node[t].inc += c;
        node[t].sum += c * (r - l + 1);
        return;
    }
    if (node[t].inc) {
        node[R(t)].inc += node[t].inc;
        node[L(t)].inc += node[t].inc;
        node[R(t)].sum += node[t].inc * (node[R(t)].right - node[R(t)].left + 1);
        node[L(t)].sum += node[t].inc * (node[L(t)].right - node[L(t)].left + 1);
        node[t].inc = 0;
    }
    const int mid = (node[t].left + node[t].right) / 2;
    if (l > mid)
        update(R(t), l, r, c);
    else if (r <= mid)
        update(L(t), l, r, c);
    else {
        update(L(t), l, mid, c);
        update(R(t), mid + 1, r, c);
    }
    node[t].sum = node[L(t)].sum + node[R(t)].sum;
}

/* 查询根结点为t,区间为A[l,r]的和 */
int64_t query(int t, int l, int r) {
    if (node[t].left == l && node[t].right == r)
        return node[t].sum;
    if (node[t].inc) {
        node[R(t)].inc += node[t].inc;
        node[L(t)].inc += node[t].inc;
        node[R(t)].sum += node[t].inc * (node[R(t)].right - node[R(t)].left + 1);
        node[L(t)].sum += node[t].inc * (node[L(t)].right - node[L(t)].left + 1);
        node[t].inc = 0;
    }
    const int mid = (node[t].left + node[t].right) / 2;
    if (l > mid)
        return query(R(t), l, r);
    else if (r <= mid)
        return query(L(t), l, r);
    else
        return query(L(t), l, mid) + query(R(t), mid + 1, r);
}

int main() {
    int i, n, q;
    char s[5];
    scanf("%d%d", &n, &q);
    for (i = 1; i <= n; i++) scanf("%d", &A[i]);

    init();
    /* 建立以tree[1]为根结点,区间为A[1,n]的线段树 */
    build(1, 1, n);

    while (q--) {
        int a, b;
        int64_t c;
        scanf("%s", s);
        if (s[0] == 'Q') {
            scanf("%d%d", &a, &b);
            printf("%lld\n", query(1, a, b)); /* 查询区间A[a,b]的和 */
        } else {
            scanf("%d%d%lld", &a, &b, &c);
            if (c != 0) update(1, a, b, c);
        }
    }
    return 0;
}
\end{Codex}

\subsubsection{相关的题目}
与本题相同的题目:
\begindot
\item POJ 3468 A Simple Problem with Integers, \myurl{http://poj.org/problem?id=3468}
\myenddot

与本题相似的题目:
\begindot
\item None
\myenddot


\subsection{约瑟夫问题} %%%%%%%%%%%%%%%%%%%%%%%%%%%%%%
\subsubsection{描述}
有编号从1到$N$的$N$个小朋友在玩一种出圈的游戏。开始时$N$个小朋友围成一圈,编号为$i+1$的小朋友站在编号为$i$小朋友左边。编号为1的小朋友站在编号为$N$的小朋友左边。首先编号为1的小朋友开始报数,接着站在左边的小朋友顺序报数,直到数到某个数字$M$时就出圈。直到只剩下1个小朋友,则游戏完毕。

现在给定$N,M$,求$N$个小朋友的出圈顺序。

\subsubsection{输入}
唯一的一行包含两个整数$N,M(1 \leq N,M \leq 30000)$。

\subsubsection{输出}
唯一的一行包含$N$个整数,每两个整数中间用空格隔开,第$i$个整数表示第$i$个出圈的小朋友的编号。

\subsubsection{样例输入}
\begin{Code}
5 3
\end{Code}

\subsubsection{样例输出}
\begin{Code}
3 1 5 2 4
\end{Code}

\subsubsection{分析}
约瑟夫问题的难点在于,每一轮都不能通过简单的运算得出下一轮谁淘汰,因为中间有人已经退出了。因此一般只能模拟,效率很低。

现在考虑,每一轮都令所有剩下的人从左到右重新编号,例如3退出后,场上还剩下1、2、4、5,则给1新编号1,2新编号2,4新编号3,5新编号4。不妨称这个编号为“剩余队列编号”。如下所示,括号内为原始编号:
\begin{Code}
1(1) 2(2) 3(3) 4(4) 5(5) --> 剩余队列编号3淘汰,对应原编号3
1(1) 2(2) 3(4) 4(5) --> 剩余队列编号1淘汰,对应原编号1
1(2) 2(4) 3(5) --> 剩余队列编号3淘汰,对应原编号5
1(2) 2(4) --> 剩余队列编号1淘汰,对应原编号2
1(4) --> 剩余队列编号1滔天,对应原编号4
\end{Code}

一个人在当前剩余队列中编号为$i$,则说明他是从左到右数第$i$个人,这启发我们可以用线段树来解决问题。用线段树维护原编号$[i..j]$内还有多少人没 有被淘汰,这样每次选出被淘汰者后,在当前线段树中查找位置就可以了。

例如我们有5个原编号,当前淘汰者在剩余队列中编号为3,先看左子树,即原编号[1..3]区间内,如果剩下的人不足3个,则说明当前剩余编号为3的 这个人原编号只能是在[4..5]区间内,继续在[4..5]上搜索;如果[1..3]内剩下的人大于等于3个,则说明就在[1..3]内,也继续缩小范围查找,这样即可在$O(\log N)$时间内完成对应。问题得到圆满的解决。

\subsubsection{代码}
\begin{Codex}[label=josephus_problem.c]
/* wikioi 1282 约瑟夫问题, http://www.wikioi.com/problem/1282/ */
#include <stdio.h>
#include <string.h>

#define L(a) ((a)<<1)
#define R(a) (((a)<<1)+1)
#define MAXN 30001

typedef struct node_t {
    int left, right;
    int count; /* 区间内的元素个数 */
} node_t;

/* 完全二叉树,结点编号从1开始,层次从1开始.
 * 用一维数组存储完全二叉树,空间约为4N,
 * 参考 http://comzyh.tk/blog/archives/479/
 */
node_t node[MAXN * 4];

void init() {
    memset(node, 0, sizeof(node));
}

/* 以t为根结点,为区间[l,r]建立线段树 */
void build(int t, int l, int r) {
    node[t].left = l;
    node[t].right = r;
    node[t].count = r - l + 1;
    if (l == r) return;

    const int mid = (r + l) / 2;
    build(L(t), l, mid);
    build(R(t), mid + 1, r);
}

/**
 * @brief 输出i
 * @param[in] t 根节点
 * @param[in] i 剩余队列编号
 * @return 被删除的实际数字
 */
int delete(int t, int i) {
    node[t].count--;
    if (node[t].left == node[t].right) {
        printf("%d ", node[t].left);
        return node[t].left;
    }
    if (node[L(t)].count >= i) return delete(L(t), i);
    else return delete(R(t), i - node[L(t)].count); /* 左子树人数不足,则在右子树查找 */
}

/**
 * @brief 返回 1到i内的活人数
 * @param[in] t 根节点
 * @param[in] i 原始队列的数字
 * @return 1到i内的活人数
 */
int get_count(int t, int i) {
    if (node[t].right <= i) return node[t].count;

    const int mid = (node[t].left + node[t].right) / 2;
    int s = 0;
    if (i > mid) {
        s += node[L(t)].count;
        s += get_count(R(t), i);
    } else
        s += get_count(L(t), i);
    return s;
}

int main() {
    int n, m;
    scanf("%d%d", &n, &m);

    init();
    build(1, 1, n);

    int i;
    int j = 0; /* 剩余队列的虚拟编号 */
    for (i = 1; i <= n; i++) {
        j += m;
        if (j > node[1].count)
            j %= node[1].count;
        if (j == 0) j = node[1].count;
        const int k = delete(1, j);
        j = get_count(1, k);
    }
    return 0;
}
\end{Codex}

\subsubsection{相关的题目}
与本题相同的题目:
\begindot
\item wikioi 1282 约瑟夫问题, \myurl{http://www.wikioi.com/problem/1282/}
\myenddot

与本题相似的题目:
\begindot
\item None
\myenddot


\section{Trie 树} %%%%%%%%%%%%%%%%%%%%%%%%%%%%%%
支持插入一个字符串,查询一个字符串是否存在。


\subsection{原理和实现}

\begin{Codex}[label=trie_tree.c]
#include <stdio.h>
#include <string.h>
#include <stdlib.h>


#define MAXN 10000   /** 输入的编码的最大个数. */
#define CHAR_COUNT  10 /** 字符的种类,也即单个节点的子树的最大个数 */
#define MAX_CODE_LEN 10 /** 编码的最大长度. */
#define MAX_NODE_COUNT  (MAXN * MAX_CODE_LEN + 1)  /** 字典树的最大节点个数. */
                   /* 如果没有指定MAXN,则是 CHAR_COUNT^(MAX_CODE_LEN+1)-1 */

/** 字典树的节点 */
typedef struct trie_node_t {
    struct trie_node_t* next[CHAR_COUNT];
    bool is_tail; /** 标记当前字符是否位于某个串的尾部 */
} trie_node_t;

/** 字典树. */
typedef struct trie_tree_t {
    trie_node_t *root; /** 树的根节点 */
    int size; /** 树中实际出现的节点数 */

    trie_node_t nodes[MAX_NODE_COUNT]; /* 开一个大数组,加快速度 */
} trie_tree_t;

/** 创建. */
trie_tree_t* trie_tree_create(void) {
    trie_tree_t *tree = (trie_tree_t*)malloc(sizeof(trie_tree_t));
    tree->root = &(tree->nodes[0]);
    memset(tree->nodes, 0, sizeof(tree->nodes));
    tree->size = 1;
    return tree;
}

/** 销毁. */
void trie_tree_destroy(trie_tree_t *tree) {
    free(tree);
    tree = NULL;
}

/** 将当前字典树中的所有节点信息清空 */
void trie_tree_clear(trie_tree_t *tree) {
    memset(tree->nodes, 0, sizeof(tree->nodes));
    tree->size = 1; // 清空时一定要注意这一步!
}

/** 在当前树中插入word字符串,若出现非法,返回false */
bool trie_tree_insert(trie_tree_t *tree, char *word) {
    int i;
    trie_node_t *p = tree->root;
    while (*word) {
        int curword = *word - '0';
        if (p->next[curword] == NULL) {
            p->next[curword] = &(tree->nodes[tree->size++]);
        }
        p = p->next[curword];
        if (p->is_tail) return false; // 某串是当前串的前缀

        word++; // 指针下移
    }

    p->is_tail = true; // 标记当前串已是结尾

    // 判断当前串是否是某个串的前缀
    for (i = 0; i < CHAR_COUNT; i++)
        if (p->next[i] != NULL)
            return false;
    return true;
}
\end{Codex}

\subsection{Immediate Decodebility}


\subsubsection{描述}
An encoding of a set of symbols is said to be immediately decodable if no code for one symbol is the prefix of a code for another symbol. We will assume for this problem that all codes are in binary, that no two codes within a set of codes are the same, that each code has at least one bit and no more than ten bits, and that each set has at least two codes and no more than eight. 

Examples: Assume an alphabet that has symbols \fn{\{A, B, C, D\}}.

The following code is immediately decodable: 
\begin{Code}
A:01 B:10 C:0010 D:0000 
\end{Code}

but this one is not: 
\begin{Code}
A:01 B:10 C:010 D:0000 (Note that A is a prefix of C) 
\end{Code}


\subsubsection{输入}
Write a program that accepts as input a series of groups of records from standard input. Each record in a group contains a collection of zeroes and ones representing a binary code for a different symbol. Each group is followed by a single separator record containing a single 9; the separator records are not part of the group. Each group is independent of other groups; the codes in one group are not related to codes in any other group (that is, each group is to be processed independently).


\subsubsection{输出}
For each group, your program should determine whether the codes in that group are immediately decodable, and should print a single output line giving the group number and stating whether the group is, or is not, immediately decodable.

\subsubsection{样例输入}
\begin{Code}
01
10
0010
0000
9
01
10
010
0000
9
\end{Code}

\subsubsection{样例输出}
\begin{Code}
Set 1 is immediately decodable
Set 2 is not immediately decodable
\end{Code}

\subsubsection{分析}
判断一个串是否是另一个串的前缀,这正是Trie树(即字典树)的用武之地。


\subsubsection{代码}
\begin{Codex}[label=immediate_decodebility.c]
/* POJ 1056 IMMEDIATE DECODABILITY, http://poj.org/problem?id=1056 */

#define CHAR_COUNT  2
#define MAX_CODE_LEN 10
/** 字典树的最大节点个数.
 * 本题中每个code不超过10bit,即树的高度不超过11,因此最大节点个数为2^11-1
 */
#define MAX_NODE_COUNT  ((1<<(MAX_CODE_LEN+1))-1)

/* 等价于复制粘贴,这里为了节约篇幅,使用include,在OJ上提交时请用复制粘贴 */
#include "trie_tree.c"  /* 见“树->Trie树”这节 */

int main() {
    int T = 0;  // 测试用例编号
    char line[MAX_NODE_COUNT]; // 输入的一行
    trie_tree_t *trie_tree = trie_tree_create();
    bool islegal = true;

    while (scanf("%s", line) != EOF) {
        if (strcmp(line, "9") == 0) {
            if (islegal)
                printf("Set %d is immediately decodable\n", ++T);
            else
                printf("Set %d is not immediately decodable\n", ++T);
            trie_tree_clear(trie_tree);
            islegal = true;
        } else {
            if (islegal)
                islegal = trie_tree_insert(trie_tree, line);
        }
    }
    trie_tree_destroy(trie_tree);
    return 0;
}
\end{Codex}


\subsubsection{相关的题目}
与本题相同的题目:
\begindot
\item POJ 1056 IMMEDIATE DECODABILITY, \myurl{http://poj.org/problem?id=1056}
\myenddot

与本题相似的题目:
\begindot
\item POJ 3630 Phone List, \myurl{http://poj.org/problem?id=3630} \\参考代码 \myurl{https://gist.github.com/soulmachine/6609332}
\myenddot


\subsection{Hardwood Species}


\subsubsection{描述}
现在通过卫星扫描,扫描了很多区域的树,并获知了每棵树的种类,求每个种类的百分比。


\subsubsection{输入}
一行一棵树,表示该树的种类。每个名字不超过30字符,树的种类不超过10,000,不超过1,000,000棵树。


\subsubsection{输出}
按字母顺序,打印每个种类的百分比,精确到小数点后4位。

\subsubsection{样例输入}
\begin{Code}
Red Alder
Ash
Aspen
Basswood
Ash
Beech
Yellow Birch
Ash
Cherry
Cottonwood
Ash
Cypress
Red Elm
Gum
Hackberry
White Oak
Hickory
Pecan
Hard Maple
White Oak
Soft Maple
Red Oak
Red Oak
White Oak
Poplan
Sassafras
Sycamore
Black Walnut
Will
\end{Code}

\subsubsection{样例输出}
\begin{Code}
Ash 13.7931
Aspen 3.4483
Basswood 3.4483
Beech 3.4483
Black Walnut 3.4483
Cherry 3.4483
Cottonwood 3.4483
Cypress 3.4483
Gum 3.4483
Hackberry 3.4483
Hard Maple 3.4483
Hickory 3.4483
Pecan 3.4483
Poplan 3.4483
Red Alder 3.4483
Red Elm 3.4483
Red Oak 6.8966
Sassafras 3.4483
Soft Maple 3.4483
Sycamore 3.4483
White Oak 10.3448
Willow 3.4483
Yellow Birch 3.4483
\end{Code}

\subsubsection{分析}
无


\subsubsection{代码}
\begin{Codex}[label=hardwood_species.c]
/* POJ 2418 Hardwood Species, http://poj.org/problem?id=2418 */
#include <stdio.h>
#include <string.h>
#include <stdlib.h>


#define MAXN 1000   /**  no more than 10,000 species,会MLE,因此减一个0 */
#define CHAR_COUNT  128 /** ASCII 编码范围 */
#define MAX_WORD_LEN 30 /** 编码的最大长度. */
#define MAX_NODE_COUNT  (MAXN * MAX_WORD_LEN + 1)  /** 字典树的最大节点个数. */


/** 字典树的节点 */
typedef struct trie_node_t {
    struct trie_node_t* next[CHAR_COUNT];
    int count;  /** 该单词出现的次数 */
} trie_node_t;

/** 字典树. */
typedef struct trie_tree_t {
    trie_node_t *root; /** 树的根节点 */
    int size; /** 树中实际出现的节点数 */

    trie_node_t nodes[MAX_NODE_COUNT]; /* 开一个大数组,加快速度 */
} trie_tree_t;

/** 创建. */
trie_tree_t* trie_tree_create(void) {
    trie_tree_t *tree = (trie_tree_t*)malloc(sizeof(trie_tree_t));
    tree->root = &(tree->nodes[0]);
    memset(tree->nodes, 0, sizeof(tree->nodes));
    tree->size = 1;
    return tree;
}

/** 销毁. */
void trie_tree_destroy(trie_tree_t *tree) {
    free(tree);
    tree = NULL;
}

/** 将当前字典树中的所有节点信息清空 */
void trie_tree_clear(trie_tree_t *tree) {
    memset(tree->nodes, 0, sizeof(tree->nodes));
    tree->size = 1; // 清空时一定要注意这一步!
}

/** 在当前树中插入word字符串 */
void trie_tree_insert(trie_tree_t *tree, char *word) {
    trie_node_t *p = tree->root;
    while (*word) {
        if (p->next[*word] == NULL) {
            p->next[*word] = &(tree->nodes[tree->size++]);
        }
        p = p->next[*word];

        word++; // 指针下移
    }
    p->count++;
    return;
}


int n = 0;  // 输入的行数

/** 深度优先遍历. */
void dfs_travel(trie_node_t *root) {
    static char word[MAX_WORD_LEN + 1]; /* 中间结果 */
    static int pos;  /* 当前位置 */
    int i;

    if (root->count) { /* 如果count不为0,则肯定找到了一个单词 */
        word[pos] = '\0';
        printf("%s %0.4f\n", word, ((float)root->count * 100) / n);
    }
    for (i = 0; i < CHAR_COUNT; i++) {  /* 扩展 */
        if (root->next[i]) {
            word[pos++] = i;
            dfs_travel(root->next[i]);
            pos--; /* 返回上一层时恢复位置 */
        }
    }
}

int main() {
    char line[MAX_WORD_LEN + 1];
    trie_tree_t *trie_tree = trie_tree_create();

    while (gets(line)) {
        trie_tree_insert(trie_tree, line);
        n++;
    }
    dfs_travel(trie_tree->root);

    trie_tree_destroy(trie_tree);
    return 0;
}
\end{Codex}


\subsubsection{相关的题目}
与本题相同的题目:
\begindot
\item POJ 2418 Hardwood Species, \myurl{http://poj.org/problem?id=2418}
\myenddot

与本题相似的题目:
\begindot
\item 无
\myenddot

\section{Expression Tree} %%%%%%%%%%%%%%%%%%%%%%%%%%%%%%
\begin{Codex}
	#include <iostream>
	#include <conio.h>
	using namespace std;
	struct EXTree{
		char data;
		EXTree *l, *r;
	}*root = NULL, *p = NULL, *t = NULL, *y = NULL;
	struct EXTreeNode{
		EXTree *pt;
		EXTreeNode *next;
	}*top = NULL, *q = NULL, *np = NULL;
	void push(EXTree *ptr)
	{
		np = new node;
		np->pt = ptr;
		np->next = NULL;
		if (top == NULL)
		{
			top = np;
		}
		else
		{
			q = top;
			top = np;
			np->next = q;
		}
	}
	EXTree *pop()
	{
		if (top == NULL)
		{
			cout<<"underflow\n";
		}
		else
		{
			q = top;
			top = top->next;
			return(q->pt);
			delete(q);
		}
	}
	void oprnd_str(char val)
	{
		if (val >= 48 && val <= 57)
		{
			t = new tree;
			t->data = val;
			t->l = NULL;
			t->r = NULL;
			push(t);
		}
		else if (val >= 42 && val <= 47)
		{
			p = new tree;
			p->data = val;
			p->l = pop();
			p->r = pop();
			push(p);
		}
	}
	char pstorder(EXTree *w)
	{
		if (w != NULL)
		{
			pstorder(w->l);
			pstorder(w->r);
			cout<<w->data;
		}
	}
	int main()
	{
		char a[15];
		int i;
		int j = -1;
		cout<<"enter the value of character string\n";
		cin>>a;
		i = strlen(a);
		while (i >= 0)
		{
			i--;
			oprnd_str(a[i]);
		}
		cout<<"displaying in postorder\n";
		pstorder(pop());
		getch();
	}
	
	
	Output:
	enter the value of character string
	-+-5/763*48
	displaying in postorder
	576/-3+48*-
\end{Codex}

\section{树状数组} %%%%%%%%%%%%%%%%%%%%%%%%%%%%%%
对于一个数组A[1..n],在O(logn)的时间内完成以下操作:

(1) 给A[i]加一个数

(2) 求A[1]+A[2]+...+A[i]的和

\section{左偏树}
实现一个最小优先队列,使得插入、删除、合并等操作均在O(logn)的时间复杂度。

树状数组是一个优美小巧的数据结构,在很多时候可以代替线段树。一句话概括就是,凡是树状数组可以解决的问题,线段树都可以解决,反过来线段树可以解决的问题,树状数组不一定能解决。

\section{Treap}
动态维护一个有序表,支持在O(logn)的时间内完成插入一个元素,删除一个元素和查找第K大元素的任务。

\section{伸展树Splay}
实现能够在O(logn)时间复杂度内实现各类二叉树操作的数据结构。

\section{ST表}
给定一个数组A[n],动态查询数组元素A[l],A[l+1],...,A[r]的最小值。
\section{动态树}

\section{可并堆}